\section{Точки множества}

\subsection{Точка относительно множества}

\begin{definition}[Окрестность точки]
	Пусть \((X, \Omega)\) -- топологическое пространство, \(x \in X\) -- точка. Окрестностью точки \(x\) называется любое открытое множество \(U \in \Omega\), такое что \(x \in U\).
\end{definition}

Аналитики и представители французской школы математики (по наследию Н. Бурбаки) понимают понятие окрестности шире, чем принято в классической традиции: они называют окрестностью любое множество, включающее в себя окрестность в более узком, традиционном смысле.\footnote{Строчка памяти Олега Александровича Иванова (1951-2019), выдающегося математика, профессора, кандидата физико-математических наук, заведующего кафедрой общей информатики мехмата СПБГУ}
\begin{definition}[Типы точек множества]
	В любом множестве всегда есть следующие точки:
	\begin{enumerate}
		\item \textbf{Внутренняя точка \(x\) множества \(A\):} Точка \(x\) является внутренней точкой множества \(A\), если существует окрестность \(U\) точки \(x\), целиком содержащаяся в \(A\), то есть \(U \subseteq A\).
		
		\item \textbf{Точка прикосновения \(x\) множества \(A\):} Точка \(x\) является точкой прикосновения множества \(A\), если для любой окрестности \(U\) точки \(x\) выполнено \(U \cap A \neq \emptyset\).
		
		\item \textbf{Граничная точка \(x\) множества \(A\):} Точка \(x\) является граничной точкой множества \(A\), если для любой окрестности \(U\) точки \(x\) выполнено \(U \cap A \neq \emptyset\) и \(U \cap (X \setminus A) \neq \emptyset\).
	\end{enumerate}
\end{definition}

\subsection{Операции на множествах}

Пусть \( (X, \Omega)\) -- топологическое пространство, а \( A \subseteq X \) -- произвольное подмножество.
\begin{definition}[Внутренность множества]
	Внутренностью множества \( A \) называется наибольшее (по включению) открытое множество:
	\begin{equation*}
		\mathrm{Int}(A) = \bigcup_{U \subseteq A} U, \quad U \open \Omega.
	\end{equation*}
		Внутренность множества \( A \) обозначается символом \( \text{Int}( A) \).
\end{definition}

\begin{remark}
	\(\text{Int} (A) \) происходит от французского слова "intérieur" и английского "interior"\ , что аналогично русскому слову "интерьер" \ , которое обозначает вид изнутри на какой-либо объект.
\end{remark}

Внешностью множества называется наибольшее не пересекающиеся с ним открытое множество. Ясно, что внешность $A$ совпадает с $\mathrm{Int}(X \setminus A)$

\begin{definition}[Замыкание множество]
	 Замыканием множества \( A \) называется наименьшее содержащее его замкнутое множество:
	 \begin{equation*}
		\mathrm{Cl}(A) = \bigcap_{F \supseteq A} F, \quad F \closed \Omega.
	 \end{equation*}
	Замыкание множества \( A \) обозначается символом \( \cl{A} \).
\end{definition}


\begin{remark}
\( \text{Cl}_X \, A \) происходит от французского слова "clôture" и английского "closure"\ , что аналогично русскому слову "ограждение"\ , которое обозначает предел, ограничивающий или замкнувший пространство.
\end{remark}



\begin{definition}[Граница множества]
	Границей множества \( A \) называется множество \( \cl{A} \setminus \mathrm{Int}(A) \). Обозначается граница множества \( A \) символом \( \fr{A} \) или \(\partial A\).
\end{definition}

\begin{remark}
\( \text{Fr}_X \, A \) происходит от французского слова "frontière" и английского "frontier"\ , что аналогично русскому слову "фронт"\ , которое обозначает передовую линию или границу между различными территориями или областями.
\end{remark}

\begin{theorem}[Эквивалетные опредления через точки множества]

	Для связи точек и элементов множества запишем следующее:

	\begin{enumerate}
		\item Внутренностью всякого множества является множество его внутренних точек.
		
		\item Замыкание множества \(A\) совпадает с множеством его точек прикосновения.
		
		\item Граница множества совпадает с множеством его граничных точек.
	\end{enumerate}
	
\end{theorem}

\begin{proof}
	\begin{enumerate}
		\item Пусть точка \( x \) внутренняя, т.е. существует открытое множество \( U_x \) такое, что \( x \in U_x \subseteq A \). Тогда \( U_x \subseteq \text{Int} \, A \) (поскольку \( \text{Int} \, A \) - наибольшее из всех открытых множеств, содержащихся в \( A \)), а, значит, и \( x \in \text{Int} \, A \). Обратно, если \( x \in \text{Int} \, A \), то само множество \( \text{Int} \, A \) и есть содержащаяся в \( A \) окрестность точки \( x \).
		\item Если \( x \notin \text{Cl} \, A \), то найдётся такое замкнутое множество \( F \), что \( F \supseteq A \) и \( x \notin F \). Значит, \( x \notin \text{Int} \, (X \setminus F) \), таким образом, \( x \) не является точкой прикосновения множества \( A \).
		\item Если \( x \in \text{Fr} \, A \), то \( x \in \text{Cl} \, A \setminus \text{Int} \, A \). Значит, во-первых, всякая окрестность точки \( x \) пересекается с \( A \), во-вторых, никакая окрестность этой точки не содержится в \( A \), следовательно, всякая окрестность пересекается с дополнением множества \( A \). Таким образом, точка \( x \) является граничной.
	\end{enumerate}
\end{proof}

\subsection{Всюду плотное и нигде не плотное множество}

\begin{definition}[Всюду плотное множество]
	Пусть \( A \) и \( B \) — подмножества топологического пространства \( X \). Говорят, что \( A \) плотно в \( B \), если 
\[
\text{Cl}(A) \supseteq B,
\]
и что \( A \) всюду плотно, если 
\[
\text{Cl}(A) = X.
\]
\end{definition}

Если множество \( A \) плотно в \( B \), это означает, что любой элемент из \( B \) можно аппроксимировать элементами из \( A \). Это свойство используется при работе с функциональными пространствами и в численных методах. Например, идея плотности множества функций в пространстве непрерывных функций лежит в основе различных разложений, таких как ряды Тейлора, тригонометрические ряды Фурье. Все эти разложения базируются на интуиции, что соответствующие множества функций являются всюду плотными в пространстве \( C[a, b] \).

\begin{theorem}[Критерий всюду плотного множества]
	Множество \( A \subseteq X \) является всюду плотным в топологическом пространстве \( X \) тогда и только тогда, когда для любого непустого открытого множества \( U \subseteq X \) выполняется \( A \cap U \neq \emptyset \).
\end{theorem}
\begin{proof} \
	\begin{itemize}
		\item[\(\boxed{\Longrightarrow}\)] Для каждой точки \( x \in X \) существует непустое открытое множество \( U \subseteq X \), такое что \( x \in U \). Поскольку \( A \) всюду плотно, то \( A \cap U \neq \emptyset \), что означает, что \( x \in \cl{A} \). Следовательно, \( \cl{A} = X \).
		\item[\(\boxed{\Longleftarrow}\)]  Если \( A \) пересекается с каждым непустым открытым множеством, то \( A \subseteq \cl{A} = X \). 
	\end{itemize}
\end{proof}

\begin{definition}[Нигде не плотное множество]
	Множество \( A \subseteq X \) называется нигде не плотным, если его внешность всюду плотна.
\end{definition}

\begin{theorem}[Эквивалетное определение]
	Множество \( A \subseteq X \) называется нигде не плотным, если и только если для каждой точки \( x \in X \) и для каждой окрестности \( U \) точки \( x \) существует точка \( y \in U \setminus A \), такая что некоторая окрестность точки \( y \) полностью содержится в \( U \setminus A \).
\end{theorem}
\begin{proof}
	Рассмотрим произвольную точку \( x \in X \) и её окрестность \( U \subseteq X \). Согласно условию, в любой окрестности \( U \) точки \( x \) существует точка \( y \in U \setminus A \), такая что для неё существует окрестность \( V \), удовлетворяющая \( V \subseteq U \setminus A \). Таким образом, для любой окрестности \( U \) точки \( x \) существует точка \( y \), которая находится в дополнении множества \( A \) с некоторой окрестностью. Это означает, что \( \text{Int}(\cl{A}) = X \), и следовательно, внешность множества \( A \) всюду плотна.
\end{proof}

\bigskip
\starsection{Задачи и упражнения}

\begin{task}[Свойства функции внутренности]
	Докажите, что для любой топологии на множестве \( X \) функция, задаваемая правилом: 
\[
f(A) = \text{Int}(A),
\]  
где \( A \subseteq X \), является:

\begin{enumerate}
	\item \textbf{идемпотентной}, то есть для любого множества \( A \subseteq X \) выполняется равенство:  
	\[
	\text{Int}(\text{Int}(A)) = \text{Int}(A).
	\]	
	\item \textbf{монотонной}, то есть для любых множеств \( A, B \subseteq X \), таких что \( A \subseteq B \), выполняется неравенство:  
	\[
	\text{Int}(A) \subseteq \text{Int}(B).
	\]	
	\item \textbf{супераддативой} относительно объединения, то есть для любых множеств \( A, B \subseteq X \) выполняется неравенство:  
	\[
	\text{Int}(A \cup B) \supseteq \text{Int}(A) \cup \text{Int}(B).
	\]  
	\item \textbf{мультипликативной} относительно пересечения, то есть для любых множеств \( A, B \subseteq X \) выполняется равенство:  
	\[
	\text{Int}(A \cap B) = \text{Int}(A) \cap \text{Int}(B).
	\]
\end{enumerate}
\end{task}
\begin{task}
	Аналогично, проведите анализ свойств функции \( \text{Cl}(A) \), определяющей замыкание множества \( A \subseteq X \). 
\end{task}



\begin{task}[Задача замыкания и дополнения]
	Найдите максимальное количество различных множеств, которые можно получить из одного множества \( A \subseteq X \), применяя только операции замыкания (\( \text{Cl} \)) и внутренности (\( \text{Int} \)) в различных комбинациях.  
\end{task}
Задача замыкания и дополнения, также известная как задача Куратовского\footnote{Казимеж Куратовский (1896–1980) — польский математик, основоположник современной топологии. В 1921 году он получил степень доктора философии за свою работу, в которой аксиоматически построил топологию через аксиомы замыкания. Его диссертация, опубликованная в 1922 году, оказала значительное влияние на развитие топологии и теории множеств.}, является классическим упражнением в общей топологии.

\begin{task}[О всюду плотном множестве]
	Доказать, что множество \(\mathbb{Q}\) всюду плотно \(\mathbb{R}\).
\end{task}

\begin{task}[О нигде не плотном множестве]
	Показать, что прямая \(\mathbb{R}\) нигде не плотна на плоскости \(\mathbb{R}^2\).
\end{task}
\begin{task}
    Рассмотрите множество Кантора \( \mathcal{C}\). Подумайте, что является внутренностью этого множества, а что является замыканием.
\end{task}
