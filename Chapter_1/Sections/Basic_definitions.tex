\section{Топология. Основные понятия}
\subsection{Первичное определение}
\begin{definition}[Топология]
	Топология на множестве $X$ -- это семейство подмножеств $\Omega$ множества $X$, удовлетворяющее следующим условиям:
	\begin{enumerate}
		\item Пустое множество и само множество $X$ принадлежат $\Omega$: $\emptyset, X \in \Omega$.
		\item Любое объединение элементов из $\Omega$ также принадлежит $\Omega$: если $A_\alpha \in \Omega$ для всех $\alpha$ из некоторого индексного множества, то $\bigcup_\alpha A_\alpha \in \Omega$.
		\item Любое пересечение конечного числа элементов из $\Omega$ также принадлежит $\Omega$: если $A_1, A_2, \ldots, A_n \in \Omega$, то $A_1 \cap A_2 \cap \ldots \cap A_n \in \Omega$.
	\end{enumerate}
\end{definition}

Так как большая часть последующих размышлений посвящена топологическим пространствам, то часто в дальнейшем мы будем опускать пару множество-топология и ограничимся лишь -- $X$.

\begin{definition}[Топологическое пространство]
	Топологическим пространством называется упорядоченная пара $(X, \Omega)$, где $X$ -- произвольное множество, а $\Omega$ -- топология на $X$.
\end{definition}

\begin{example}
	\begin{enumerate}
		
		\item Стандартная топология на вещественной числовой прямой $\mathbb{R}$ определяется множеством всех открытых интервалов. Это может быть записано следующим образом:

		\[
		\Omega_{\text{std}} = \{ (a, b) \mid a, b \in \bar{\mathbb{R}}, a < b \} \cup \mathbb{R}
		\]

		\item Пусть $X$ — произвольное множество. Дискретная топология на $X$ определяется следующим образом:  
		\[
		\Omega_{\text{disc}} = 2^X,
		\]  
		где $2^X$ обозначает булеан множества $X$, то есть множество всех подмножеств множества $X$.  
		

		\item Антидискретная топология на множестве \(X\) определяется следующим образом:

		\[
		\Omega_{\text{antidisc}} = \{ \varnothing, X\}
		\]
		
		\item Пусть \(X\) -- произвольное множество, а \(d\) -- метрика на \(X\). Метрическая топология на \(X\) порождается метрикой \(d\) следующим образом:

		\[
		\Omega_d = \{ B(x, r) \mid x \in X, r > 0 \}
		\]
		
		где \(B(x, r)\) обозначает открытый шар с центром в точке \(x\) и радиусом \(r\).
		
		\item  Топология Зарицкого на множестве \(\mathbb{R}\), обозначаемая \(\Omega_{\text{zar}}\), состоит из подмножеств \(A\) множества \(\mathbb{R}\), таких что \(A = \emptyset\) или \(X \setminus A\) является конечным множеством. Формально:

		\[
		\Omega_{\text{zar}} = \{ A \subseteq \mathbb{R} \mid A = \emptyset \text{ или } X \setminus A \text{ конечно} \}
		\]

		\item Топология двоеточие Александрова или же топология Серпинского на множестве \(X = \{\circ, \bullet\}\), обозначаемая \(\Omega_{:A}\), где \(\circ\) -- открыто:
		\[
			\Omega_{:A} = \{\varnothing, \circ, \{\bullet, \circ\}\}
		\]

		\item Топология Зоргенфрея или же топология стрелки на множестве \(X = [a, +\infty)\), \(a \geq 0\):
		\[
			\Omega_{l} = \{\varnothing, X\} \cup \{(a, +\infty): \ \forall a \geq 0 \in \mathbb{R}\}
		\]
		\item Пусть $(X, \Omega)$ -- топологическое пространство, а $Y$ -- множество, полученное из $X$ добавлением к нему одного элемента $p$, если \(\Omega_x\) -- дискретна, то следующая топология называется топологией всюду плотной точки:
		\[
			\Omega_p = \left\{ \{p\} \cup U \ \middle| \ U \in \Omega \right\} \cup \{ \emptyset \}	
		\]
	\end{enumerate}
\end{example}

\subsection{Рассуждение о типах множеств}

Если $(X, \Omega)$ -- топологическое пространство, то элементы $X$ называются точками, а элементы множества $\Omega$ -- открытыми множествами. Открытое множество $U \in \Omega$  будем обозначать $U \open X$.

\( U \) для открытых множеств происходит от немецкого слова "Umgebung"\ , что в переводе означает "окрестность". Это объясняется тем, что несколько выдающихся немецких математиков, таких как Георг Кантор и Феликс Хаусдорф, сыграли важную роль в развитии топологии. 


\begin{remark}
	Греческая буква \(\Omega\) -- это аналог буквы О, используемый в различных языках для обозначения одного и того же понятия. Например, в английском это "open"\ , в русском "открыто"\ , в немецком "offen" \ , во французском "ouvert" \ .
\end{remark}

\begin{definition}[Замкнутное множество]
	Говорят, что множество \(F \subseteq X\) замкнуто в пространстве \((X, \Omega)\), если его дополнение \(X \setminus F\) открыто, то есть если \(X \setminus F \in \Omega\). Будем обозначать следующим образом: $F \closed X$.
\end{definition}

Буква \( F \), происходящая от французского слова "fermé" \ (замкнутое), традиционно используется для обозначения замкнутых множеств. 


\begin{remark}
	Обратите внимание на то, что замкнутость не есть отрицание открытости.
\end{remark}

\begin{example}[Множетсво, которое ни открыто, ни замкнуто]
Рассмотрим множество вещественных чисел $\mathbb{R}$ с обычной топологией. Пусть $A = [0, 1)$.  

Заметим, что множество $A$ не является открытым, так как точка $0$ не имеет окрестности, содержащейся в $A$. Однако $A$ также не является замкнутым, так как его дополнение $\mathbb{R} \setminus A = (-\infty, 0) \cup [1, \infty)$ не является открытым (точка $1$ не имеет окрестности, содержащейся в дополнении).  

Таким образом, $A$ является примером множества, которое ни открыто, ни замкнуто, что иллюстрирует, что замкнутость не равнозначна отрицанию открытости.

\end{example}

Замкнутость и открытость во многом аналогичные свойства. Фундаментальное различие между ними состоит в том, что пересечение бесконечного набора открытых множеств не обязательно открыто, тогда как пересечение любого набора замкнутых множеств замкнуто, а объединение бесконечного набора замкнутых множеств не обязательно замкнуто, тогда как объединение любого набора открытых множеств открыто.

\subsection{Иерархия топологий}
\begin{definition}[Иерархия тополгий]
	Пусть \( \Omega_1 \) и \( \Omega_2 \) — топологические структуры на множестве \( X \), причём \( \Omega_1 \subseteq \Omega_2 \). Тогда говорят, что структура \( \Omega_2 \) \textbf{тоньше} (или \textbf{тоньше, чем}) структура \( \Omega_1 \), а структура \( \Omega_1 \) \textbf{толще} (или \textbf{грубее, чем}) структура \( \Omega_2 \).
\end{definition}


\bigskip

\begin{example}
	Рассмотрим множество \( X = \{a, b, c\} \). В этом множестве можно задать следующие топологии:

	\begin{enumerate}
		\item Антидискретная топология: \( \Omega_{\text{грубая}} = \{\varnothing, X\} \).
		\item Промежуточная топология: \( \Omega_{\text{средняя}} = \{\varnothing, \{a\}, X\} \).
		\item Дискретная топология: \( \Omega_{\text{тонкая}} = 2^X \), где \( 2^X \) —- булеан множества \( X \), содержащий все его подмножества.
	\end{enumerate}
	
	Здесь \(\Omega_{\text{грубая}} \subseteq \Omega_{\text{средняя}} \subseteq \Omega_{\text{тонкая}}\), то есть грубая топология является самой слабой в терминах различения подмножеств, а тонкая — самой сильной.	
\end{example}
\bigskip

\begin{remark}
	Стоит отметить, что термины «грубая» и «тонкая» топологии иногда воспринимаются неоднозначно. Некоторые интерпретируют грубую топологию как «максимально простой подход», а тонкую — как «максимально сложный и детализированный». Однако такая интерпретация не совсем точна. Грубая топология задаёт минимальную структуру, обеспечивая самые общие свойства, например, для изучения связности или компактности. Тонкая топология, напротив, отличается высокой детализацией, позволяя учитывать особенности каждого элемента множества.  
\end{remark}

\begin{example}
	На пространстве функций топология равномерной сходимости является сильнее топологии поточечной сходимости: она задаёт более тонкую структуру, требуя согласованности сходимости на всех точках одновременно, а не только на отдельных точках. 
\end{example}

\begin{definition}
	Если на множестве заданы топологии \(\Omega_1\) и \(\Omega_2\), при этом \(\Omega_1\) не слабее или не сильнее \(\Omega_2\), то говорят, что \(\Omega_1\) и \(\Omega_2\) \textbf{несравнимы}.
\end{definition}

\starsection{Задачи и упражнения}

\begin{task}
	Определите, является ли заданное множество подмножеств топологией на соответствующем множестве:  
\begin{enumerate}
    \item На $X = \{a, b, c\}$ задано $\Omega = \{\emptyset, \{a\}, \{a, b, c\}\}$.  
    \item На $X = \mathbb{R}$ задано $\Omega = \{\emptyset, \mathbb{R}, (-\infty, a], [b, +\infty) \mid a, b \in \mathbb{R}\}$.  
    \item На $X = \mathbb{Z}$ пусть задано семейство $\Omega$ -- объединение всех конечных подмножеств целых чисел, включая само множество $\mathbb{Z}$.  
\end{enumerate}  
\end{task}

\begin{task}
	Рассмотрите следующие топологии на множестве $X = \mathbb{R}$:  
	\begin{enumerate}
		\item Дискретная топология.  
		\item Топология Зарицкого.  
		\item Стандартная топология.  
		\item Антидискретная топология.
	\end{enumerate}
	Упорядочите эти топологии.
\end{task}
	

\begin{task}[Определение топологии через замкнутые множества]
	Переформулируйте определение топологического пространства в терминах замкнутых множеств.  \\ 
	\textbf{Указание:} Воспользуйтесь законами Август де Морган.
\end{task}


\begin{definition}[Канторово множество]
	Канторово множество — это множество, полученное из отрезка $[0,1]$ последовательным удалением средних третьей части каждого оставшегося интервала. Формально:  
\[
\mathcal{C} = [0,1] \setminus \bigcup_{n=1}^\infty \bigcup_{k=1}^{2^{n-1}} \left( \frac{3k-2}{3^n}, \frac{3k-1}{3^n} \right).
\]  

\end{definition}

\begin{task}
	Докажите, что Канторово множество $\mathcal{C}$ замкнуто.  
\end{task}

