
\newpage
\section{Аксиомы топологии}

\subsection{Аксиомы отделимости}
В данном параграфе речь пойдет о первом топологическом инварианте, связанном с тем, как мы можем отделить какие-то множества или же в простейших случаях точки друг от друга.
\begin{enumerate}
	\item \textbf{Т0-аксиома (Колмогоров)}: Для любых двух различных точек \( x \) и \( y \) из множества \( X \) существует открытое множество \( U \) такое, что либо \( x \in U \), \( y \notin U \), либо \( y \in U \), \( x \notin U \).
    \item \textbf{Т1-аксиома (Тихонова-Фреше)}: Для любых двух различных точек \( x \) и \( y \) из множества \( X \) существуют открытые множества \( U \) и \( V \), такие что \( x \in U \), \( y \notin U \), \( y \in V \), \( x \notin V \).
    
    \item \textbf{Т2-аксиома (Хаусдорф)}: Для любых двух различных точек \( x \) и \( y \) из множества \( X \) существуют открытые множества \( U \) и \( V \), такие что \( x \in U \), \( y \in V \) и \( U \cap V = \varnothing \).
    
    \item \textbf{Т3-аксиома (Регулярность)}: Для любой замкнутой множества \( A \) и точки \( x \notin A \) существуют открытые множества \( U \) и \( V \), такие что \( x \in U \), \( A \subseteq V \) и \( U \cap V = \varnothing \).
    
    \item \textbf{Т4-аксиома (Нормальность)}: Для любых двух непересекающихся замкнутых множеств \( A \) и \( B \) существуют открытые множества \( U \) и \( V \), такие что \( A \subseteq U \), \( B \subseteq V \) и \( U \cap V = \varnothing \).
\end{enumerate}

\begin{statement}[О замкнутости графика]
	Пусть \(f: \ X \to Y\) -- непрерывное отображение, причём \(Y\) -- хаусдорфово. Тогда график этой функции \(\Gamma_f = \{(x, f(x)): \ x \in X\}\) замкнуто в \(X \times Y\).
\end{statement}
\begin{proof}
	Рассмотрим произвольные точки $x \in X$ и $y \in Y$ такие, что $y \neq f(x)$.
	В силу Хаусдорфовости пространства $Y$ найдутся непересекающиеся окрестности $U$ и $V$ точек $y$ и $f(x)$ соответственно. Тогда из непрерывности отображения $f^{-1}(V) \open X$, то есть множество \(\{(x, y): y \neq f(x)\}\) открыто в произведении. А значит дополнение к нему, то есть \(\Gamma_f\) замкнуто.
\end{proof}

\begin{statement}[Критерий Хаусдорфовости]
	\((X, \Omega_x)\) -- Хаусдорфово тогда и только тогда, когда диагональ \(\Delta_x = \{(x, x): x \in X\} \closed X\times X\).
\end{statement}

\begin{proof} \
	\begin{itemize}
		\item[ \(\boxed{\Longrightarrow}\)] Достаточно показать, что \((X \times X) \setminus \Delta\) -- открыто. Возьмём пару \(p_1, p_2\) из \((X \times X) \setminus \Delta\), тогда известно, что \(p_1 \neq p_2\). \(X\) -- хаусдорфово, следовательно существуют окрестности \(U_1, \ U_1\) точек \(p_1, \ p_2\), соответственно, такие что \(U_1 \cap U_2 = \varnothing\). \(U_1 \times U_2\) -- открыто. \(U_1 \times U_2 \subseteq X \times X\) и \(U_1 \times U_2 \cap \Delta_x = \varnothing\), следовательно \((p_1, p_2) \in U_1 \times U_2 \subseteq \left(X \times X\right) \setminus \Delta_X\), отсюда \(\left(X \times X\right) \setminus \Delta_X\) открыто.
		\item[ \(\boxed{\Longleftarrow}\)]  \((X \times X) \setminus \Delta_X\) -- открыто. Пусть \(x_1 \neq x_2 \in X\), тогда \((x_1, x_2) \in (X, X)\setminus \Delta_X\). Существуют открытые множества \((U_1 \times U_2) \in X \times X\), такие что \((x_1, x_2) \in U_1 \times U_2 \subseteq (X \times X) \setminus \Delta_x\). Следовательно, \(x_1 \in U_1\), \(x_2 \in U_2\) и \(U_1 \cap U_2 = \varnothing\). Отсюда следует Хаусдоровость. 
	\end{itemize}
\end{proof}


\subsection{Аксиомы счётности}
Для топологического пространства \( X \) могут выполняются следующие аксиомы счётности:
\begin{enumerate}
    \item \textbf{Первая аксиома счётности}: Для каждой точки \( x \) пространства \( X \) существует база топологии, состоящая из счётного числа открытых множеств, содержащих \( x \).
    
    \item \textbf{Вторая аксиома счётности}: Существует не более чем счётная база $\mathcal{B}$ пространства $X$.
    \end{enumerate}

\begin{example}
	
\begin{enumerate}
    \item \textbf{Дискретное пространство}: Все подмножества множества \(X\) являются открытыми. Базой этого пространства являются одноточечные множества \( \{x\} \), где \(x\) пробегает всё множество \(X\).
    
    \item \textbf{Стандартная топология на множестве натуральных чисел \( \mathbb{N} \)}: Открытыми множествами являются конечные подмножества и все подмножества, дополнения которых конечны. Базой этой топологии являются множества всех конечных последовательностей натуральных чисел.
    
    \item \textbf{Топология Канторова множества}: Множество Кантора получается удалением средней трети каждого отрезка на каждом шаге процесса. Открытыми множествами в этой топологии являются все открытые интервалы и их дополнения. Базой этой топологии являются интервалы вида \((a, b)\), где \(a\) и \(b\) - рациональные числа.
\end{enumerate}

\end{example}

\begin{definition}[Покрытие]
	\textbf{Покрытием} пространства \( X \) называется семейство открытых подмножеств \( \mathcal{U} \), объединение которых содержит все точки пространства \( X \), то есть:

\[
X = \bigcup_{U \in \mathcal{U}} U
\]
\end{definition}

\begin{theorem}[Линделёфа]
	Пусть \( X \) -- топологическое пространство со второй аксиомой счётности, тогда из любого открытого покрытия \( \mathcal{U} \) пространства \( X \) можно выделить счётное подпокрытие.
\end{theorem}
\begin{proof}
	Пусть \(\mathcal{B} = \{V_k\}_{k \in \mathbb{N}}\) -- счётная база из 2 аксиомы счётности и \(\{U_i\}_{i \in \mathbb{I}}\) -- некоторое покрытие \(X\).

	Для каждого \(V_k\) выберем такое множество \(U_{i(k)}\) из покрытия, что \(V_k \subset U_{i(k)}\). Тогда \(X = \bigcup_{k \in \mathbb{N}} V_k \subseteq \bigcup_{k \in \mathbb{N}} U_{i(k)}\). При этом также \(\bigcup_{k \in \mathbb{N}} U_{i(k)} \subseteq X\). А значит на самом деле они равны, то есть \(\bigcup_{k \in \mathbb{N}} U_{i(k)}\) -- необходимое подпокрытие.
\end{proof}

