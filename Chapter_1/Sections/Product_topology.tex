\section{Топология произведения, подпространство}
\subsection{Топология подпространства}

\begin{definition}[Индуцированная топология]
	Пусть \((X, \Omega)\) — топологическое пространство, а \(S \subset X\) — подмножество \(X\). \textbf{Индуцированная топология} (или подпространственная топология) на \(S\) определяется как множество всех пересечений открытых множеств в \(X\) с \(S\):
\[
\Omega_S = \{ S \cap U \mid U \in \Omega \}.
\]
Это означает, что множество \(V \subset S\) является открытым в индуцированной топологии, если существует множество \(U \open \Omega\) такое, что \(V = S \cap U\).
\end{definition}

Индуцированная топология вводится для того, чтобы наделить подмножество топологического пространства собственной топологией, согласованной с исходным пространством. Эта концепция позволяет нам рассматривать подмножество как самостоятельное топологическое пространство, сохраняя при этом естественную связь с исходным. Интуитивно, индуцированная топология обеспечивает минимальное количество открытых множеств в подмножестве, чтобы топологическая структура оставалась совместимой с исходной топологией.

\begin{example}
	Рассмотрим вещественную прямую \(\mathbb{R}\) с её топологией \(\Omega_{\text{std}}\). Пусть \(S = [0, 1]\) — замкнутый отрезок на \(\mathbb{R}\). Индуцированная топология \(\Omega_S\) на \(S\) состоит из множеств вида \(S \cap U\), где \(U \open \mathbb{R}\). Например:
	\begin{enumerate}
		\item Если \(U = (0.5, 0.7)\), то \(S \cap U = (0.5, 0.7)\).
		\item Если \(U = (0, 2)\), то \(S \cap U = (0, 1]\).
		\item Если \(U = (-1, 0.5)\), то \(S \cap U = [0, 0.5)\).
		\item Если \(U = (-2, 1)\), то \(S \cap U = [0, 1]\).
	\end{enumerate}
	
Таким образом, открытые множества в \([0, 1]\) в индуцированной топологии могут быть промежутками любого вида и их объединениями, где \(0 \leq a < b \leq 1\).

\end{example}

Из приведённого примера видно, что открытые множества в индуцированной топологии не всегда являются открытыми в исходном пространстве. 

Однако для замкнутых множеств ситуация другая: замкнутые множества в индуцированной топологии всегда остаются замкнутыми в исходном пространстве. 

\begin{statement}
	Пусть \( (X, \Omega) \) — топологическое пространство, \( S \subset X \), а \( \Omega_S \) — индуцированная топология на \(S\). Тогда:
	\begin{enumerate}
		\item \(F \closed S \iff F = S \cap E\), где \(E \closed X\).
		\item Если \(S \open X \Longrightarrow \Omega_S \subset \Omega_X\).
	\end{enumerate}
\end{statement}
% \begin{proof}
% 	\begin{enumerate} 
% 		\item \ \begin{itemize}
% 			\item[\(\boxed{\Longrightarrow}\)] 
% 			\item[\(\boxed{\Longleftarrow}\)]
% 		\end{itemize}
% 		\item 
% 	\end{enumerate}
% \end{proof}

\subsection{Топология коробки}

Выше мы рассмотрели идею, как можно углубиться внутрь исходного пространства с помощью индуцированной топологии. Мы изучили, как выделить подпространство и определить на нём новую топологию, основываясь на открытых множествах объемлющего пространства. Этот процесс позволяет детально изучить внутреннюю структуру подпространства.

Однако возникает вопрос: можем ли мы расшириться вверх? Можем ли мы, как говорят в линейной алгебре, повысить размерность пространства? На самом деле, это возможно, если ввести понятие топологии произведения.

\begin{definition}[Топология произведения]
	Пусть \(\displaystyle X = \prod_{i \in I} X_i\) -- декартово произведение топологических пространств \(X_i\), индексированных множеством \(i \in I\). \textbf{Топология коробки} (топология произведения) на \(X\) задаётся базой:
		\[
		\mathcal{B} = \left\{ \prod_{i \in I} U_i \mid U_i \open X_i \right\}.
		\]
\end{definition}
Эта топология позволяет изучать свойства пространств более высокой размерности и исследовать их взаимосвязи через произведение координатных пространств. Интересное наблюдение, что	Декартову прямоугольную систему координат можно естественным образом представить как произведение двух координатных пространств \(X\) и \(Y\). В данном случае пространство \(X\) соответствует оси абсцисс, а пространство \(Y\) — оси ординат.
\begin{remark}
	Название \textit{топология коробки} происходит от случая, когда рассматривается пространство \(\mathbb{R}^n\), где элементы базы выглядят как прямоугольные "коробки".
\end{remark}


\begin{remark}
	Топология коробки является самой интуитивной топологией произведения, поскольку она естественно возникает при рассмотрении множества всех декартовых произведений открытых множеств. Однако, это не единственная возможная топология для произведения топологических пространств. 

Существует другая важная топология, известная как \textbf{Тихоновская топология}, в отличие от топологии коробки, она применяется для .
\end{remark}

\starsection{Задачи и упражнения}
\begin{task}
	Рассмотрим множество натуральных чисел \(\mathbb{N}\) как подмножество вещественной прямой \(\mathbb{R}\) с её стандартной топологией. Докажите, что подпространственная топология на \(\mathbb{N}\) является дискретной. 
\end{task}

\begin{task}

	Рассмотрим множество рациональных чисел \(\mathbb{Q}\) как подмножество вещественной прямой \(\mathbb{R}\) с её стандартной топологией.
	\begin{enumerate}
		\item Покажите, что синглетон \(\{0\}\) не является открытым в подпространственной топологии на \(\mathbb{Q}\).
		\item Докажите, что если \(a, b \in \mathbb{Q}\), то интервалы \((a, b)\) и \([a, b]\) соответственно открыты и замкнуты в \(\mathbb{Q}\).
		\item Покажите, что если \(a, b \in \mathbb{R} \setminus \mathbb{Q}\), то множество всех рациональных чисел \(x\), таких что \(a < x < b\), одновременно открыто и замкнуто в \(\mathbb{Q}\).
	\end{enumerate}
\end{task}