
\section{Отображения между топологическими пространствами}
\subsection{Определения непрерывного отображения}
Пусть \( X \) и \( Y \) — топологические пространства. 

\begin{definition}[Непрерывное отображение]
	Для отображения \( f : X \rightarrow Y \) между топологическими пространствами \( X \) и \( Y \), \( f \) называется \textbf{непрерывным} в точке \( x \in X \), если для любой окрестности \( V \) точки \( f(x) \) в \( Y \) существует окрестность \( U \) точки \( x \) в \( X \), такая что \( f(U) \subset V \). 

	Если отображение \( f \) непрерывно в каждой точке \( x \in X \), то оно называется \textbf{непрерывным} на \( X \).
\end{definition}

\begin{statement}[Эквивалентные определения]
	Для отображения \( f : X \rightarrow Y \) между топологическими пространствами следующие условия эквивалентны:
	\begin{enumerate}
		\item \(f\) — непрерывное.
		\item Для любого открытого множества \( U \subset Y \), \( f^{-1}[U] \open X \).
		\item Для любого замкнутого множества \( F \subset Y \), \( f^{-1}[F]\closed X \).
		\item Для любого множества \( M \subset X \) выполняется \( f(\mathrm{Cl}(M)) \subseteq \mathrm{Cl}(f(M)) \).
	\end{enumerate}
\end{statement}

\begin{proof} \ 
	\begin{itemize}
		\item[ \(\boxed{1 \longrightarrow 2}\)] Пусть \( x \in \mathrm{Cl}(A) \) и \( U(f(x)) \) — произвольная окрестность точки \( f(x) \in Y \). Согласно условию предпосылки, существует окрестность \( V(x) \) точки \( x \in X \), такая что \( f(V(x)) \subseteq U(f(x)) \).

		Поскольку \( x \in \mathrm{Cl}(A) \), то существует точка \( a \in A \cap V(x) \). Следовательно, \( f(a) \in f(A) \cap f(V(x)) \subseteq f(A) \cap U(f(x)) \). Поскольку выбор окрестности \( U(f(x)) \) был произвольным, для каждой точки \( f(x) \in f(\mathrm{Cl}(A)) \) верно, что \( f(x) \in \mathrm{Cl}(f(A)) \).
		
		\item[ \(\boxed{2 \longrightarrow 1}\)]  Пусть \( x \in X \) и \( U(f(x)) \subseteq Y \) — произвольная окрестность точки \( f(x) \). По условию предпосылки, \( f^{-1}[U(f(x))] \subseteq X \).

		Так как \( x \in f^{-1}[U(f(x))] \), то можно взять окрестность \( V(x) = f^{-1}[U(f(x))] \), что и требовалось.
		
		\item[ \(\boxed{3 \longrightarrow 2}\)] Пусть \( F \closed Y \). По условию предпосылки, \( f^{-1}[F] \closed X \). Рассмотрим множество \( U = Y \setminus F \subseteq Y \), которое открыто.

		Тогда \( f^{-1}[U] = f^{-1}[Y \setminus F] = f^{-1}[Y] \setminus f^{-1}[F] = X \setminus f^{-1}[F] \open X \), то есть \( f^{-1}[U] \open X \), как и требовалось.		
		\item[ \(\boxed{4 \longrightarrow 3}\)] Пусть \( F \subset Y \), но \( f^{-1}[F] \) не замкнуто. Рассмотрим точку \( x \in \mathrm{Cl}(f^{-1}[F]) \setminus f^{-1}[F] \). Тогда \( f(x) \in f(\mathrm{Cl}(f^{-1}[F])) \subseteq f^{-1}[F] \) по условию предпосылки.

		Однако, если \( f(x) \in f(\mathrm{Cl}(f^{-1}[F])) \subseteq \mathrm{Cl}(f(F)) \), то возникает противоречие, так как \( f^{-1}[f(d)] \subseteq f^{-1}[F] \), что противоречит предположению, что \( x \in \mathrm{Cl}(f^{-1}[F]) \).
		
		Следовательно, предположение о том, что \( f^{-1}[F] \) не замкнуто, ошибочно, и \( f^{-1}[F] \) должно быть замкнутым в \( X \).		
	\end{itemize}
\end{proof}

\begin{remark}
	Все функции, которые изучали в курсе стандартного анализа и называли непрерывными (например, полиномиальные, рациональные, показательные, тригонометрические функции и другие), продолжают оставаться непрерывными и в контексте топологических пространств. В этом параграфе мы уточнили определение непрерывности, распространяя его на более общий случай, где под отображениями понимаются функции между топологическими пространствами.
\end{remark}

\subsection{Открытое и замкнутое отображение}

\begin{definition}[Открытое отображение]
	Пусть \( f : X \to Y \) — отображение между топологическими пространствами. Говорят, что \( f \) является \textbf{открытым отображением}, если для любого открытого множества \( U \open X \) его образ \( f(U) \open Y \) также является открытым.
\end{definition}

\begin{definition}[Замкнутое отображение]
	Пусть \( f : X \to Y \) — отображение между топологическими пространствами. Говорят, что \( f \) является \textbf{замкнутым отображением}, если для любого замкнутого множества \( F \closed X \) его образ \( f(F) \closed Y \) также является замкнутым.
\end{definition}

\begin{remark}[о не эквивалентности непрерывного о открытого/замкнутого отображения]
	Если в определении непрерывного отображения заменить оба упоминания слова "открытое" на "замкнутое"\ , то результат будет эквивалентен определению непрерывности. Однако аналогичное преобразование для определения открытого отображения не приводит к эквивалентному утверждению, так как «образ любого замкнутого множества замкнут», соответствует определению замкнутого отображения, которое в общем случае не эквивалентно открытости.

	Существуют открытые отображения, которые не являются замкнутыми и наоборот — замкнутые отображения не являются открытыми. Это различие между открытыми/замкнутыми отображениями и непрерывными отображениями обусловлено тем, что для любого множества \( S \) в общем случае выполняется включение \( f(X \setminus S) \supseteq f(X) \setminus f(S) \), тогда как для прообразов всегда выполняется равенство \( f^{-1}[Y \setminus S] = f^{-1}[Y] \setminus f^{-1}[S] \).
\end{remark}

\begin{example}[открытое и замнкутое отображение] \
	\begin{enumerate} 
		\item Рассмотрим функцию \( f: (\mathbb{R}, \Omega_{\text{std}}) \to (\mathbb{R}, \Omega_{\text{std}})\), заданную как \( f(x) = x^2 \). Эта функция является непрерывной, замкнутой, но не открытой:
		\begin{itemize}
			\item Пусть \( U = (a, b) \) — открытый интервал в \( \mathbb{R} \), который не содержит \( 0 \). Тогда образ этого интервала при отображении \( f \) будет равен \( f(U) = (\min\{a^2, b^2\}, \max\{a^2, b^2\}) \open \mathbb{R}\).
			\item Если же \( U = (a, b) \) — интервал, содержащий \( 0 \), то образ этого интервала будет равен \( f(U) = [0, \max\{a^2, b^2\}) \). Это множество не является открытым в \( \mathbb{R} \).
		\end{itemize}
		\item Рассмотрим функцию \( f: (\mathbb{R}, \Omega_{\text{std}}) \to (\mathbb{Z}, \Omega_{\text{disc}}) \), заданную как \( f(x) = \mathrm{floor}(x) \). Эта функция является открытой и замкнутой, но не является непрерывной. 
\end{enumerate}
\end{example}

\begin{statement}[об открытости проекции]
	Пусть $X$ и $Y$ -- произвольные топологические пространства. Проекция 
	\[
	\mathrm{pr}_1 : X \times Y \to X,
	\]
	определённая как $\mathrm{pr}_1(x, y) = x$, является открытым отображением.	
\end{statement}
\begin{proof}
	Рассмотрим произвольное открытое множество $U \open X \times Y$. По определению топологии произведения, множество $U$ является объединением конечных пересечений множеств вида $V \times W$, где $V \open X$ и $W \open Y$. Следовательно, без ограничения общности, положим, что $U = V \times W$.

Тогда 
\[
\pr{1}{U} = \pr{1}{V \times W} = V.
\]

Поскольку $V$ открыто в $X$, получаем, что образ $\pr{1}{U}$ также является открытым множеством в $X$. 

Так как $U$ было произвольным открытым множеством в $X \times Y$, проекция отображает открытые множества в открытые, то есть проекция является открытым отображением.
\end{proof}
Стоит отметить, что открытые и замкнутые отображения отличаются от непрерывных. В частности, обратное утверждение, что проекция является замкнутым отображением, не выполняется без дополнительных условий, таких как компактность. Рассмотрим следующий контрпример.
\begin{remark}
	Пусть $X = Y = \mathbb{R}$, и рассмотрим проекцию 
\[
\mathrm{pr} : \mathbb{R} \times \mathbb{R} \to \mathbb{R}.
\]
Определим множество $H \subseteq \mathbb{R} \times \mathbb{R}$ как гиперболу
\[
H = \{(x, y) \in \mathbb{R} \times \mathbb{R} \mid x y = 1\}.
\]
Множество $H$ является замкнутым в топологии произведения на $\mathbb{R} \times \mathbb{R}$, поскольку оно задаётся уравнением, которое является непрерывным. Однако его проекция на первую координату
\[
\pr{1}{H} = \{x \in \mathbb{R} \mid x \neq 0\}
\]
не является замкнутым множеством в $\mathbb{R}$, так как дополнение этого множества $\mathbb{R} \setminus \pr{1}{H} = \{0\}$ является точкой, которая не входит в $\pr{1}{H}$, но точка $0$ является предельной для $\pr{1}{H}$. Следовательно, проекция не замкнута в $\mathbb{R}$.

\end{remark}


\subsection{Гомеоморфизм}

\textbf{Гомеоморфизмы} являются одним из центральных понятий топологии, позволяя установить эквивалентность пространств с точки зрения их топологической структуры.

\begin{definition}[Гомеоморфизм]
	Отображение \( f : X \to Y \) называется гомеоморфизмом, если оно биективно, непрерывно, и обратное отображение \( f^{-1} : Y \to X \) также является непрерывным. 

	В таком случае говорят, что пространства \( X \) и \( Y \) \textbf{гомеоморфны}, что обозначается как:
	\[
		X \cong Y.
	\]
\end{definition}

\textbf{Интуитивное объяснение:} Гомеоморфизм можно понимать как "гладкую деформацию" одного пространства в другое без разрывов и склеек. Гомеоморфные пространства обладают одинаковыми топологическими свойствами, такими как связность, компактность или число дыр. 

\begin{example}
	\begin{enumerate}
		\item \textbf{Тождественное отображение:} Любое топологическое пространство \( X \) гомеоморфно самому себе через тождественное отображение \( \mathrm{id}_X \).
		
		\item \textbf{Интервал и прямая:} Интервал \( (0, 1) \) и вся числовая прямая \( \mathbb{R} \) гомеоморфны. Отображение \( f(x) = \tan\left(\pi \left(x - \frac{1}{2}\right)\right) \) задаёт гомеоморфизм между этими пространствами.

		\item \textbf{Квадрат и круг:} Открытый квадрат \( (0, 1) \times (0, 1) \) и открытый круг \( \{(x, y) \in \mathbb{R}^2 : x^2 + y^2 < 1\} \) гомеоморфны. Такой гомеоморфизм может быть построен через радиальное преобразование координат.

		\item \textbf{Сфера и плоскость:} Сфера Римана (сфера \( S^2 \) с удалённым северным полюсом) гомеоморфна плоскости \( \mathbb{R}^2 \). Это отображение обычно реализуется стереографической проекцией.
	\end{enumerate}
\end{example}

\bigskip

\begin{remark}
	Понятие гомеоморфизма лежит в основе топологии, так как позволяет классифицировать пространства с точки зрения их "деформируемости". Например, задача понять, являются ли два объекта "одинаковыми" (гомеоморфными) возникает как в алгебраической топологии, так и в теории динамических систем, где гомеоморфизмы описывают устойчивые состояния систем.
\end{remark}

\begin{statement}[Отношение эквивалентности]
	Гомеоморфность является отношением эквивалентности.
\end{statement}
\begin{proof}
	Для доказательства, что гомеоморфность является отношением эквивалентности, проверим три свойства: рефлексивность, симметричность и транзитивность.
	\begin{itemize}
		\item \textbf{Рефлексивность.} Пусть \( X \) — топологическое пространство. Тождественное отображение \( \mathrm{id}_X : X \to X \), заданное как \( \mathrm{id}_X(x) = x \) для всех \( x \in X \), является биективным, непрерывным, и обратное ему \( \mathrm{id}_X^{-1} = \mathrm{id}_X \) также непрерывно. Следовательно, \( X \cong X \), то есть гомеоморфность рефлексивна.
		\item \textbf{Симметричность.} Пусть \( X \cong Y \), то есть существует биективное отображение \( f : X \to Y \), непрерывное, с непрерывным обратным \( f^{-1} : Y \to X \). Тогда \( f^{-1} \) является гомеоморфизмом \( Y \to X \), что означает \( Y \cong X \). Таким образом, гомеоморфность симметрична.
		\item \textbf{Транзитивность.} Пусть \( X \cong Y \) и \( Y \cong Z \). Тогда существуют гомеоморфизмы \( f : X \to Y \) и \( g : Y \to Z \). Композиция \( g \circ f : X \to Z \) является биективным отображением, так как \( f \) и \( g \) биективны. Более того, так как композиция непрерывных отображений непрерывна, \( g \circ f \) непрерывно. Обратное отображение \( (g \circ f)^{-1} = f^{-1} \circ g^{-1} \) также является непрерывной композицией непрерывных отображений. Следовательно, \( X \cong Z \), то есть гомеоморфность транзитивна.
	\end{itemize}
	Проверенные свойства рефлексивности, симметричности и транзитивности показывают, что гомеоморфность является отношением эквивалентности.
\end{proof}


\begin{remark}[Cовпадения непрерывных отображений в хаусдорфово пространство.]
	Классическая проблема топологии -- определить, являются ли два пространства гомеоморфными. Для доказательства гомеоморфности обычно строят соответствующий гомеоморфизм. Для доказательства негомеоморфности часто используются косвенные методы: находят свойство, которым обладает одно пространство, но не другое, и которое сохраняется при гомеоморфизме. Топологические свойства и инварианты, такие как мощность множества точек и мощность топологической структуры, являются очевидными примерами.
\end{remark}

\begin{statement}[О сужение на всюду плотное множество]
	Предположим, что \( f \) и \( g \) -- две непрерывные функции такие, что \( f : X \rightarrow Y \) и \( g : X \rightarrow Y \). \( Y \) -- пространство Хаусдорфа. Предположим, что \( f(x) = g(x) \) для всех \( x \in A \subseteq X \), где \( A \) является всюду плотным в \( X \), тогда \( f(x) = g(x) \) для всех \( x \in X \).
\end{statement}
\begin{proof}
	Предположим, что \( f(x_0) \neq g(x_0) \). Так как \( Y \) является пространством Хаусдорфа, существуют открытые множества \( U, V \subset Y \) такие, что \( f(x_0) \in U \), \( g(x_0) \in V \), и \( U \cap V = \varnothing \). Теперь
\[ x_0 \in f^{-1}(U) \cap g^{-1}(V) =: W \]
и \( W \) открыто, следовательно, \(\exists a \in A \cap W\), откуда
\[ f(a) = g(a) \in U \cap V \Rightarrow U \cap V \neq \varnothing \]
Это противоречие доказывает результат.
\end{proof}

\begin{statement}[О замкнутости непрерывных отображений в Хаусдорфовом пространстве]
	Пусть \( Y \) -- топологическое пространство, а \( f, g : Y \rightarrow X \) -- непрерывные функции. Докажите, что множество
\[ E = \{ y \in Y : f(y) = g(y) \} \] -- замкнуто.
\end{statement}
\begin{proof}
	Функция \( f - g \) непрерывна, следовательно, \( E = (f - g)^{-1}(\{0\}) \) является обратным образом множества относительно непрерывного отображения. Осталось убедиться, что \(\{0\}\) замкнуто.
 Рассмотрим \( h : Y \rightarrow X \times X \), \( y \mapsto (f(y), g(y)) \). Заметим, что \( E = h^{-1}(\Delta) \). Поскольку \( X \) является Хаусдорфом, то \(\Delta\) замкнута, следовательно, также и \( E \).
\end{proof}

\subsection{Примеры гомеоморфных пространств}
\begin{statement}
	Отрезок \( [0; 1] \) гомеоморфен отрезку \( [a; b] \), где \( a < b \).
\end{statement}
\begin{proof} Проиллюстрируем доказательство.
\begin{figure}[h!]
	\begin{minipage}{0.55\textwidth}
		Рассмотрим отображение \( f: [0; 1] \to [a; b] \), определённое следующим образом:
		\[
		f(x) = a + (b - a) \cdot x \quad \text{для} \quad x \in [0; 1].
		\]
		Это отображение является непрерывным, биективным (для каждого \( y \in [a; b] \) существует единственное \( x \in [0; 1] \), такое что \( f(x) = y \)), и его обратное отображение имеет вид:
		\[
		f^{-1}(y) = \frac{y - a}{b - a} \quad \text{для} \quad y \in [a; b].
		\]
		Следовательно, \( f \) является гомеоморфизмом.
	\end{minipage}
	\hfill
	\begin{minipage}{0.45\textwidth}
		\begin{center}
			\begin{tikzpicture}[scale=1, line join=round,line cap=round,
				dot/.style={circle,fill,inner sep=1pt},>={Stealth[length=1.2ex]}]
				% Оси
				\draw[->] (-0.5, 0) -- (5, 0) node[below] {$x$}; 
				\draw[->] (0, -0.5) -- (0, 6) node[left] {$f(x)$};

				% Отрезок [0, 1] на Ox
				\draw (0, 0) -- (3, 0);
				\fill (0, 0) circle (2pt) node[below left] {$0$};
				\fill (3, 0) circle (2pt) node[below right] {$1$};

				% Отрезок [a, b] на Oy
				\draw (0, 2) -- (0, 5);
				\fill (0, 2) circle (2pt) node[left] {$a$};
				\fill (0, 5) circle (2pt) node[left] {$b$};

				% Прямая f(x) = a + (b-a)x
				\draw (0, 2) -- (3, 5) node[above] {};

				% Проекция x
				\draw[->-] (1.5, 0) -- (1.5, 3.5);
				\draw[->-] (1.5, 3.5)  -- (0, 3.5);

				% Точка на графике
				\fill (1.5, 3.5) circle (2pt);
				\fill (1.5, 0) circle (2pt) node[below] {$x$};
				\fill (0, 3.5) circle (2pt) node[left] {$x^*$};
			\end{tikzpicture}
		\end{center}
	\end{minipage}
\end{figure}
\end{proof}

\begin{definition}[Сфера Римана]
	\textbf{Сфера Римана} в комплексной плоскости \( \mathbb{C} \) — это единичная сфера в трёхмерном пространстве, где выкалывается точка северного полюса \( N = (0, 0, 1) \).
\end{definition}

\begin{statement}
	Множество комплексных чисел \( \mathbb{C} \) гомеоморфно сфере Римана, где сфера Римана задается уравнением \( x^2 + y^2 + z^2 = 1 \) в трехмерном пространстве. 
\end{statement}
\begin{proof}
	Проиллюстрируем доказательство.
	\begin{center}
			\begin{tikzpicture}[declare function={%
			stereox(\x,\y)=2*\x/(1+\x*\x+\y*\y);%
			stereoy(\x,\y)=2*\y/(1+\x*\x+\y*\y);%
			stereoz(\x,\y)=(-1+\x*\x+\y*\y)/(1+\x*\x+\y*\y);
			Px=1.75;Py=-1.5;Qx=-1.5;Qy=-1.25;amax=2.5;},scale=2.5,
			line join=round,line cap=round,
			dot/.style={circle,fill,inner sep=1pt},>={Stealth[length=1.2ex]}]

		\pgfmathsetmacro{\myaz}{15}
		\pgfmathsetmacro{\amax}{2.5}

		\path[save path=\pathSphere] (0,0) circle[radius=1];

		\begin{scope}[3d view={\myaz}{18}]

			\draw[->] (-1.5, 0, 0) -- (1.5, 0, 0);
			\draw[->] (0, -1.5, 0) -- (0, 1.5, 0);
			\draw[->] (0, 0, -1.5) -- (0, 0, 1.5);
			
			\draw (-\amax,\amax) -- (-\amax,-\amax) coordinate (bl) -- (\amax,-\amax)
				coordinate (br) -- (\amax,\amax);

			\begin{scope}
				\tikzset{protect=\pathSphere}
				\draw (-\amax,\amax) -- (\amax,\amax) node[below left,xshift=-2em]{$\mathbb{C}$};
			\end{scope}

			\begin{scope}
				\clip[reuse path=\pathSphere];
				\draw[dashed] (-\amax,\amax) -- (\amax,\amax);
			\end{scope}

			\begin{scope}[canvas is xy plane at z=0]
				\draw[thick, dashed] (\myaz:1) arc[start angle=\myaz,end angle=\myaz+180,radius=1];
				\draw[thick] (\myaz:1) arc[start angle=\myaz,end angle=\myaz-180,radius=1];

				\path[save path=\pathPlane] (\myaz:\amax) -- (\myaz+180:\amax) --(bl) -- (br) -- cycle;

				\begin{scope}
					\clip[use path=\pathPlane];
					\draw[dashed,thick,use path=\pathSphere];
				\end{scope}

				\begin{scope}
					\tikzset{protect=\pathPlane}
					\draw[thick, use path=\pathSphere];
				\end{scope}
			\end{scope}

			\draw[->-=0.3] (Px,Py,0) node[dot,label=below:{$w$}](w){}
			-- ({stereox(Px,Py)},{stereoy(Px,-1)},{stereoz(Px,Py)})
			node[dot,label=above right:{$w^*$}](w*){};

			\draw[->-] (Qx,Qy,0) node[dot,label=below:{$z$}](z){}
			--  ({stereox(Qx,Qy)},{stereoy(Qx,-1)},{stereoz(Qx,Qy)})
			node[dot,label=below right:{$z^*$}](z*){};
			\draw[dashed] (w*) -- (0,0,1) node[dot,label=above left:{$N$}](zeta){} -- (z*) -- (w*);
			\draw[dashed] (w*) -- (0,0,1) node[circle, draw, inner sep=0.7pt,fill=white, label=above left:{$N$}](zeta){} -- (z*) -- (w*);
		\end{scope}
	\end{tikzpicture}
\end{center}
Рассмотрим отображение \( f: \mathbb{C} \to S^2 \), задаваемое формулой:

\[
f(u) = \left( \frac{2\Re(u)}{1 + |u|^2}, \frac{2\Im(u)}{1 + |u|^2}, \frac{|u|^2 - 1}{1 + |u|^2} \right),
\]

где \( u \in \mathbb{C} \), а \( S^2 \) — сфера Римана. Это отображение непрерывно и биективно, поскольку для каждой точки на сфере существует уникальное \( u \in \mathbb{C} \). Обратное отображение:

\[
u = \frac{x + iy}{1 - z},
\]

где \( (x, y, z) \) — координаты точки на сфере, является непрерывным. Следовательно, \( f \) — гомеоморфизм.
\end{proof}
\starsection{Задачи и упражнения}

\begin{task}[Вновь Канторово множество]
    Рассмотрим множество \( \mathcal{C}^2 \), содержащее пары точек из множества \( \mathcal{C} \) в пространстве \( \mathbb{R}^2 \), то есть
    \[
		\mathcal{C}^2 = \{(x, y) \in \mathbb{R}^2 \mid x \in \mathcal{C}, y \in \mathcal{C}\}.
    \]
    Докажите, что отображение \( f: \mathcal{C} \to \mathcal{C}^2 \), заданное по формулам
    \[
    f: \sum_{k=1}^{\infty} \frac{a_k}{3^k} \mapsto  \left(\sum_{k=1}^{\infty} \frac{a_{2k-1}}{3^k}, \sum_{k=1}^{\infty} \frac{a_{2k}}{3^k}\right),
    \]
    является непрерывной сюръекцией.
\end{task}

\begin{task}
    Постройте непрерывную биекцию между интервалом \( [0; 1) \) и окружностью \( S^1 \), которая не будет являться гомеоморфизмом.
\end{task}


\begin{task}
    Пусть \( f : X \to Y \) — гомеоморфизм. Тогда для любого множества \( A \subset X \) выполняются следующие утверждения:
    \begin{enumerate}
        \item Множество \( A \) замкнуто в \( X \), тогда и только тогда, когда \( f(A) \) замкнуто в \( Y \);
        \item \( f(\text{Cl} A) = \text{Cl} f(A) \);
        \item \( f(\text{Int} A) = \text{Int} f(A) \);
        \item \( f(\text{Fr} A) = \text{Fr} f(A) \);
        \item \( A \) — окрестность точки \( x \in X \), тогда и только тогда, когда  \( f(A) \) — окрестность точки \( f(x) \).
    \end{enumerate}
\end{task}


\begin{task}[Комплексный гомеоморфизм]
	Докажите, что конформное отображение, заданное дробно-линейным преобразованием \( f : \hat{\mathbb{C}} \to \hat{\mathbb{C}} \), является гомеоморфизмом на расширенной комплексной плоскости \( \hat{\mathbb{C}} = \mathbb{C} \cup \{\infty\} \). 

	Преобразование задается соотношением:
	\[
	f: z \mapsto \frac{az + b}{cz + d}, \quad ad - bc \neq 0,
	\]
	где \( a, b, c, d \) — комплексные числа.	
\end{task}
\begin{task}[Вещественный гомеоморфизм]
    Докажите, что любое биективное монотонное отображение \( f : \mathbb{R} \to \mathbb{R} \) является гомеоморфизмом. 
\end{task}

\begin{task}[Классические гомеооморфизмы]
		Докажите, что выполняются следующие гомеоморфизмы и проиллюстрируйте их:
		\begin{enumerate}
			\item \( [0; 1) \cong [a; b) \cong (0; 1] \cong (a; b] \) для любых \( a < b \);
			\item \( (0; 1) \cong (a; b) \) для любых \( a < b \);
			\item \( (-1; 1) \cong \mathbb{R} \);
			\item \( [0; 1) \cong [0; +\infty) \) и \( (0; 1) \cong (0; +\infty) \);
			\item \( \mathbb{C}^n \cong \mathbb{R}^{2n} \);
			\item \( S^n \setminus N \cong \mathbb{R}^n \).
		\end{enumerate}	
\end{task}

\begin{task}[Гомеоморфизм Канторово множества]
    Покажите, что существует гомеоморфизм между множеством Кантора \( \mathcal{C} \) и счётным произведением двуточечного множества \( \{0, 1\}^{\aleph_0} \) с дискретной топологией, то есть:
    \[
    \{0, 1\}^{\aleph_0} \cong \mathcal{C}
    \]
\end{task}
