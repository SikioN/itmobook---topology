\section{База топологии}
\subsection{Определение}
Часто топологическую структуру задают посредством описания некоторой её части, достаточной для восстановления всей структуры. Например, топологию можно задать указанием базы топологии или системы окрестностей, которые определяют топологическую структуру.
\begin{definition}[База топологии]
	\textbf{Базой топологии} называется некоторый набор открытых множеств \( \mathcal{B} \), такой, что всякое непустое открытое множество представимо в виде объединения множеств из этого набора:
	\begin{equation*}
		\forall U \in \Omega:\  U = \bigcup_{i \in \mathbb{I}} B_i \quad B_i \in \mathcal{B}
	\end{equation*}
\end{definition}

\begin{statement}[Критерий базы]
	Пусть $\mathcal{B} \subseteq 2^X$, удовлетворяющее двум условиям:
	\begin{enumerate}
		\item Для каждой точки \( x \in X \) существует такое множество \( B \in \mathcal{B} \), что \( x \in B \)
		\item Для любых двух элементов \( B_1 \) и \( B_2 \) из \( \mathcal{B} \) и для любой точки \( x \), которая принадлежит пересечению \( B_1 \cap B_2 \), существует элемент \( B_3 \) из \( \mathcal{B} \), такой что \( x \in B_3 \) и \( B_3 \subseteq B_1 \cap B_2 \)
	\end{enumerate}
	тогда $\mathcal{B}$ -- база некоторой однозначно определённой топологии.
\end{statement}

\begin{proof}
	Пусть $\Omega = \left\{\bigcup B_\alpha: \ B_\alpha \in \mathcal{B}\right\}$, из пункта $1$ ясно, что $\varnothing \in \Omega$ и $X \in \Omega$. В объединение находятся $U_1, U_2 \in \Omega$; $U_1 \cap U_2 = \bigcup_j \Omega_j$
\end{proof}

\begin{example}
	Рассмотрим несколько примеров баз топологии:
	\begin{enumerate}
		\item \textbf{База стандартной топологии на \( \mathbb{R} \):} набор всех интервалов вида \( (a, b) \), где \( a, b \in \mathbb{R} \), \( a < b \). Любое открытое множество на прямой \( \mathbb{R} \) можно представить как объединение таких интервалов.
		\item \textbf{База топологии Зарисского на \( \mathbb{R}^n \):} набор множеств вида \( \mathbb{R}^n \setminus V \), где \( V \) — множество нулей некоторого конечного набора многочленов. Эта топология используется в алгебраической геометрии.
		\item \textbf{База дискретной топологии на произвольном множестве \( X \):} набор всех одноэлементных подмножеств \( \{x\} \), где \( x \in X \). Любое открытое множество в дискретной топологии является объединением этих одноэлементных множеств.
	\end{enumerate}
\end{example}


\begin{definition}[Предбаза топологии]
	\textbf{Предбазой топологии} на множестве называется некоторый набор открытых подмножеств \( \{U_\alpha\} \), такой, что любое открытое множество можно представить в виде объединения конечных пересечений элементов из базы топологии:
	\begin{equation*}
		\forall U \in \Omega:\  U = \bigcup_{i \in \mathbb{I}} \bigcap_{j \in \mathbb{J}_i} U_{\alpha_j}, \quad U_{\alpha_j} \in \{U_\alpha\},
	\end{equation*}
	где \( \mathbb{I} \) и \( \mathbb{J}_i \) — индексы объединений и пересечений соответственно.
\end{definition}

\begin{remark}
	Любой набор множеств \( \{U_\alpha\} \subseteq 2^M \), такой, что \( \bigcup_\alpha U_\alpha = M \), является предбазой некоторой топологии на \( M \).
\end{remark}

\begin{remark}
	Любой набор множеств \( \{U_\alpha\} \subseteq 2^M \), замкнутый относительно конечных пересечений и такой, что \( \bigcup_\alpha U_\alpha = M \), является базой некоторой топологии на \( M \).
\end{remark}



\starsection{Задачи и упражнения}
\begin{task}
	Существуют ли примеры различных топологических структур, которые обладают одной и той же базой? Подумайте, какие особенности могут при этом сохраняться.
\end{task}

\begin{task}
	Составьте базы для следующих топологических пространств, стараясь выбрать минимально возможные наборы: 
\begin{itemize}
    \item дискретное пространство;
    \item антидискретное пространство;
    \item топология стрелки.
\end{itemize}
Объясните ваш выбор.
\end{task}

\begin{task}
	Существуют ли топологические структуры, у которых может быть лишь одна возможная база? Найдите такие примеры и объясните их уникальность.
\end{task}

	

\begin{task}
	Рассмотрим три набора подмножеств плоскости \( \mathbb{R}^2 \), которые описывают круги в различных нормированных пространствах:
	\begin{enumerate}
		\item Открытые круги в евклидовой норме.
		\item Открытые квадраты без граничных точек, стороны которых параллельны координатным осям.
		\item Открытые квадраты без граничных точек, стороны которых параллельны биссектрисам координатных углов.
	\end{enumerate}
	Докажите следующие утверждения:
	\begin{enumerate}
		\item Любой открытый круг в евклидовой норме можно представить как объединение открытых квадратов в норме Чебышёва.
		\item Пересечение любых двух открытых квадратов в Манхэттенской норме является объединением других квадратов в норме Чебышёва.
		\item Каждый из описанных наборов служит базой некоторой топологии на \( \mathbb{R}^2 \), и все три топологии совпадают.
	\end{enumerate}
\end{task}
