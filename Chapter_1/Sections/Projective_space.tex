
\section{Проективное пространство}
\subsection{Определение через факторизацию}
% \begin{definition}[Проективное пространство] 
% 	Пусть \( V \) -- векторное пространство над полем \( K \). Проективным пространством называется множество всех одномерных подпространств в \( V \), обозначаемое как \( \mathbb{P}(V) \). 
% \end{definition}
% \begin{example}
% 	\begin{enumerate}
% 		\item \textbf{Проективное пространство двумерных подпространств}: Пусть \( V \) -- четырехмерное векторное пространство над полем \( \mathbb{R} \). Тогда множество всех плоскостей в \( V \) является проективным пространством, обозначаемым как \( \mathbb{P}^2(\mathbb{R}) \). Каждая плоскость в \( V \) соответствует одному элементу в \( \mathbb{P}^2(\mathbb{R}) \).
    
% 		\item \textbf{Проективное пространство прямых в трехмерном пространстве}: Пусть \( V \) -- трехмерное векторное пространство над полем \( \mathbb{R} \). Тогда множество всех прямых в \( V \) является проективным пространством, обозначаемым как \( \mathbb{P}^2(\mathbb{R}) \). Каждая прямая в \( V \) соответствует одному элементу в \( \mathbb{P}^2(\mathbb{R}) \).
% 	\end{enumerate}
% \end{example}

% \begin{gather*}
% 	\mathbb{R}/\mathbb{Z} \cong S^1 \\
% 	\mathbb{R}\mathbb{P} = S^n/(\mathbb{Z}/2) = S^n/_{x \sim -x} \\
% 	\mathbb{Z}/2 \curvearrowright S^n \left(a \cdot s = s\right)
% \end{gather*}




Проективное пространство является важным примером в теории многообразий и играет ключевую роль в дальнейшем повествовании. Оно возникает естественным образом при изучении геометрии и топологии, а также находит применение в алгебраической геометрии, физике и других областях.


\begin{definition}[Проективное пространство]
$n$-мерное \textbf{вещественное проективное пространство} $ \mathbb{RP}^n $ определяется как факторпространство сферы $ S^n $ по отношению эквивалентности, которое отождествляет диаметрально противоположные точки:
\[
\mathbb{RP}^n = S^n / \sim, \quad \text{где } x \sim y \iff y = -x.
\]
Эквивалентно, $ \mathbb{RP}^n $ можно определить как множество прямых в $ \mathbb{R}^{n+1} $, проходящих через начало координат:
\[
\mathbb{RP}^n = (\mathbb{R}^{n+1} \setminus \{0\}) / \sim, \quad \text{где } x \sim y \iff y = \lambda x \text{ для некоторого } \lambda \in \mathbb{R} \setminus \{0\}.
\]
\end{definition}


Построение проективного пространства можно наглядно представить с помощью следующей коммутативной диаграммы:

\[
\begin{tikzcd}[row sep=40pt, column sep=60pt]
    S^n \arrow{r}{\mathrm{pr}} \arrow{d}{q} & \mathbb{RP}^n \\
    S^n / \sim \arrow{ur}{\cong}
\end{tikzcd}
\]
\begin{example}
    \begin{enumerate}
        \item \textbf{Одномерное проективное пространство $ \mathbb{RP}^1 $} можно интерпретировать как множество всех прямых в $ \mathbb{R}^2 $, проходящих через начало координат. Каждая прямая определяется своим направлением, и противоположные направления отождествляются. На рисунке ниже различные прямые изображены разными стилями линий, чтобы подчеркнуть их эквивалентность по направлениям.
        \begin{center}
            \begin{tikzpicture}[scale=1.5]
        
                % Оси координат
                \draw[thick, ->] (-2, 0) -- (2, 0) node[right] {$x$};
                \draw[thick, ->] (0, -2) -- (0, 2) node[above] {$y$};
        
                % Примеры прямых через начало координат
                \draw[thick, densely dotted] (-2, 1) -- (2, -1); % Прямая 1 (пунктирная)
                \draw[thick, dashed] (-2, -1) -- (2, 1); % Прямая 2 (штриховая)
                \draw[thick] (-1, -2) -- (1, 2); % Прямая 3 (сплошная)
                \draw[thick, dash dot] (-1, 2) -- (1, -2); % Прямая 4 (штрих-пунктирная)
        
                % Единичная окружность
                \draw[dashed] (0, 0) circle (1);
        
                % Стрелка перехода
                \node at (3, 0) {$\cong$};
        
                % Проективная прямая (окружность)
                \begin{scope}[shift={(5, 0)}]
                    \draw[thick] (0, 0) circle (1);
                    \fill[black] (1, 0) circle (1pt); % Точка для прямой 1
                    \fill[black] (0, 1) circle (1pt); % Точка для прямой 3
                    \node at (0, -1.3) {$\mathbb{RP}^1$};
                \end{scope}
        
            \end{tikzpicture}
        \end{center}
        \item \textbf{Двумерное проективное пространство $ \mathbb{RP}^2 $:}
        Двумерное проективное пространство $ \mathbb{RP}^2 $ можно представить как фактор сферы $ S^2 $, где каждая пара диаметрально противоположных точек отождествлена. Это пространство является примером двумерного многообразия, которое не может быть вложено в $ \mathbb{R}^3 $ без самопересечений.
    \end{enumerate}
\end{example}

\begin{theorem}[Проективное пространство как фактор сферы]
$ \mathbb{RP}^n $ можно представить как фактор двумерной сферы $ S^n $ по действию группы $ \mathbb{Z}_2 $, где элемент $ 1_{\mathbb{Z}_2} $ переводит точку в диаметрально противоположную:
\[
\mathbb{RP}^n \cong S^n / \mathbb{Z}_2.
\]
\end{theorem}

\begin{proof}
Каждая прямая в $ \mathbb{R}^{n+1} $, проходящая через начало координат, соответствует классу эквивалентности точек на сфере $ S^n $. При этом диаметрально противоположные точки на сфере отождествляются, что приводит к изоморфизму $ \mathbb{RP}^n \cong S^n / \mathbb{Z}_2 $.
\end{proof}

\subsection{Однородные координаты}

\begin{definition}[Однородные координаты]
    \textbf{Однородные координаты} — это способ представления точек проективного пространства $ \mathbb{RP}^n $ с помощью наборов чисел $ [x_0 : x_1 : \dots : x_n] $, где $ (x_0, x_1, \dots, x_n) \in \mathbb{R}^{n+1} \setminus \{0\} $. Эти координаты определены с точностью до пропорциональности, то есть:
    \[
    [x_0 : x_1 : \dots : x_n] = [\lambda x_0 : \lambda x_1 : \dots : \lambda x_n], \quad \forall \lambda \in \mathbb{R} \setminus \{0\}.
    \]
    \end{definition}
    
    \begin{remark}[Связь с проективным пространством]
    Однородные координаты естественным образом возникают при построении проективного пространства. Каждая точка $ \mathbb{RP}^n $ соответствует классу эквивалентности ненулевых векторов $ (x_0, x_1, \dots, x_n) \in \mathbb{R}^{n+1} $, где два вектора считаются эквивалентными, если они пропорциональны. Таким образом, однородные координаты предоставляют удобный способ описания точек проективного пространства.
    \end{remark}
    
    \begin{example}[Однородные координаты на проективной прямой $ \mathbb{RP}^1 $]
    Точки проективной прямой $ \mathbb{RP}^1 $ могут быть заданы однородными координатами $ [x_0 : x_1] $, где $ (x_0, x_1) \neq (0, 0) $. Например:
    \[
    [1 : 2] = [2 : 4], \quad [0 : 1] \neq [1 : 0].
    \]
    Здесь $ [1 : 0] $ и $ [0 : 1] $ представляют две "бесконечно удаленные" точки на проективной прямой.
    \end{example}
    
    \begin{statement}[Геометрическая интерпретация]
    Однородные координаты позволяют рассматривать проективное пространство как множество прямых в $ \mathbb{R}^{n+1} $, проходящих через начало координат. Каждая такая прямая задается набором однородных координат $ [x_0 : x_1 : \dots : x_n] $, который определяет направление прямой.
    \end{statement}
    \starsection{Задачи и упражнени}

\begin{task}[Проективное замыкание аффинной прямой]
    Докажите, что проективное замыкание аффинной прямой содержит ровно одну бесконечно удалённую точку.
    
    \textbf{Пояснение:} 
    Аффинная прямая в $\mathbb{R}^2$ задается уравнением $ax + by + c = 0$. При переходе к проективному пространству $\mathbb{RP}^2$, уравнение прямой записывается в однородных координатах как $aX_0 + bX_1 + cX_2 = 0$. Бесконечно удалённые точки соответствуют $X_2 = 0$. Покажите, что существует ровно одна такая точка.
    \end{task}
    
    \begin{task}[Пересечение двух различных проективных прямых]
    Докажите, что две различные проективные прямые в $\mathbb{RP}^2$ пересекаются ровно в одной точке.
    \end{task}
    
    \begin{task}[Уравнение прямой через две точки]
    Найдите уравнение прямой, проходящей через точки $[1 : 0 : -1]$ и $[2 : 1 : 0]$ в $\mathbb{RP}^2$.
    \end{task}
    
    \begin{task}[Точки на проективном замыкании единичной окружности]
    Найдите все точки на проективном замыкании единичной окружности $C : x^2 + y^2 = 1$ над полями $K = \mathbb{F}_3$, $\mathbb{F}_5$ и $\mathbb{F}_7$.
    \end{task}