\section{Компактное пространство}
\subsection{Топологическое определение}

В математическом анализе часто возникают множества с особыми свойствами, которые обеспечивают нам удобные условия для работы. Например, в контексте анализа мы сталкиваемся с леммой Гейне-Бореля, которая утверждает, что замкнутые и ограниченные множества в евклидовой топологии обладают удивительными свойствами. На курсах дифференциальных уравнений мы сталкиваемся с теоремой Асколи — Арцела, которая для пространств функций утверждает подобное свойство.

Эти теоремы подсказывают нам важность множества, которое можно "обрабатывать" в каком-то смысле — извлекать сходящиеся подпоследовательности или гарантировать существование решения, ограничивая его поиски конечным числом шагов. Эти свойства лежат в основе ключевой идеи в топологии, связанной с компактностью.

Компактность множества помогает гарантировать, что оно будет вести себя "хорошо" в различных математических контекстах, и является важным понятием в топологии. Перейдем к формальному определению компактного множества.

\begin{definition}[Компакт]
	\textbf{Компактное топологическое пространство} \( C \) — это пространство, в котором из любого открытого покрытия можно выбрать конечное подпокрытие. То есть, для любого покрытия \( \{ U_i \}_{i \in \mathbb{I}} \) пространства \( X \) существует конечное подмножество индексов \( \mathcal{J} \subseteq \mathbb{I} \), такое что:
	\begin{equation*}
		C = \bigcup_{i \in \mathcal{J}} U_i
	\end{equation*}
	где \( C \Subset X \) — компактное подмножество.
\end{definition}


\begin{example}
	Рассмотрим следующие примеры компактных пространств:
	\begin{enumerate}
		\item \textbf{Замкнутый отрезок на прямой:} Отрезок \( [0,1]^n \) с обычной топологией компактен. Это следует из леммы Гейне-Бореля, которая утверждает, что в пространстве размерности \( n \) замкнутое и ограниченное множество компактно.
		\item \textbf{Тор:} Прямоугольник, образующий тор \( S^1 \times S^1 \), является компактным пространством.
		
		\item \textbf{Замкнутый шар в \( \mathbb{R}^n \):} Замкнутый шар \( \{x \in \mathbb{R}^n : \|x\| \leq r\} \) в евклидовом пространстве \( \mathbb{R}^n \) также компактен.
	\end{enumerate}
\end{example}


\begin{statement}[О замкнутых подмножествах компакта]
	Если множество $S$ содержится в компактном множестве $T$ и $S$ замкнуто, то $S$ также компактно.
\end{statement}
\begin{proof}
	Предположим, что $\mathcal{U}$ -- открытое покрытие множества $S$. Каждое открытое множество в $\mathcal{U}$ имеет вид $U \cap S$ для некоторого открытого множества $U$ (открытого в $T$). Пусть $\mathcal{V} = \{ U \subseteq T \mid U$ открыто, и $\exists U' \in \mathcal{U} : U \cap S = U' \}$. Тогда $\mathcal{V}$ также является открытым покрытием множества $S$, так как $S$ замкнуто, мы имеем, что $T \setminus S$ открыто, поэтому $\mathcal{V} \cup \{ T \setminus S \}$ -- открытое покрытие множества $T$.

Из компактности $T$ следует, что у нас есть конечное подпокрытие, из которого мы можем получить конечное подпокрытие для $\mathcal{U}$.
\end{proof}


\begin{statement}[О произведение компактов]
	Пусть \(X, \ Y\) -- компактные пространство, тогда \(X \times Y\) -- компактное пространство.
\end{statement}
\begin{proof}
	Пусть $M$ -- открытое множество в $X \times Y$, такое что $\{x_0\} \times Y \subseteq M$. Тогда существует открытое множество $W \subseteq X$, содержащее $x_0$, такое что $W \times Y \subseteq M$.

Используя данное утверждение, необходимо показать, что для любого $x \in X$ можно создать трубку (\textit{tubular}) (покрытую конечным количеством открытых множеств) вокруг $\{x\} \times Y$.

Для начала рассмотрим открытое покрытие для $X \times Y$, обозначим его как $\{U_{\alpha}\}_{\alpha \in J} = \{A_{\alpha} \times B_{\alpha}\}$, где каждое $A_{\alpha}$ открыто в $X$, а каждое $B_{\alpha}$ открыто в $Y$. Что дальше?

Мы знаем, что для любого открытого покрытия для $X$, обозначим его как $\{A_{\alpha}\}_{\alpha \in J}$, существует конечное подпокрытие для $X$. Аналогично для $Y$.
\end{proof}

\begin{statement}[О замкнутости компакта в Хаусдорфовом пространстве]
	Компактное множество \( K \) в Хаусдорфовом пространстве \( X \) замкнуто.
\end{statement}
\begin{proof}
	Зафиксируем \( x \in X \setminus K \). Так как \( X \) является пространством Хаусдорфа, для каждого \( y \in K \) существуют различные открытые множества \( U_y \) и \( V_y \), такие что \( x \in U_y \) и \( y \in V_y \). Множество \( \{ V_y : y \in K \} \) является открытым покрытием \( K \), поэтому оно имеет конечное подпокрытие, скажем, \( \{ V_y : y \in F \} \), где \( F \) -- некоторое конечное подмножество множества \( K \). Положим:
\[ U = \bigcap_{y \in F} U_y; \]
очевидно, что \( U \) является открытым окрестностью \( x \), не пересекающейся с \( K \). Так как \( x \) была произвольной точкой из \( X \setminus K \), то \( K \) должно быть замкнутым.
\end{proof}

\subsection{Определение в метрическом пространстве}

В дальнейшем мы будем рассматривать многообразия, которые могут быть метризуемыми. Это означает, что на таких многообразиях можно ввести метрику, позволяющую работать с ними как с метрическими пространствами. В связи с этим имеет смысл сформулировать эквивалентное определение компактного множества в контексте метрических пространств. Однако прежде чем это сделать, напомним определение самого метрического пространства.

\begin{definition}[Метрическое пространство]
    Пусть \( M \) — множество. Метрикой на \( M \) называется функция \( d: M \times M \to \mathbb{R}_{\geq 0} \), удовлетворяющая следующим аксиомам:
    \begin{enumerate}
        \item \textbf{Невырожденность:} \( d(x, y) = 0 \) тогда и только тогда, когда \( x = y \);
        \item \textbf{Симметричность:} \( d(x, y) = d(y, x) \);
        \item \textbf{Субаддитивность:} \( d(x, y) \leq d(x, z) + d(z, y) \),
        \\ для любых точек \( x, y, z \in M \).
    \end{enumerate}
\end{definition}

Перед введением понятия открытого шара отметим, что в метрических пространствах для изучения окрестностей точек необходимо выделить базовое множество, позволяющее задавать такие окрестности. Этим множеством является открытый шар.

\begin{definition}[Открытый шар в метрическом пространстве]
	Пусть \( x \in M \) — точка в метрическом пространстве. \textbf{Открытый} \( \varepsilon \)-\textbf{шар} \( B_\varepsilon(x) \) с центром в \( x \) — это множество всех точек, отстоящих от \( x \) на расстояние меньше, чем \( \varepsilon \):
	\[
	B_\varepsilon(x) = \{ y \in M \mid d(x, y) < \varepsilon \}.
	\]
\end{definition}

Понятие ограниченного множества уже встречалось в курсе математического анализа, где оно играло важную роль при работе с числовыми множествами. В контексте метрических пространств это определение также имеет смысл, так как позволяет говорить об удалённости точек множества от заданной точки. Прежде чем продолжить, напомним формальное определение ограниченного множества в метрическом пространстве.

\begin{definition}[Ограниченность]
	Метрическое пространство \( (M,d) \) называется ограниченным, если 
\[ \exists R > 0 : \forall x \in M, \exists a \in M : d(x,a) \leq R \]
\end{definition}

\begin{statement}[Об ограниченности компакта в метрическом пространстве]
	Компактное подмножество метрического пространства ограниченно.
\end{statement}
\begin{proof}
	Пусть \( a \in M \). Пусть \( n \in \mathbb{N} > 0 \). Пусть \( B_n(a) \) -- открытый \( n \)-шар с центром в точке \( a \). Тогда \( C \subseteq \bigcup_{n=1}^{\infty} B_n(a) \), потому что для любой точки \( x \in C \) существует такое \( n \in \mathbb{N} \), что \( d(x, a) < n \).

Таким образом, множество \( \{ B_n(a) : n \in \mathbb{N} \} \) образует открытое покрытие \( C \). Поскольку \( C \) компактно, у него есть конечное подпокрытие, скажем: \( \{ B_{n_1}(a), B_{n_2}(a), \ldots, B_{n_r}(a) \} \). Пусть \( n = \max\{ n_1, n_2, \ldots, n_r \} \). Тогда:
\[ C \subseteq \bigcup_{n=1}^{r} B_{n_r}(a) = B_n(a) \]
Результат следует из определения ограниченности.
\end{proof}

\begin{theorem}[Критерий компактности в метрическом пространстве]
	Пространство \( X \) в \( \mathbb{R}^n \) является компактным тогда и только тогда, когда \( X \) замкнуто и ограничено.
\end{theorem}
\begin{proof}
	Воспользоваться доказанными утверждениями выше.
\end{proof}

\subsection{Связь компактности и непрерывных отображений}

\begin{definition}[Собственное отображение]
	Пусть \( f : X \to Y \) — отображение между топологическими пространствами. Говорят, что \( f \) является \textbf{собственным отображением}, если для любого компактного множества \( C \Subset X \) его образ \( f(C) \Subset Y \).
\end{definition}


\begin{statement}[Непрерывный образ компакта]
	Пусть \( K \Subset X \) -- компактное множество, и \( f : X \rightarrow Y \) -- непрерывная функция. Тогда множество \( f(K) \) является компактным.
\end{statement}
\begin{proof}
	Пусть \( f \) непрерывно. Возьмем любое открытое покрытие \( f(K) \). Так как \( f \) непрерывно, обратные образы этих открытых множеств образуют открытое покрытие \( K \). Так как \( K \) компактно, существует конечное подпокрытие. По построению образы этого конечного подпокрытия дают конечное подпокрытие \( f(K) \), поэтому \( f(K) \) также компактно.
\end{proof}

\begin{corollary}[Теорема Вейерштрасса]
	Если множество $A$ компактно, то непрерывная функция \( f : A \rightarrow \mathbb{R} \) принимает максимальное и минимальное значение.
\end{corollary}
\begin{proof}
	Множество \( f(A) \) является непрерывным образом компактного множества, поэтому оно является компактным подмножеством \( \mathbb{R} \). Следовательно, оно ограничено и имеет как верхнюю, так и нижнюю грани. Докажем, что верхняя грань достигается в \( A \). Обозначим верхнюю грань \( s \). По определению верхней грани вещественных чисел существует последовательность \( (y_n) \) точек из \( f(A) \), сходящаяся к \( s \). Но тогда, так как множество \( f(A) \) замкнуто, мы имеем \( s \in f(A) \).
\end{proof}

\begin{statement}
	Пусть \( f : X \rightarrow Y \) -- непрерывное биективное отображение. Если \( X\) компактно, а \( Y \) -- хаусдорфово, то \( f \) является гомеоморфизмом.
\end{statement}
\begin{proof}
	Пусть \( g = f^{-1} \). Нам нужно показать, что \( g : Y \rightarrow X \) непрерывно. Для любого \( V \subseteq X \) у нас есть \( g^{-1}(V) = f(V) \). Нам нужно показать, что если \( V \) замкнуто в \( X \), то \( g^{-1}(V) \) замкнуто в \( Y \). 
	
	Предположим, что \( V \) замкнуто в \( X \). Поскольку \( X \) компактно, \( V \) компактно. Таким образом, \( f(V) \) компактно и \( Y \) хаусдорфово, \( f(V) \) замкнуто. Но \( f[V] = g^{-1}(V) \), поэтому \( g^{-1}(V) \) замкнуто. Из того следует, что \( g \) непрерывно. Таким образом, по определению, \( f \) является гомеоморфизмом.
\end{proof}

\begin{remark}
	Непрерывное отображение из компакта \( X \) в хаусдорфово топологическое пространство \( Y \) является собственным.
\end{remark}

\begin{corollary}
	Непрерывное отображение из компакта \( X \) в хаусдорфово топологическое пространство \( Y \) является замкнутым.
\end{corollary}

\begin{definition}
	Пусть \( f : X \to Y \) — непрерывное отображение. Напомним, что для любой точки \( y \in Y \) прообраз \( f^{-1}(y) \) называется слоем \( f \).
\end{definition}

\begin{theorem}
	Пусть \( f : X \to Y \) — замкнутое, непрерывное отображение, при этом все слои \( f \) компактны. Тогда \( f \) является собственным.
\end{theorem}

\begin{proof}
	Пусть \( K \Subset Y \). Необходимо показать, что \( f^{-1}[K] \) компактно. Заменив \( Y \) на \( K \), а \( X \) на \( f^{-1}[K] \), можно считать, что \( Y \) компактно.

	\begin{remark}
		Компактность \( M \) эквивалентна следующему свойству: пусть \( \{A_{\alpha}\} \) — набор замкнутых подмножеств \( M \), такой, что любое конечное подмножество \( \{A_1, A_2, \dots, A_n\} \subset \{A_{\alpha}\} \) имеет общую точку. Тогда все \( A_{\alpha} \) имеют общую точку. 
	\end{remark}

	Действительно, отсутствие общей точки у \( \{A_{\alpha}\} \) означает, что \( \{M \setminus A_{\alpha}\} \) — покрытие, а наличие общих точек у \( \{A_1, A_2, \dots, A_n\} \) означает, что в \( \{M \setminus A_{\alpha}\} \) нет конечного подпокрытия.

	Добавив к \( \{A_{\alpha}\} \) все конечные пересечения элементов \( \{A_{\alpha}\} \), получим набор замкнутых подмножеств \( X \), обладающий тем же свойством. Поэтому можно считать, что \( \{A_{\alpha}\} \) содержит все конечные пересечения своих элементов.

	Пусть \( \{A_{\alpha}\} \) — набор замкнутых подмножеств в \( X \), такой что любое конечное подмножество \( \{A_1, A_2, \dots, A_n\} \subset \{A_{\alpha}\} \) имеет общую точку. Поскольку \( Y \) компактно, а все \( f(A_{\alpha}) \) замкнуты, то \( \{f(A_{\alpha})\} \) имеет общую точку \( y \in Y \).

	Пусть \( y = \alpha f(A_{\alpha}) \). Любое конечное пересечение \( \bigcap_i A_i \) лежит в \( \{A_{\alpha}\} \), значит, пересекается с \( f^{-1}[y] \). В силу компактности \( f^{-1}[y] \), из этого следует, что \( \{A_{\alpha} \cap f^{-1}[y]\} \) имеет общую точку.

	Таким образом, \( X \) компактен.
\end{proof}

Наконец, можно закрыть гештальт относительно проекций. Мы уже убедились, что проекция является открытым отображением. Теперь, для проекций, возникает интересная возможность рассмотреть еще одно важное свойство — замкнутость отображений. Однако для того чтобы это свойство было верно, нам нужно наложить некоторое ограничение на домен функции. В частности, будем рассматривать только проекции, определенные на компактных множествах, и тогда утверждение о замкнутости проекций будет истинным.

\begin{statement}[о замкнутости проекции компактного множества]
    Пусть \(\pr{X \times Y}{Y}\), причём \(X\) компактно. Тогда \(\pr{X \times Y}{Y}\) замкнуто.
\end{statement}

\begin{proof}
    Пусть \(Z \subset X \times Y\) — замкнутое подмножество. Если \(\pr{X \times Y}{Y}(Z)\) не замкнуто, то для какой-то предельной точки \(y \in \mathrm{pr}(Z)\) имеем \(\mathrm{pr}[y] \cap Z = \emptyset\).

    У каждой точки \((x, y) \in \mathrm{pr}^{-1}[y]\) есть окрестность \(U_{x, y} = V_{x, y} \times W_{x, y}\), не пересекающаяся с \(Z\).

    Поскольку \(\mathrm{pr}^{-1}[y]\) компактно, можем выбрать конечное покрытие \(\{V_i \times W_i\}\) множества \(\mathrm{pr}^{-1}[y]\), не пересекающееся с \(Z\).

    Множество \(\displaystyle X \times (\bigcup_i W_i)\) не пересекает \(Z\), значит, \(y\) не предельная точка \(\mathrm{pr}(Z)\).
\end{proof}

\subsection{Последовательности и секвенциальная компактность}

\begin{definition}[Последовательность в топологическом пространстве]
	Пусть \( X \) — топологическое пространство. Последовательность в \( X \) — это отображение \( x: \mathbb{N} \to X \), которое ставит в соответствие каждому натуральному числу \( n \) элемент \( x_n \in X \). Т.е. последовательность в \( X \) представляет собой последовательность элементов \( \{x_n\} \), где для каждого \( n \in \mathbb{N} \) \( x_n \in X \).
\end{definition}
Как и всегда после введения последовательности, необходимо ввести определение её предельного значения. 
\begin{definition}[Предел последовательности в топологическом пространстве]
	\( \{a_n\} \) — последовательность элементов пространства \( X \). Точка \( L \in X \) называется пределом последовательности \( \{a_n\} \), если:
	\[
		\forall U(L) \ \exists N \in \mathbb{N}\st \forall n > N \ a_n \in U(L),
	\]
	В этом случае говорят, что последовательность \( \{a_n\} \) сходится к \( L \), и пишут \(\displaystyle \lim_{n \to \infty} a_n = L \).
\end{definition}

Подход к изучению через последовательности широко применяется в математическом анализе и других областях математики. Этот метод позволяет формулировать и доказывать результаты, используя понятие предела последовательности, что значительно упрощает понимание и решение задач. Часто для известных понятий, даются формулировки именно через последовательности, которые в общем случаи не эквивалентны топологическим. 
Слово "последовательность" на латинском языке будет \textit{sequance}, и по этой причине подход, в котором используется анализ последовательностей, часто называют \textit{секвенциальным}.

\begin{definition}[Секвенциальная компактность]
	Топологическое пространство \(X\) называется секвенциально компактным, если из любой последовательности точек в \(X\) можно выделить сходящуюся подпоследовательность.
\end{definition}


\begin{statement}
	Из компактности следует секвенциальная компактность в пространствах со счётной базой. То есть, если \( M \) — компактное множество в \( X \), то всякая последовательность \( \{x_i\} \) в \( M \) имеет сходящуюся подпоследовательность.
\end{statement}

\begin{proof}
	Пусть \( \{x_i\} \) — последовательность точек множества \( M \subset X \). Рассмотрим множество \( \mathcal{R}_n = \{ x_n, x_{n+1}, x_{n+2}, \dots \} \) и его замыкание \( \cl{\mathcal{R}_n} \). Тогда пересечение всех таких множеств
	\[
		\bigcap_{i} \cl{\mathcal{R}_n}
	\]
	непусто. Ясно, что это пересечение состоит из предельных точек последовательности \( \{ x_i \} \).
\end{proof}

\begin{remark}
    Любая непрерывная функция \( f : K \to \mathbb{R} \) на секвенциально компактном множестве \( K \) принимает максимум и минимум. Это утверждение известно как теорема Вейерштрасса.
\end{remark}

\begin{remark}
    Для метрических пространств секвенциальная компактность эквивалентна компактности в обычном смысле. Это утверждение является следствием теоремы Гейне-Бореля-Лебега.
\end{remark}


\begin{remark}

	В общем случае из компактности не следует секвенциальная компактность, и наоборот. Приведённые примеры не являются частью основного курса, но служат для иллюстрации того, что в общем смысле из одного вида компактности не следует другой.	\begin{enumerate}
		\item Примером секвенциально компактного пространства, которое не является компактным, служит первое несчётное ординальное число с топологией порядка.
		\item Примером компактного пространства, которое не является секвенциально компактным, является топологический произведение \([0,1]^{\mathfrak{c}}\).
	\end{enumerate}
\end{remark}

\subsection{Компактификация Александрова}

Ранее мы рассматривали гомеоморфизм между сферой без одной точки и комплексной плоскостью. Однако, при удалении точки из сферы теряется её компактность.

Добавив точку бесконечности, мы превращаем комплексную плоскость в компактное пространство, где все последовательности, стремящиеся к бесконечности, теперь сходятся к этой новой точке. Интуитивно это можно представить как процесс сворачивания всей комплексной плоскости в сферу.

Таким образом, добавление одной точки превращает комплексную плоскость в компактное пространство, позволяя работать с бесконечностью в компактных рамках.
\begin{definition}
	Пусть \( X \) — топологическое пространство. Одноточечная компактификация пространства \( X \) — это расширение \( \hat{X} = X \cup \{\infty\} \), где добавляется одна новая точка \( \infty \), называемая точкой бесконечности. Топология \( \hat{\Omega} \) на \( \hat{X} \) строится следующим образом:
	
	\begin{itemize}
		\item открытые множества \( X \open \hat{X} \),
		\item множества вида \( \hat{X} \setminus K \), где \( K \Subset X \), являются открытыми в \( \hat{X} \).
	\end{itemize}
\end{definition}
\begin{statement}
	\((\hat{X}, \hat{\Omega})\) -- компактное топологическое пространство.
\end{statement}

\begin{proof}
	Рассмотрим топологическое пространство \( (\hat{X}, \hat{\Omega}) \), где \( \hat{X} = X \cup \{\infty\} \), и \( \hat{\Omega} \) — топология на \( \hat{X} \).

\( \displaystyle  \varnothing \) и \( \hat{X} \) лежат в \( \hat{\Omega} \) по определению топологии.

Рассмотрим множества вида \( \displaystyle \hat{X} \setminus K_i \), где \( K_i \) — компактные подмножества \( X \). Эти множества открыты в \( \hat{\Omega} \).

Объединение \(\displaystyle  \bigcup_{i \in \mathbb{I}} (\hat{X} \setminus K_i) = \hat{X} \setminus \bigcup_{i \in \mathbb{I}} K_i \) также открыто в \( \hat{\Omega} \), так как топология замкнута относительно объединений.

Для конечных объединений \(\displaystyle  \bigcup_{j = 1}^N (\hat{X} \setminus K_j) = \hat{X} \setminus \left( \bigcup_{j = 1}^N K_j \right) \) также справедливо, так как конечное объединение компактных подмножеств остаётся компактным.

Таким образом, пространство \( \hat{X} \) компактно, так как все необходимые объединения открыты в \( \hat{\Omega} \).

\end{proof}
	
\begin{example}
	Компактификация вещественной прямой \( \mathbb{R} \) — это окружность \( S^1 \). 
	
	Добавим к \( \mathbb{R} \) точку бесконечности \( \infty \). Её окрестностью будем считать множество вида \( (-\infty, -a) \cup (a, \infty) \), где \( a > 0 \), то есть открытое множество, состоящее из двух бесконечных интервалов. Дополнением к такой окрестности является множество \( \mathbb{R} \setminus U(\infty) = [-a, a] \), которое представляет собой отрезок, а каждый отрезок, как известно, является компактным.
	
	Аналогично, компактификация \( \mathbb{R}^n \) — это \( n \)-мерная сфера \( S^n \), получаемая путём добавления одной точки бесконечности \( \infty \). Для любой окрестности \( \infty \) её дополнение является замкнутым и компактным множеством в \( \mathbb{R}^n \).
	
	\end{example}
	
\begin{example}[двуточненая компактификация]
Рассмотрим пространство \( X = (a, b) \), где \( a, b \in \mathbb{R} \) и \( a < b \). Компактификация \( \hat{X} \) получается добавлением концов \( a \) и \( b \), превращая \( X \) в замкнутый отрезок \( \hat{X} = [a, b] \). В новой топологии окрестностями точек \( a \) и \( b \) являются полуинтервалы вида \( [a, a+\varepsilon) \) и \( (b-\varepsilon, b] \), соответственно, что делает \( \hat{X} \) компактным пространством.
\end{example}
	
	
\starsection{Задачи и упражнения}

\begin{task}
	Докажите, что любое конечное дискретное и любое антидискретное пространства компактны.
\end{task}

\begin{task}
	Приведите пример замкнутого и ограниченного подмножества метрического пространства, которое не будет являться компактным.
\end{task}

\begin{task}
	Показать, что куб \(I^n\) компактен, где куб задаётся следующим множеством: 
	\[
		I^n = \{x \in \mathbb{R}^n: \ x_i \in \left[0, 1\right] \ \text{для} \ i=1,\dots,n\}.
	\]
\end{task}

\begin{task}
    Пусть \( X \) — секвенциально компактное пространство, а \( \{A_n\} \) — последовательность непустых замкнутых множеств, упорядоченная по включению, то есть \( A_1 \supset A_2 \supset A_3 \supset \dots \). Докажите, что пересечение всех этих множеств непусто, то есть
    \[
    \bigcap_{n=1}^{\infty} A_n \neq \emptyset.
    \]
\end{task}

\begin{task}[Вновь Канторово множество]
    Докажите, что множество Кантора является компактным.
\end{task}

\begin{task}
	Покажите, что \(f: X \to Y\) -- собственное и непрерывное отображение, тогда и только тогда, когда \(\hat{f}\) -- непрерывное отображение между \(\hat{X}\) и \(\hat{Y}\), где:
	\begin{equation*}
		\hat{f}: x \mapsto \begin{cases}
			f(x) & x \in X, \\
			\hat{Y} \setminus Y & x = \hat{X} \setminus X.
		\end{cases}
	\end{equation*}
\end{task}

\begin{task}
	Явно опишите одноточечные компактификации следующих топологических пространств:
	\begin{enumerate}
		\item Кольцо \( \{(x, y) \in \mathbb{R}^2 \ | \ 1 < x^2 + y^2 < 2\} \).
		\item Квадрат без вершин \( \{(x, y) \in \mathbb{R}^2 \ | \ x, y \in [-1, 1], \ |xy| < 1\} \).
		\item Полоса \( \{(x, y) \in \mathbb{R}^2 \ | \ x \in [0, 1]\} \).
		\item Компактное пространство.
	\end{enumerate}
\end{task}	