\documentclass[a4paper, 12pt]{book}

\title{Размышления об общей топологии}
\author{Кумаллагов Д. З. \\ Муравья Н. Р.}

\usepackage{itmobook}
\usepackage{import}
\usepackage{xifthen}
\usepackage{pdfpages}
\usepackage{transparent}
\usepackage{tikz-cd}
\usepackage{titlesec}
\usepackage{tocloft}
\usepackage{titletoc}
\usepackage{pgfplots}
\usepackage{subcaption}
\usepackage{tikz-3dplot}
\usepackage{esint}
\usepackage{tikz-cd}
\usepackage{amsmath}  
\usepackage{MnSymbol}
\usepackage{pifont}
\usepackage{cmap} 
\usepackage{amssymb} 
\usepackage{ifthen}
\usepackage{etoolbox}
\usetikzlibrary{patterns}

\makeatletter 
\tikzset{ 
reuse path/.code={\pgfsyssoftpath@setcurrentpath{#1}} 
} 
\tikzset{even odd clip/.code={\pgfseteorule}, 
protect/.code={ 
\clip[overlay,even odd clip,reuse path=#1] 
(current bounding box.south west) rectangle (current bounding box.north east)
; 
}} 
\makeatother 
\usetikzlibrary{3d,arrows.meta,decorations.markings,perspective}
\tikzset{->-/.style={decoration={
  markings,
  mark=at position #1 with {\arrow{>}}},postaction={decorate}},
  ->-/.default=0.55}



\newcommand{\X}{\mathbf{X}}
\newcommand{\topX}{$\left(\X,\, \mathbf{\Omega_{\X}}\right)$}
\newcommand{\Y}{\mathbf{Y}}
\newcommand{\topY}{$\left(\Y,\, \mathbf{\Omega_{\Y}}\right)$}
\newcommand{\Z}{\mathbf{Z}}
\newcommand{\topZ}{$\left(\Z,\, \mathbf{\Omega_{\Z}}\right)$}

\let\oldforall\forall
\renewcommand{\forall}{\oldforall \, }
\let\oldexist\exists
\renewcommand{\exists}{\oldexist \: }
\renewcommand{\Im}{\mathrm{Im}}
\renewcommand{\Re}{\mathrm{Re}}
\newcommand*{\defeq}{\stackrel{\text{def}}{=}}
\DeclareMathOperator{\Der}{Der}


\newcommand{\st}{\; : \;}

\newcommand*{\open}{\raisebox{-0.75ex}{\begin{tikzpicture}
		\draw (0,0) node {$\subset$};
		\draw (0.1em, -0.575ex) circle (0.125em);
\end{tikzpicture}}}
\newcommand*{\closed}{\raisebox{-0.75ex}{\begin{tikzpicture}
			\draw (0,0) node {$\subset$};
			\draw[thick] (0.1em, -0.35ex) -- (0.3em, -0.85ex);
\end{tikzpicture}}}


\usetikzlibrary{decorations.markings, arrows.meta}


\tikzset{
    closed hook arrow/.style={
        {Hooks[harpoon, length=1mm]}->, 
        decoration={
            markings,
            mark=at position 0.5 with {
                \draw[-] (-0.2em, 0.2em) -- (0.2em, -0.2em); 
            },
        },
        postaction={decorate},
    }
}

\tikzset{
    open hook arrow/.style={
        {Hooks[harpoon, length=1mm]}->, 
        decoration={
            markings,
            mark=at position 0.5 with {
                \draw[fill=white, draw=black] (0em, -0.05em) circle (0.15em);
            },
        },
        postaction={decorate},
    }
}

\newcommand{\cl}[1]{\mathrm{Cl}\left(#1\right)}
\newcommand{\fr}[1]{\mathrm{Fr}\left(#1\right)}


\newcommand{\incfig}[2][1]{%
    \def\svgwidth{#1\columnwidth}
    \import{./figures/}{#2.pdf_tex}
}

\newcommand{\image}[2]{
    {\begin{center}
   \incfig[0.8]{#1}
   \captionof{figure}{#2 \label{#1}}  \par
 \end{center}}
}


\newcommand{\mysetminusD}{\hbox{\tikz{\draw[line width=0.6pt,line cap=round] (3pt,0) -- (0,6pt);}}}
\newcommand{\mysetminusT}{\mysetminusD}
\newcommand{\mysetminusS}{\hbox{\tikz{\draw[line width=0.45pt,line cap=round] (2pt,0) -- (0,4pt);}}}
\newcommand{\mysetminusSS}{\hbox{\tikz{\draw[line width=0.4pt,line cap=round] (1.5pt,0) -- (0,3pt);}}}


\renewcommand{\setminus}{\mathbin{\mathchoice{\mysetminusD}{\mysetminusT}{\mysetminusS}{\mysetminusSS}}}

\newcommand{\pr}[2]{\mathrm{pr}_{#1}(#2)}

\newcommand{\starsection}[1]{%
    \renewcommand{\thesubsection}{\Large \textcircled{\centering \small{\ding{72}}}}% Изменяем стиль номера
    \subsection{#1}% Используем стандартную subsection
    \renewcommand{\thesubsection}{\Large \textcircled{\centering \small{{\arabic{subsection}}}}}% Возвращаем стандартный стиль
}


\newcommand{\homeo}{\overset{\text{\tiny homeo}}{\cong}}
\begin{document}
	
	\renewcommand{\contentsname}{\hfillОГЛАВЛЕНИЕ\hfill} 
	\frontmatter
	\titlepage
	
	\doublespacing
	\tableofcontents
	\let\cleardoublepage\clearpage
	\singlespacing
	
	\mainmatter
	
	\pagestyle{style}
	
	
	\chapter{Общая топология}
	\epigraph{Узнать тебе пора, \\ Что при подъеме кажется сначала \\ Всегда крутою всякая гора. \\ Но выше мы взбираемся, и мало -  \\ Помалу путь удобнее идет,\\ И та тропа что силы отнимала, \\ Вдруг легкой станет..	\leavevmode
	}{\itshape ``Божественная комедия''\\ Данте Алигьери}

В математике существует множество разделов, которые кажутся сложными и оторванными от повседневной жизни. Однако, если копнуть глубже, выясняется, что эти абстрактные теории имеют широкий спектр применений: от физики и экономики до компьютерной графики и машинного обучения. Одним из таких разделов является \textbf{топология} — наука о свойствах пространств, которые сохраняются при их деформациях. 

\bigskip

В процессе изучения математического анализа мы не раз сталкивались с понятиями, которые принадлежат к топологии. Например: 

\begin{itemize}
    \item Определение предела функции \( \displaystyle \lim_{x \to a} f(x) \) требует понимания окрестностей точки.
    \item Теория непрерывности функции формально основана на свойствах открытых множеств.
    \item Понятие связности используется при изучении графиков функций и исследовании экстремумов.
\end{itemize}

Каждое из этих понятий уходит корнями в топологию — раздел математики, который изучает форму и структуру пространств.

\bigskip

История топологии начинается с задач, которые на первый взгляд не имеют ничего общего с математикой в привычном понимании. Одним из первых примеров использования топологических идей можно считать задачу о семи мостах Кёнигсберга, решённую Леонардом Эйлером в 1736 году. Эта задача, казавшаяся чисто географической, привела к созданию основы теории графов и пониманию структурных свойств объектов без привязки к их конкретной форме.

\bigskip

В XIX веке Бернхард Риман заложил основы анализа и геометрии, ввёл понятие \textbf{многообразия} и описал, как его можно наделить геометрической или топологической структурой. Это стало ключевым шагом в развитии математической физики, так как позволило изучать сложные поверхности, такие как сферы или торы, в абстрактной форме.

\bigskip

В начале XX века Анри Пуанкаре систематизировал знания о пространственных структурах и ввёл понятие гомеоморфизма — взаимно-однозначного и непрерывного отображения пространств, которое сохраняет их основные свойства. Его работы положили начало исследованию инвариантов, таких как число дыр или связность пространства. Знаменитая *гипотеза Пуанкаре*, сформулированная в 1904 году, оставалась одной из самых сложных нерешённых задач математики вплоть до её доказательства Григорием Перельманом в 2003 году.

\bigskip

Одним из ярких примеров, иллюстрирующих топологический подход, является известная шутка: ``Тополог не отличает кружку от бублика''. Это связано с тем, что с точки зрения топологии обе фигуры имеют одну дырку и могут быть преобразованы друг в друга через плавную деформацию.

\bigskip


Первая глава будет посвящена основам общей топологии. Мы познакомимся с такими ключевыми понятиями, как открытые множества, замыкания, базы топологий и связность. Эти определения послужат отправной точкой для более сложных конструкций, которые мы изучим в дальнейшем.

\bigskip


	
\section{Топология. Основные понятия}
\subsection{Первичное определение}
\begin{definition}[Топология]
	Топология на множестве $X$ -- это семейство подмножеств $\Omega$ множества $X$, удовлетворяющее следующим условиям:
	\begin{enumerate}
		\item Пустое множество и само множество $X$ принадлежат $\Omega$: $\emptyset, X \in \Omega$.
		\item Любое объединение элементов из $\Omega$ также принадлежит $\Omega$: если $A_\alpha \in \Omega$ для всех $\alpha$ из некоторого индексного множества, то $\bigcup_\alpha A_\alpha \in \Omega$.
		\item Любое пересечение конечного числа элементов из $\Omega$ также принадлежит $\Omega$: если $A_1, A_2, \ldots, A_n \in \Omega$, то $A_1 \cap A_2 \cap \ldots \cap A_n \in \Omega$.
	\end{enumerate}
\end{definition}

Так как большая часть последующих размышлений посвящена топологическим пространствам, то часто в дальнейшем мы будем опускать пару множество-топология и ограничимся лишь -- $X$.

\begin{definition}[Топологическое пространство]
	Топологическим пространством называется упорядоченная пара $(X, \Omega)$, где $X$ -- произвольное множество, а $\Omega$ -- топология на $X$.
\end{definition}

\begin{example}
	\begin{enumerate}
		
		\item Стандартная топология на вещественной числовой прямой $\mathbb{R}$ определяется множеством всех открытых интервалов. Это может быть записано следующим образом:

		\[
		\Omega_{\text{std}} = \{ (a, b) \mid a, b \in \bar{\mathbb{R}}, a < b \} \cup \mathbb{R}
		\]

		\item Пусть $X$ — произвольное множество. Дискретная топология на $X$ определяется следующим образом:  
		\[
		\Omega_{\text{disc}} = 2^X,
		\]  
		где $2^X$ обозначает булеан множества $X$, то есть множество всех подмножеств множества $X$.  
		

		\item Антидискретная топология на множестве \(X\) определяется следующим образом:

		\[
		\Omega_{\text{antidisc}} = \{ \varnothing, X\}
		\]
		
		\item Пусть \(X\) -- произвольное множество, а \(d\) -- метрика на \(X\). Метрическая топология на \(X\) порождается метрикой \(d\) следующим образом:

		\[
		\Omega_d = \{ B(x, r) \mid x \in X, r > 0 \}
		\]
		
		где \(B(x, r)\) обозначает открытый шар с центром в точке \(x\) и радиусом \(r\).
		
		\item  Топология Зарицкого на множестве \(\mathbb{R}\), обозначаемая \(\Omega_{\text{zar}}\), состоит из подмножеств \(A\) множества \(\mathbb{R}\), таких что \(A = \emptyset\) или \(X \setminus A\) является конечным множеством. Формально:

		\[
		\Omega_{\text{zar}} = \{ A \subseteq \mathbb{R} \mid A = \emptyset \text{ или } X \setminus A \text{ конечно} \}
		\]

		\item Топология двоеточие Александрова или же топология Серпинского на множестве \(X = \{\circ, \bullet\}\), обозначаемая \(\Omega_{:A}\), где \(\circ\) -- открыто:
		\[
			\Omega_{:A} = \{\varnothing, \circ, \{\bullet, \circ\}\}
		\]

		\item Топология Зоргенфрея или же топология стрелки на множестве \(X = [a, +\infty)\), \(a \geq 0\):
		\[
			\Omega_{l} = \{\varnothing, X\} \cup \{(a, +\infty): \ \forall a \geq 0 \in \mathbb{R}\}
		\]
		\item Пусть $(X, \Omega)$ -- топологическое пространство, а $Y$ -- множество, полученное из $X$ добавлением к нему одного элемента $p$, если \(\Omega_x\) -- дискретна, то следующая топология называется топологией всюду плотной точки:
		\[
			\Omega_p = \left\{ \{p\} \cup U \ \middle| \ U \in \Omega \right\} \cup \{ \emptyset \}	
		\]
	\end{enumerate}
\end{example}

\subsection{Рассуждение о типах множеств}

Если $(X, \Omega)$ -- топологическое пространство, то элементы $X$ называются точками, а элементы множества $\Omega$ -- открытыми множествами. Открытое множество $U \in \Omega$  будем обозначать $U \open X$.

\( U \) для открытых множеств происходит от немецкого слова "Umgebung"\ , что в переводе означает "окрестность". Это объясняется тем, что несколько выдающихся немецких математиков, таких как Георг Кантор и Феликс Хаусдорф, сыграли важную роль в развитии топологии. 


\begin{remark}
	Греческая буква \(\Omega\) -- это аналог буквы О, используемый в различных языках для обозначения одного и того же понятия. Например, в английском это "open"\ , в русском "открыто"\ , в немецком "offen" \ , во французском "ouvert" \ .
\end{remark}

\begin{definition}[Замкнутное множество]
	Говорят, что множество \(F \subseteq X\) замкнуто в пространстве \((X, \Omega)\), если его дополнение \(X \setminus F\) открыто, то есть если \(X \setminus F \in \Omega\). Будем обозначать следующим образом: $F \closed X$.
\end{definition}

Буква \( F \), происходящая от французского слова "fermé" \ (замкнутое), традиционно используется для обозначения замкнутых множеств. 


\begin{remark}
	Обратите внимание на то, что замкнутость не есть отрицание открытости.
\end{remark}

\begin{example}[Множетсво, которое ни открыто, ни замкнуто]
Рассмотрим множество вещественных чисел $\mathbb{R}$ с обычной топологией. Пусть $A = [0, 1)$.  

Заметим, что множество $A$ не является открытым, так как точка $0$ не имеет окрестности, содержащейся в $A$. Однако $A$ также не является замкнутым, так как его дополнение $\mathbb{R} \setminus A = (-\infty, 0) \cup [1, \infty)$ не является открытым (точка $1$ не имеет окрестности, содержащейся в дополнении).  

Таким образом, $A$ является примером множества, которое ни открыто, ни замкнуто, что иллюстрирует, что замкнутость не равнозначна отрицанию открытости.

\end{example}

Замкнутость и открытость во многом аналогичные свойства. Фундаментальное различие между ними состоит в том, что пересечение бесконечного набора открытых множеств не обязательно открыто, тогда как пересечение любого набора замкнутых множеств замкнуто, а объединение бесконечного набора замкнутых множеств не обязательно замкнуто, тогда как объединение любого набора открытых множеств открыто.

\subsection{Иерархия топологий}
\begin{definition}[Иерархия тополгий]
	Пусть \( \Omega_1 \) и \( \Omega_2 \) — топологические структуры на множестве \( X \), причём \( \Omega_1 \subseteq \Omega_2 \). Тогда говорят, что структура \( \Omega_2 \) \textbf{тоньше} (или \textbf{тоньше, чем}) структура \( \Omega_1 \), а структура \( \Omega_1 \) \textbf{толще} (или \textbf{грубее, чем}) структура \( \Omega_2 \).
\end{definition}


\bigskip

\begin{example}
	Рассмотрим множество \( X = \{a, b, c\} \). В этом множестве можно задать следующие топологии:

	\begin{enumerate}
		\item Антидискретная топология: \( \Omega_{\text{грубая}} = \{\varnothing, X\} \).
		\item Промежуточная топология: \( \Omega_{\text{средняя}} = \{\varnothing, \{a\}, X\} \).
		\item Дискретная топология: \( \Omega_{\text{тонкая}} = 2^X \), где \( 2^X \) —- булеан множества \( X \), содержащий все его подмножества.
	\end{enumerate}
	
	Здесь \(\Omega_{\text{грубая}} \subseteq \Omega_{\text{средняя}} \subseteq \Omega_{\text{тонкая}}\), то есть грубая топология является самой слабой в терминах различения подмножеств, а тонкая — самой сильной.	
\end{example}
\bigskip

\begin{remark}
	Стоит отметить, что термины «грубая» и «тонкая» топологии иногда воспринимаются неоднозначно. Некоторые интерпретируют грубую топологию как «максимально простой подход», а тонкую — как «максимально сложный и детализированный». Однако такая интерпретация не совсем точна. Грубая топология задаёт минимальную структуру, обеспечивая самые общие свойства, например, для изучения связности или компактности. Тонкая топология, напротив, отличается высокой детализацией, позволяя учитывать особенности каждого элемента множества.  
\end{remark}

\begin{example}
	На пространстве функций топология равномерной сходимости является сильнее топологии поточечной сходимости: она задаёт более тонкую структуру, требуя согласованности сходимости на всех точках одновременно, а не только на отдельных точках. 
\end{example}

\begin{definition}
	Если на множестве заданы топологии \(\Omega_1\) и \(\Omega_2\), при этом \(\Omega_1\) не слабее или не сильнее \(\Omega_2\), то говорят, что \(\Omega_1\) и \(\Omega_2\) \textbf{несравнимы}.
\end{definition}

\starsection{Задачи и упражнения}

\begin{task}
	Определите, является ли заданное множество подмножеств топологией на соответствующем множестве:  
\begin{enumerate}
    \item На $X = \{a, b, c\}$ задано $\Omega = \{\emptyset, \{a\}, \{a, b, c\}\}$.  
    \item На $X = \mathbb{R}$ задано $\Omega = \{\emptyset, \mathbb{R}, (-\infty, a], [b, +\infty) \mid a, b \in \mathbb{R}\}$.  
    \item На $X = \mathbb{Z}$ пусть задано семейство $\Omega$ -- объединение всех конечных подмножеств целых чисел, включая само множество $\mathbb{Z}$.  
\end{enumerate}  
\end{task}

\begin{task}
	Рассмотрите следующие топологии на множестве $X = \mathbb{R}$:  
	\begin{enumerate}
		\item Дискретная топология.  
		\item Топология Зарицкого.  
		\item Стандартная топология.  
		\item Антидискретная топология.
	\end{enumerate}
	Упорядочите эти топологии.
\end{task}
	

\begin{task}[Определение топологии через замкнутые множества]
	Переформулируйте определение топологического пространства в терминах замкнутых множеств.  \\ 
	\textbf{Указание:} Воспользуйтесь законами Август де Морган.
\end{task}


\begin{definition}[Канторово множество]
	Канторово множество — это множество, полученное из отрезка $[0,1]$ последовательным удалением средних третьей части каждого оставшегося интервала. Формально:  
\[
\mathcal{C} = [0,1] \setminus \bigcup_{n=1}^\infty \bigcup_{k=1}^{2^{n-1}} \left( \frac{3k-2}{3^n}, \frac{3k-1}{3^n} \right).
\]  

\end{definition}

\begin{task}
	Докажите, что Канторово множество $\mathcal{C}$ замкнуто.  
\end{task}



\newpage
\section{База топологии}
\subsection{Определение}
Часто топологическую структуру задают посредством описания некоторой её части, достаточной для восстановления всей структуры. Например, топологию можно задать указанием базы топологии или системы окрестностей, которые определяют топологическую структуру.
\begin{definition}[База топологии]
	\textbf{Базой топологии} называется некоторый набор открытых множеств \( \mathcal{B} \), такой, что всякое непустое открытое множество представимо в виде объединения множеств из этого набора:
	\begin{equation*}
		\forall U \in \Omega:\  U = \bigcup_{i \in \mathbb{I}} B_i \quad B_i \in \mathcal{B}
	\end{equation*}
\end{definition}

\begin{statement}[Критерий базы]
	Пусть $\mathcal{B} \subseteq 2^X$, удовлетворяющее двум условиям:
	\begin{enumerate}
		\item Для каждой точки \( x \in X \) существует такое множество \( B \in \mathcal{B} \), что \( x \in B \)
		\item Для любых двух элементов \( B_1 \) и \( B_2 \) из \( \mathcal{B} \) и для любой точки \( x \), которая принадлежит пересечению \( B_1 \cap B_2 \), существует элемент \( B_3 \) из \( \mathcal{B} \), такой что \( x \in B_3 \) и \( B_3 \subseteq B_1 \cap B_2 \)
	\end{enumerate}
	тогда $\mathcal{B}$ -- база некоторой однозначно определённой топологии.
\end{statement}

\begin{proof}
	Пусть $\Omega = \left\{\bigcup B_\alpha: \ B_\alpha \in \mathcal{B}\right\}$, из пункта $1$ ясно, что $\varnothing \in \Omega$ и $X \in \Omega$. В объединение находятся $U_1, U_2 \in \Omega$; $U_1 \cap U_2 = \bigcup_j \Omega_j$
\end{proof}

\begin{example}
	Рассмотрим несколько примеров баз топологии:
	\begin{enumerate}
		\item \textbf{База стандартной топологии на \( \mathbb{R} \):} набор всех интервалов вида \( (a, b) \), где \( a, b \in \mathbb{R} \), \( a < b \). Любое открытое множество на прямой \( \mathbb{R} \) можно представить как объединение таких интервалов.
		\item \textbf{База топологии Зарисского на \( \mathbb{R}^n \):} набор множеств вида \( \mathbb{R}^n \setminus V \), где \( V \) — множество нулей некоторого конечного набора многочленов. Эта топология используется в алгебраической геометрии.
		\item \textbf{База дискретной топологии на произвольном множестве \( X \):} набор всех одноэлементных подмножеств \( \{x\} \), где \( x \in X \). Любое открытое множество в дискретной топологии является объединением этих одноэлементных множеств.
	\end{enumerate}
\end{example}


\begin{definition}[Предбаза топологии]
	\textbf{Предбазой топологии} на множестве называется некоторый набор открытых подмножеств \( \{U_\alpha\} \), такой, что любое открытое множество можно представить в виде объединения конечных пересечений элементов из базы топологии:
	\begin{equation*}
		\forall U \in \Omega:\  U = \bigcup_{i \in \mathbb{I}} \bigcap_{j \in \mathbb{J}_i} U_{\alpha_j}, \quad U_{\alpha_j} \in \{U_\alpha\},
	\end{equation*}
	где \( \mathbb{I} \) и \( \mathbb{J}_i \) — индексы объединений и пересечений соответственно.
\end{definition}

\begin{remark}
	Любой набор множеств \( \{U_\alpha\} \subseteq 2^M \), такой, что \( \bigcup_\alpha U_\alpha = M \), является предбазой некоторой топологии на \( M \).
\end{remark}

\begin{remark}
	Любой набор множеств \( \{U_\alpha\} \subseteq 2^M \), замкнутый относительно конечных пересечений и такой, что \( \bigcup_\alpha U_\alpha = M \), является базой некоторой топологии на \( M \).
\end{remark}



\starsection{Задачи и упражнения}
\begin{task}
	Существуют ли примеры различных топологических структур, которые обладают одной и той же базой? Подумайте, какие особенности могут при этом сохраняться.
\end{task}

\begin{task}
	Составьте базы для следующих топологических пространств, стараясь выбрать минимально возможные наборы: 
\begin{itemize}
    \item дискретное пространство;
    \item антидискретное пространство;
    \item топология стрелки.
\end{itemize}
Объясните ваш выбор.
\end{task}

\begin{task}
	Существуют ли топологические структуры, у которых может быть лишь одна возможная база? Найдите такие примеры и объясните их уникальность.
\end{task}

	

\begin{task}
	Рассмотрим три набора подмножеств плоскости \( \mathbb{R}^2 \), которые описывают круги в различных нормированных пространствах:
	\begin{enumerate}
		\item Открытые круги в евклидовой норме.
		\item Открытые квадраты без граничных точек, стороны которых параллельны координатным осям.
		\item Открытые квадраты без граничных точек, стороны которых параллельны биссектрисам координатных углов.
	\end{enumerate}
	Докажите следующие утверждения:
	\begin{enumerate}
		\item Любой открытый круг в евклидовой норме можно представить как объединение открытых квадратов в норме Чебышёва.
		\item Пересечение любых двух открытых квадратов в Манхэттенской норме является объединением других квадратов в норме Чебышёва.
		\item Каждый из описанных наборов служит базой некоторой топологии на \( \mathbb{R}^2 \), и все три топологии совпадают.
	\end{enumerate}
\end{task}


\newpage
\section{Топология произведения, подпространство}
\subsection{Топология подпространства}

\begin{definition}[Индуцированная топология]
	Пусть \((X, \Omega)\) — топологическое пространство, а \(S \subset X\) — подмножество \(X\). \textbf{Индуцированная топология} (или подпространственная топология) на \(S\) определяется как множество всех пересечений открытых множеств в \(X\) с \(S\):
\[
\Omega_S = \{ S \cap U \mid U \in \Omega \}.
\]
Это означает, что множество \(V \subset S\) является открытым в индуцированной топологии, если существует множество \(U \open \Omega\) такое, что \(V = S \cap U\).
\end{definition}

Индуцированная топология вводится для того, чтобы наделить подмножество топологического пространства собственной топологией, согласованной с исходным пространством. Эта концепция позволяет нам рассматривать подмножество как самостоятельное топологическое пространство, сохраняя при этом естественную связь с исходным. Интуитивно, индуцированная топология обеспечивает минимальное количество открытых множеств в подмножестве, чтобы топологическая структура оставалась совместимой с исходной топологией.

\begin{example}
	Рассмотрим вещественную прямую \(\mathbb{R}\) с её топологией \(\Omega_{\text{std}}\). Пусть \(S = [0, 1]\) — замкнутый отрезок на \(\mathbb{R}\). Индуцированная топология \(\Omega_S\) на \(S\) состоит из множеств вида \(S \cap U\), где \(U \open \mathbb{R}\). Например:
	\begin{enumerate}
		\item Если \(U = (0.5, 0.7)\), то \(S \cap U = (0.5, 0.7)\).
		\item Если \(U = (0, 2)\), то \(S \cap U = (0, 1]\).
		\item Если \(U = (-1, 0.5)\), то \(S \cap U = [0, 0.5)\).
		\item Если \(U = (-2, 1)\), то \(S \cap U = [0, 1]\).
	\end{enumerate}
	
Таким образом, открытые множества в \([0, 1]\) в индуцированной топологии могут быть промежутками любого вида и их объединениями, где \(0 \leq a < b \leq 1\).

\end{example}

Из приведённого примера видно, что открытые множества в индуцированной топологии не всегда являются открытыми в исходном пространстве. 

Однако для замкнутых множеств ситуация другая: замкнутые множества в индуцированной топологии всегда остаются замкнутыми в исходном пространстве. 

\begin{statement}
	Пусть \( (X, \Omega) \) — топологическое пространство, \( S \subset X \), а \( \Omega_S \) — индуцированная топология на \(S\). Тогда:
	\begin{enumerate}
		\item \(F \closed S \iff F = S \cap E\), где \(E \closed X\).
		\item Если \(S \open X \Longrightarrow \Omega_S \subset \Omega_X\).
	\end{enumerate}
\end{statement}
% \begin{proof}
% 	\begin{enumerate} 
% 		\item \ \begin{itemize}
% 			\item[\(\boxed{\Longrightarrow}\)] 
% 			\item[\(\boxed{\Longleftarrow}\)]
% 		\end{itemize}
% 		\item 
% 	\end{enumerate}
% \end{proof}

\subsection{Топология коробки}

Выше мы рассмотрели идею, как можно углубиться внутрь исходного пространства с помощью индуцированной топологии. Мы изучили, как выделить подпространство и определить на нём новую топологию, основываясь на открытых множествах объемлющего пространства. Этот процесс позволяет детально изучить внутреннюю структуру подпространства.

Однако возникает вопрос: можем ли мы расшириться вверх? Можем ли мы, как говорят в линейной алгебре, повысить размерность пространства? На самом деле, это возможно, если ввести понятие топологии произведения.

\begin{definition}[Топология произведения]
	Пусть \(\displaystyle X = \prod_{i \in I} X_i\) -- декартово произведение топологических пространств \(X_i\), индексированных множеством \(i \in I\). \textbf{Топология коробки} (топология произведения) на \(X\) задаётся базой:
		\[
		\mathcal{B} = \left\{ \prod_{i \in I} U_i \mid U_i \open X_i \right\}.
		\]
\end{definition}
Эта топология позволяет изучать свойства пространств более высокой размерности и исследовать их взаимосвязи через произведение координатных пространств. Интересное наблюдение, что	Декартову прямоугольную систему координат можно естественным образом представить как произведение двух координатных пространств \(X\) и \(Y\). В данном случае пространство \(X\) соответствует оси абсцисс, а пространство \(Y\) — оси ординат.
\begin{remark}
	Название \textit{топология коробки} происходит от случая, когда рассматривается пространство \(\mathbb{R}^n\), где элементы базы выглядят как прямоугольные "коробки".
\end{remark}


\begin{remark}
	Топология коробки является самой интуитивной топологией произведения, поскольку она естественно возникает при рассмотрении множества всех декартовых произведений открытых множеств. Однако, это не единственная возможная топология для произведения топологических пространств. 

Существует другая важная топология, известная как \textbf{Тихоновская топология}, в отличие от топологии коробки, она применяется для .
\end{remark}

\starsection{Задачи и упражнения}
\begin{task}
	Рассмотрим множество натуральных чисел \(\mathbb{N}\) как подмножество вещественной прямой \(\mathbb{R}\) с её стандартной топологией. Докажите, что подпространственная топология на \(\mathbb{N}\) является дискретной. 
\end{task}

\begin{task}

	Рассмотрим множество рациональных чисел \(\mathbb{Q}\) как подмножество вещественной прямой \(\mathbb{R}\) с её стандартной топологией.
	\begin{enumerate}
		\item Покажите, что синглетон \(\{0\}\) не является открытым в подпространственной топологии на \(\mathbb{Q}\).
		\item Докажите, что если \(a, b \in \mathbb{Q}\), то интервалы \((a, b)\) и \([a, b]\) соответственно открыты и замкнуты в \(\mathbb{Q}\).
		\item Покажите, что если \(a, b \in \mathbb{R} \setminus \mathbb{Q}\), то множество всех рациональных чисел \(x\), таких что \(a < x < b\), одновременно открыто и замкнуто в \(\mathbb{Q}\).
	\end{enumerate}
\end{task}


\newpage
\section{Точки множества}

\subsection{Точка относительно множества}

\begin{definition}[Окрестность точки]
	Пусть \((X, \Omega)\) -- топологическое пространство, \(x \in X\) -- точка. Окрестностью точки \(x\) называется любое открытое множество \(U \in \Omega\), такое что \(x \in U\).
\end{definition}

Аналитики и представители французской школы математики (по наследию Н. Бурбаки) понимают понятие окрестности шире, чем принято в классической традиции: они называют окрестностью любое множество, включающее в себя окрестность в более узком, традиционном смысле.\footnote{Строчка памяти Олега Александровича Иванова (1951-2019), выдающегося математика, профессора, кандидата физико-математических наук, заведующего кафедрой общей информатики мехмата СПБГУ}
\begin{definition}[Типы точек множества]
	В любом множестве всегда есть следующие точки:
	\begin{enumerate}
		\item \textbf{Внутренняя точка \(x\) множества \(A\):} Точка \(x\) является внутренней точкой множества \(A\), если существует окрестность \(U\) точки \(x\), целиком содержащаяся в \(A\), то есть \(U \subseteq A\).
		
		\item \textbf{Точка прикосновения \(x\) множества \(A\):} Точка \(x\) является точкой прикосновения множества \(A\), если для любой окрестности \(U\) точки \(x\) выполнено \(U \cap A \neq \emptyset\).
		
		\item \textbf{Граничная точка \(x\) множества \(A\):} Точка \(x\) является граничной точкой множества \(A\), если для любой окрестности \(U\) точки \(x\) выполнено \(U \cap A \neq \emptyset\) и \(U \cap (X \setminus A) \neq \emptyset\).
	\end{enumerate}
\end{definition}

\subsection{Операции на множествах}

Пусть \( (X, \Omega)\) -- топологическое пространство, а \( A \subseteq X \) -- произвольное подмножество.
\begin{definition}[Внутренность множества]
	Внутренностью множества \( A \) называется наибольшее (по включению) открытое множество:
	\begin{equation*}
		\mathrm{Int}(A) = \bigcup_{U \subseteq A} U, \quad U \open \Omega.
	\end{equation*}
		Внутренность множества \( A \) обозначается символом \( \text{Int}( A) \).
\end{definition}

\begin{remark}
	\(\text{Int} (A) \) происходит от французского слова "intérieur" и английского "interior"\ , что аналогично русскому слову "интерьер" \ , которое обозначает вид изнутри на какой-либо объект.
\end{remark}

Внешностью множества называется наибольшее не пересекающиеся с ним открытое множество. Ясно, что внешность $A$ совпадает с $\mathrm{Int}(X \setminus A)$

\begin{definition}[Замыкание множество]
	 Замыканием множества \( A \) называется наименьшее содержащее его замкнутое множество:
	 \begin{equation*}
		\mathrm{Cl}(A) = \bigcap_{F \supseteq A} F, \quad F \closed \Omega.
	 \end{equation*}
	Замыкание множества \( A \) обозначается символом \( \cl{A} \).
\end{definition}


\begin{remark}
\( \text{Cl}_X \, A \) происходит от французского слова "clôture" и английского "closure"\ , что аналогично русскому слову "ограждение"\ , которое обозначает предел, ограничивающий или замкнувший пространство.
\end{remark}



\begin{definition}[Граница множества]
	Границей множества \( A \) называется множество \( \cl{A} \setminus \mathrm{Int}(A) \). Обозначается граница множества \( A \) символом \( \fr{A} \) или \(\partial A\).
\end{definition}

\begin{remark}
\( \text{Fr}_X \, A \) происходит от французского слова "frontière" и английского "frontier"\ , что аналогично русскому слову "фронт"\ , которое обозначает передовую линию или границу между различными территориями или областями.
\end{remark}

\begin{theorem}[Эквивалетные опредления через точки множества]

	Для связи точек и элементов множества запишем следующее:

	\begin{enumerate}
		\item Внутренностью всякого множества является множество его внутренних точек.
		
		\item Замыкание множества \(A\) совпадает с множеством его точек прикосновения.
		
		\item Граница множества совпадает с множеством его граничных точек.
	\end{enumerate}
	
\end{theorem}

\begin{proof}
	\begin{enumerate}
		\item Пусть точка \( x \) внутренняя, т.е. существует открытое множество \( U_x \) такое, что \( x \in U_x \subseteq A \). Тогда \( U_x \subseteq \text{Int} \, A \) (поскольку \( \text{Int} \, A \) - наибольшее из всех открытых множеств, содержащихся в \( A \)), а, значит, и \( x \in \text{Int} \, A \). Обратно, если \( x \in \text{Int} \, A \), то само множество \( \text{Int} \, A \) и есть содержащаяся в \( A \) окрестность точки \( x \).
		\item Если \( x \notin \text{Cl} \, A \), то найдётся такое замкнутое множество \( F \), что \( F \supseteq A \) и \( x \notin F \). Значит, \( x \notin \text{Int} \, (X \setminus F) \), таким образом, \( x \) не является точкой прикосновения множества \( A \).
		\item Если \( x \in \text{Fr} \, A \), то \( x \in \text{Cl} \, A \setminus \text{Int} \, A \). Значит, во-первых, всякая окрестность точки \( x \) пересекается с \( A \), во-вторых, никакая окрестность этой точки не содержится в \( A \), следовательно, всякая окрестность пересекается с дополнением множества \( A \). Таким образом, точка \( x \) является граничной.
	\end{enumerate}
\end{proof}

\subsection{Всюду плотное и нигде не плотное множество}

\begin{definition}[Всюду плотное множество]
	Пусть \( A \) и \( B \) — подмножества топологического пространства \( X \). Говорят, что \( A \) плотно в \( B \), если 
\[
\text{Cl}(A) \supseteq B,
\]
и что \( A \) всюду плотно, если 
\[
\text{Cl}(A) = X.
\]
\end{definition}

Если множество \( A \) плотно в \( B \), это означает, что любой элемент из \( B \) можно аппроксимировать элементами из \( A \). Это свойство используется при работе с функциональными пространствами и в численных методах. Например, идея плотности множества функций в пространстве непрерывных функций лежит в основе различных разложений, таких как ряды Тейлора, тригонометрические ряды Фурье. Все эти разложения базируются на интуиции, что соответствующие множества функций являются всюду плотными в пространстве \( C[a, b] \).

\begin{theorem}[Критерий всюду плотного множества]
	Множество \( A \subseteq X \) является всюду плотным в топологическом пространстве \( X \) тогда и только тогда, когда для любого непустого открытого множества \( U \subseteq X \) выполняется \( A \cap U \neq \emptyset \).
\end{theorem}
\begin{proof} \
	\begin{itemize}
		\item[\(\boxed{\Longrightarrow}\)] Для каждой точки \( x \in X \) существует непустое открытое множество \( U \subseteq X \), такое что \( x \in U \). Поскольку \( A \) всюду плотно, то \( A \cap U \neq \emptyset \), что означает, что \( x \in \cl{A} \). Следовательно, \( \cl{A} = X \).
		\item[\(\boxed{\Longleftarrow}\)]  Если \( A \) пересекается с каждым непустым открытым множеством, то \( A \subseteq \cl{A} = X \). 
	\end{itemize}
\end{proof}

\begin{definition}[Нигде не плотное множество]
	Множество \( A \subseteq X \) называется нигде не плотным, если его внешность всюду плотна.
\end{definition}

\begin{theorem}[Эквивалетное определение]
	Множество \( A \subseteq X \) называется нигде не плотным, если и только если для каждой точки \( x \in X \) и для каждой окрестности \( U \) точки \( x \) существует точка \( y \in U \setminus A \), такая что некоторая окрестность точки \( y \) полностью содержится в \( U \setminus A \).
\end{theorem}
\begin{proof}
	Рассмотрим произвольную точку \( x \in X \) и её окрестность \( U \subseteq X \). Согласно условию, в любой окрестности \( U \) точки \( x \) существует точка \( y \in U \setminus A \), такая что для неё существует окрестность \( V \), удовлетворяющая \( V \subseteq U \setminus A \). Таким образом, для любой окрестности \( U \) точки \( x \) существует точка \( y \), которая находится в дополнении множества \( A \) с некоторой окрестностью. Это означает, что \( \text{Int}(\cl{A}) = X \), и следовательно, внешность множества \( A \) всюду плотна.
\end{proof}

\bigskip
\starsection{Задачи и упражнения}

\begin{task}[Свойства функции внутренности]
	Докажите, что для любой топологии на множестве \( X \) функция, задаваемая правилом: 
\[
f(A) = \text{Int}(A),
\]  
где \( A \subseteq X \), является:

\begin{enumerate}
	\item \textbf{идемпотентной}, то есть для любого множества \( A \subseteq X \) выполняется равенство:  
	\[
	\text{Int}(\text{Int}(A)) = \text{Int}(A).
	\]	
	\item \textbf{монотонной}, то есть для любых множеств \( A, B \subseteq X \), таких что \( A \subseteq B \), выполняется неравенство:  
	\[
	\text{Int}(A) \subseteq \text{Int}(B).
	\]	
	\item \textbf{супераддативой} относительно объединения, то есть для любых множеств \( A, B \subseteq X \) выполняется неравенство:  
	\[
	\text{Int}(A \cup B) \supseteq \text{Int}(A) \cup \text{Int}(B).
	\]  
	\item \textbf{мультипликативной} относительно пересечения, то есть для любых множеств \( A, B \subseteq X \) выполняется равенство:  
	\[
	\text{Int}(A \cap B) = \text{Int}(A) \cap \text{Int}(B).
	\]
\end{enumerate}
\end{task}
\begin{task}
	Аналогично, проведите анализ свойств функции \( \text{Cl}(A) \), определяющей замыкание множества \( A \subseteq X \). 
\end{task}



\begin{task}[Задача замыкания и дополнения]
	Найдите максимальное количество различных множеств, которые можно получить из одного множества \( A \subseteq X \), применяя только операции замыкания (\( \text{Cl} \)) и внутренности (\( \text{Int} \)) в различных комбинациях.  
\end{task}
Задача замыкания и дополнения, также известная как задача Куратовского\footnote{Казимеж Куратовский (1896–1980) — польский математик, основоположник современной топологии. В 1921 году он получил степень доктора философии за свою работу, в которой аксиоматически построил топологию через аксиомы замыкания. Его диссертация, опубликованная в 1922 году, оказала значительное влияние на развитие топологии и теории множеств.}, является классическим упражнением в общей топологии.

\begin{task}[О всюду плотном множестве]
	Доказать, что множество \(\mathbb{Q}\) всюду плотно \(\mathbb{R}\).
\end{task}

\begin{task}[О нигде не плотном множестве]
	Показать, что прямая \(\mathbb{R}\) нигде не плотна на плоскости \(\mathbb{R}^2\).
\end{task}
\begin{task}
    Рассмотрите множество Кантора \( \mathcal{C}\). Подумайте, что является внутренностью этого множества, а что является замыканием.
\end{task}




\newpage

\newpage
\section{Аксиомы топологии}

\subsection{Аксиомы отделимости}
В данном параграфе речь пойдет о первом топологическом инварианте, связанном с тем, как мы можем отделить какие-то множества или же в простейших случаях точки друг от друга.
\begin{enumerate}
	\item \textbf{Т0-аксиома (Колмогоров)}: Для любых двух различных точек \( x \) и \( y \) из множества \( X \) существует открытое множество \( U \) такое, что либо \( x \in U \), \( y \notin U \), либо \( y \in U \), \( x \notin U \).
    \item \textbf{Т1-аксиома (Тихонова-Фреше)}: Для любых двух различных точек \( x \) и \( y \) из множества \( X \) существуют открытые множества \( U \) и \( V \), такие что \( x \in U \), \( y \notin U \), \( y \in V \), \( x \notin V \).
    
    \item \textbf{Т2-аксиома (Хаусдорф)}: Для любых двух различных точек \( x \) и \( y \) из множества \( X \) существуют открытые множества \( U \) и \( V \), такие что \( x \in U \), \( y \in V \) и \( U \cap V = \varnothing \).
    
    \item \textbf{Т3-аксиома (Регулярность)}: Для любой замкнутой множества \( A \) и точки \( x \notin A \) существуют открытые множества \( U \) и \( V \), такие что \( x \in U \), \( A \subseteq V \) и \( U \cap V = \varnothing \).
    
    \item \textbf{Т4-аксиома (Нормальность)}: Для любых двух непересекающихся замкнутых множеств \( A \) и \( B \) существуют открытые множества \( U \) и \( V \), такие что \( A \subseteq U \), \( B \subseteq V \) и \( U \cap V = \varnothing \).
\end{enumerate}

\begin{statement}[О замкнутости графика]
	Пусть \(f: \ X \to Y\) -- непрерывное отображение, причём \(Y\) -- хаусдорфово. Тогда график этой функции \(\Gamma_f = \{(x, f(x)): \ x \in X\}\) замкнуто в \(X \times Y\).
\end{statement}
\begin{proof}
	Рассмотрим произвольные точки $x \in X$ и $y \in Y$ такие, что $y \neq f(x)$.
	В силу Хаусдорфовости пространства $Y$ найдутся непересекающиеся окрестности $U$ и $V$ точек $y$ и $f(x)$ соответственно. Тогда из непрерывности отображения $f^{-1}(V) \open X$, то есть множество \(\{(x, y): y \neq f(x)\}\) открыто в произведении. А значит дополнение к нему, то есть \(\Gamma_f\) замкнуто.
\end{proof}

\begin{statement}[Критерий Хаусдорфовости]
	\((X, \Omega_x)\) -- Хаусдорфово тогда и только тогда, когда диагональ \(\Delta_x = \{(x, x): x \in X\} \closed X\times X\).
\end{statement}

\begin{proof} \
	\begin{itemize}
		\item[ \(\boxed{\Longrightarrow}\)] Достаточно показать, что \((X \times X) \setminus \Delta\) -- открыто. Возьмём пару \(p_1, p_2\) из \((X \times X) \setminus \Delta\), тогда известно, что \(p_1 \neq p_2\). \(X\) -- хаусдорфово, следовательно существуют окрестности \(U_1, \ U_1\) точек \(p_1, \ p_2\), соответственно, такие что \(U_1 \cap U_2 = \varnothing\). \(U_1 \times U_2\) -- открыто. \(U_1 \times U_2 \subseteq X \times X\) и \(U_1 \times U_2 \cap \Delta_x = \varnothing\), следовательно \((p_1, p_2) \in U_1 \times U_2 \subseteq \left(X \times X\right) \setminus \Delta_X\), отсюда \(\left(X \times X\right) \setminus \Delta_X\) открыто.
		\item[ \(\boxed{\Longleftarrow}\)]  \((X \times X) \setminus \Delta_X\) -- открыто. Пусть \(x_1 \neq x_2 \in X\), тогда \((x_1, x_2) \in (X, X)\setminus \Delta_X\). Существуют открытые множества \((U_1 \times U_2) \in X \times X\), такие что \((x_1, x_2) \in U_1 \times U_2 \subseteq (X \times X) \setminus \Delta_x\). Следовательно, \(x_1 \in U_1\), \(x_2 \in U_2\) и \(U_1 \cap U_2 = \varnothing\). Отсюда следует Хаусдоровость. 
	\end{itemize}
\end{proof}


\subsection{Аксиомы счётности}
Для топологического пространства \( X \) могут выполняются следующие аксиомы счётности:
\begin{enumerate}
    \item \textbf{Первая аксиома счётности}: Для каждой точки \( x \) пространства \( X \) существует база топологии, состоящая из счётного числа открытых множеств, содержащих \( x \).
    
    \item \textbf{Вторая аксиома счётности}: Существует не более чем счётная база $\mathcal{B}$ пространства $X$.
    \end{enumerate}

\begin{example}
	
\begin{enumerate}
    \item \textbf{Дискретное пространство}: Все подмножества множества \(X\) являются открытыми. Базой этого пространства являются одноточечные множества \( \{x\} \), где \(x\) пробегает всё множество \(X\).
    
    \item \textbf{Стандартная топология на множестве натуральных чисел \( \mathbb{N} \)}: Открытыми множествами являются конечные подмножества и все подмножества, дополнения которых конечны. Базой этой топологии являются множества всех конечных последовательностей натуральных чисел.
    
    \item \textbf{Топология Канторова множества}: Множество Кантора получается удалением средней трети каждого отрезка на каждом шаге процесса. Открытыми множествами в этой топологии являются все открытые интервалы и их дополнения. Базой этой топологии являются интервалы вида \((a, b)\), где \(a\) и \(b\) - рациональные числа.
\end{enumerate}

\end{example}

\begin{definition}[Покрытие]
	\textbf{Покрытием} пространства \( X \) называется семейство открытых подмножеств \( \mathcal{U} \), объединение которых содержит все точки пространства \( X \), то есть:

\[
X = \bigcup_{U \in \mathcal{U}} U
\]
\end{definition}

\begin{theorem}[Линделёфа]
	Пусть \( X \) -- топологическое пространство со второй аксиомой счётности, тогда из любого открытого покрытия \( \mathcal{U} \) пространства \( X \) можно выделить счётное подпокрытие.
\end{theorem}
\begin{proof}
	Пусть \(\mathcal{B} = \{V_k\}_{k \in \mathbb{N}}\) -- счётная база из 2 аксиомы счётности и \(\{U_i\}_{i \in \mathbb{I}}\) -- некоторое покрытие \(X\).

	Для каждого \(V_k\) выберем такое множество \(U_{i(k)}\) из покрытия, что \(V_k \subset U_{i(k)}\). Тогда \(X = \bigcup_{k \in \mathbb{N}} V_k \subseteq \bigcup_{k \in \mathbb{N}} U_{i(k)}\). При этом также \(\bigcup_{k \in \mathbb{N}} U_{i(k)} \subseteq X\). А значит на самом деле они равны, то есть \(\bigcup_{k \in \mathbb{N}} U_{i(k)}\) -- необходимое подпокрытие.
\end{proof}



\newpage

\section{Отображения между топологическими пространствами}
\subsection{Определения непрерывного отображения}
Пусть \( X \) и \( Y \) — топологические пространства. 

\begin{definition}[Непрерывное отображение]
	Для отображения \( f : X \rightarrow Y \) между топологическими пространствами \( X \) и \( Y \), \( f \) называется \textbf{непрерывным} в точке \( x \in X \), если для любой окрестности \( V \) точки \( f(x) \) в \( Y \) существует окрестность \( U \) точки \( x \) в \( X \), такая что \( f(U) \subset V \). 

	Если отображение \( f \) непрерывно в каждой точке \( x \in X \), то оно называется \textbf{непрерывным} на \( X \).
\end{definition}

\begin{statement}[Эквивалентные определения]
	Для отображения \( f : X \rightarrow Y \) между топологическими пространствами следующие условия эквивалентны:
	\begin{enumerate}
		\item \(f\) — непрерывное.
		\item Для любого открытого множества \( U \subset Y \), \( f^{-1}[U] \open X \).
		\item Для любого замкнутого множества \( F \subset Y \), \( f^{-1}[F]\closed X \).
		\item Для любого множества \( M \subset X \) выполняется \( f(\mathrm{Cl}(M)) \subseteq \mathrm{Cl}(f(M)) \).
	\end{enumerate}
\end{statement}

\begin{proof} \ 
	\begin{itemize}
		\item[ \(\boxed{1 \longrightarrow 2}\)] Пусть \( x \in \mathrm{Cl}(A) \) и \( U(f(x)) \) — произвольная окрестность точки \( f(x) \in Y \). Согласно условию предпосылки, существует окрестность \( V(x) \) точки \( x \in X \), такая что \( f(V(x)) \subseteq U(f(x)) \).

		Поскольку \( x \in \mathrm{Cl}(A) \), то существует точка \( a \in A \cap V(x) \). Следовательно, \( f(a) \in f(A) \cap f(V(x)) \subseteq f(A) \cap U(f(x)) \). Поскольку выбор окрестности \( U(f(x)) \) был произвольным, для каждой точки \( f(x) \in f(\mathrm{Cl}(A)) \) верно, что \( f(x) \in \mathrm{Cl}(f(A)) \).
		
		\item[ \(\boxed{2 \longrightarrow 1}\)]  Пусть \( x \in X \) и \( U(f(x)) \subseteq Y \) — произвольная окрестность точки \( f(x) \). По условию предпосылки, \( f^{-1}[U(f(x))] \subseteq X \).

		Так как \( x \in f^{-1}[U(f(x))] \), то можно взять окрестность \( V(x) = f^{-1}[U(f(x))] \), что и требовалось.
		
		\item[ \(\boxed{3 \longrightarrow 2}\)] Пусть \( F \closed Y \). По условию предпосылки, \( f^{-1}[F] \closed X \). Рассмотрим множество \( U = Y \setminus F \subseteq Y \), которое открыто.

		Тогда \( f^{-1}[U] = f^{-1}[Y \setminus F] = f^{-1}[Y] \setminus f^{-1}[F] = X \setminus f^{-1}[F] \open X \), то есть \( f^{-1}[U] \open X \), как и требовалось.		
		\item[ \(\boxed{4 \longrightarrow 3}\)] Пусть \( F \subset Y \), но \( f^{-1}[F] \) не замкнуто. Рассмотрим точку \( x \in \mathrm{Cl}(f^{-1}[F]) \setminus f^{-1}[F] \). Тогда \( f(x) \in f(\mathrm{Cl}(f^{-1}[F])) \subseteq f^{-1}[F] \) по условию предпосылки.

		Однако, если \( f(x) \in f(\mathrm{Cl}(f^{-1}[F])) \subseteq \mathrm{Cl}(f(F)) \), то возникает противоречие, так как \( f^{-1}[f(d)] \subseteq f^{-1}[F] \), что противоречит предположению, что \( x \in \mathrm{Cl}(f^{-1}[F]) \).
		
		Следовательно, предположение о том, что \( f^{-1}[F] \) не замкнуто, ошибочно, и \( f^{-1}[F] \) должно быть замкнутым в \( X \).		
	\end{itemize}
\end{proof}

\begin{remark}
	Все функции, которые изучали в курсе стандартного анализа и называли непрерывными (например, полиномиальные, рациональные, показательные, тригонометрические функции и другие), продолжают оставаться непрерывными и в контексте топологических пространств. В этом параграфе мы уточнили определение непрерывности, распространяя его на более общий случай, где под отображениями понимаются функции между топологическими пространствами.
\end{remark}

\subsection{Открытое и замкнутое отображение}

\begin{definition}[Открытое отображение]
	Пусть \( f : X \to Y \) — отображение между топологическими пространствами. Говорят, что \( f \) является \textbf{открытым отображением}, если для любого открытого множества \( U \open X \) его образ \( f(U) \open Y \) также является открытым.
\end{definition}

\begin{definition}[Замкнутое отображение]
	Пусть \( f : X \to Y \) — отображение между топологическими пространствами. Говорят, что \( f \) является \textbf{замкнутым отображением}, если для любого замкнутого множества \( F \closed X \) его образ \( f(F) \closed Y \) также является замкнутым.
\end{definition}

\begin{remark}[о не эквивалентности непрерывного о открытого/замкнутого отображения]
	Если в определении непрерывного отображения заменить оба упоминания слова "открытое" на "замкнутое"\ , то результат будет эквивалентен определению непрерывности. Однако аналогичное преобразование для определения открытого отображения не приводит к эквивалентному утверждению, так как «образ любого замкнутого множества замкнут», соответствует определению замкнутого отображения, которое в общем случае не эквивалентно открытости.

	Существуют открытые отображения, которые не являются замкнутыми и наоборот — замкнутые отображения не являются открытыми. Это различие между открытыми/замкнутыми отображениями и непрерывными отображениями обусловлено тем, что для любого множества \( S \) в общем случае выполняется включение \( f(X \setminus S) \supseteq f(X) \setminus f(S) \), тогда как для прообразов всегда выполняется равенство \( f^{-1}[Y \setminus S] = f^{-1}[Y] \setminus f^{-1}[S] \).
\end{remark}

\begin{example}[открытое и замнкутое отображение] \
	\begin{enumerate} 
		\item Рассмотрим функцию \( f: (\mathbb{R}, \Omega_{\text{std}}) \to (\mathbb{R}, \Omega_{\text{std}})\), заданную как \( f(x) = x^2 \). Эта функция является непрерывной, замкнутой, но не открытой:
		\begin{itemize}
			\item Пусть \( U = (a, b) \) — открытый интервал в \( \mathbb{R} \), который не содержит \( 0 \). Тогда образ этого интервала при отображении \( f \) будет равен \( f(U) = (\min\{a^2, b^2\}, \max\{a^2, b^2\}) \open \mathbb{R}\).
			\item Если же \( U = (a, b) \) — интервал, содержащий \( 0 \), то образ этого интервала будет равен \( f(U) = [0, \max\{a^2, b^2\}) \). Это множество не является открытым в \( \mathbb{R} \).
		\end{itemize}
		\item Рассмотрим функцию \( f: (\mathbb{R}, \Omega_{\text{std}}) \to (\mathbb{Z}, \Omega_{\text{disc}}) \), заданную как \( f(x) = \mathrm{floor}(x) \). Эта функция является открытой и замкнутой, но не является непрерывной. 
\end{enumerate}
\end{example}

\begin{statement}[об открытости проекции]
	Пусть $X$ и $Y$ -- произвольные топологические пространства. Проекция 
	\[
	\mathrm{pr}_1 : X \times Y \to X,
	\]
	определённая как $\mathrm{pr}_1(x, y) = x$, является открытым отображением.	
\end{statement}
\begin{proof}
	Рассмотрим произвольное открытое множество $U \open X \times Y$. По определению топологии произведения, множество $U$ является объединением конечных пересечений множеств вида $V \times W$, где $V \open X$ и $W \open Y$. Следовательно, без ограничения общности, положим, что $U = V \times W$.

Тогда 
\[
\pr{1}{U} = \pr{1}{V \times W} = V.
\]

Поскольку $V$ открыто в $X$, получаем, что образ $\pr{1}{U}$ также является открытым множеством в $X$. 

Так как $U$ было произвольным открытым множеством в $X \times Y$, проекция отображает открытые множества в открытые, то есть проекция является открытым отображением.
\end{proof}
Стоит отметить, что открытые и замкнутые отображения отличаются от непрерывных. В частности, обратное утверждение, что проекция является замкнутым отображением, не выполняется без дополнительных условий, таких как компактность. Рассмотрим следующий контрпример.
\begin{remark}
	Пусть $X = Y = \mathbb{R}$, и рассмотрим проекцию 
\[
\mathrm{pr} : \mathbb{R} \times \mathbb{R} \to \mathbb{R}.
\]
Определим множество $H \subseteq \mathbb{R} \times \mathbb{R}$ как гиперболу
\[
H = \{(x, y) \in \mathbb{R} \times \mathbb{R} \mid x y = 1\}.
\]
Множество $H$ является замкнутым в топологии произведения на $\mathbb{R} \times \mathbb{R}$, поскольку оно задаётся уравнением, которое является непрерывным. Однако его проекция на первую координату
\[
\pr{1}{H} = \{x \in \mathbb{R} \mid x \neq 0\}
\]
не является замкнутым множеством в $\mathbb{R}$, так как дополнение этого множества $\mathbb{R} \setminus \pr{1}{H} = \{0\}$ является точкой, которая не входит в $\pr{1}{H}$, но точка $0$ является предельной для $\pr{1}{H}$. Следовательно, проекция не замкнута в $\mathbb{R}$.

\end{remark}


\subsection{Гомеоморфизм}

\textbf{Гомеоморфизмы} являются одним из центральных понятий топологии, позволяя установить эквивалентность пространств с точки зрения их топологической структуры.

\begin{definition}[Гомеоморфизм]
	Отображение \( f : X \to Y \) называется гомеоморфизмом, если оно биективно, непрерывно, и обратное отображение \( f^{-1} : Y \to X \) также является непрерывным. 

	В таком случае говорят, что пространства \( X \) и \( Y \) \textbf{гомеоморфны}, что обозначается как:
	\[
		X \cong Y.
	\]
\end{definition}

\textbf{Интуитивное объяснение:} Гомеоморфизм можно понимать как "гладкую деформацию" одного пространства в другое без разрывов и склеек. Гомеоморфные пространства обладают одинаковыми топологическими свойствами, такими как связность, компактность или число дыр. 

\begin{example}
	\begin{enumerate}
		\item \textbf{Тождественное отображение:} Любое топологическое пространство \( X \) гомеоморфно самому себе через тождественное отображение \( \mathrm{id}_X \).
		
		\item \textbf{Интервал и прямая:} Интервал \( (0, 1) \) и вся числовая прямая \( \mathbb{R} \) гомеоморфны. Отображение \( f(x) = \tan\left(\pi \left(x - \frac{1}{2}\right)\right) \) задаёт гомеоморфизм между этими пространствами.

		\item \textbf{Квадрат и круг:} Открытый квадрат \( (0, 1) \times (0, 1) \) и открытый круг \( \{(x, y) \in \mathbb{R}^2 : x^2 + y^2 < 1\} \) гомеоморфны. Такой гомеоморфизм может быть построен через радиальное преобразование координат.

		\item \textbf{Сфера и плоскость:} Сфера Римана (сфера \( S^2 \) с удалённым северным полюсом) гомеоморфна плоскости \( \mathbb{R}^2 \). Это отображение обычно реализуется стереографической проекцией.
	\end{enumerate}
\end{example}

\bigskip

\begin{remark}
	Понятие гомеоморфизма лежит в основе топологии, так как позволяет классифицировать пространства с точки зрения их "деформируемости". Например, задача понять, являются ли два объекта "одинаковыми" (гомеоморфными) возникает как в алгебраической топологии, так и в теории динамических систем, где гомеоморфизмы описывают устойчивые состояния систем.
\end{remark}

\begin{statement}[Отношение эквивалентности]
	Гомеоморфность является отношением эквивалентности.
\end{statement}
\begin{proof}
	Для доказательства, что гомеоморфность является отношением эквивалентности, проверим три свойства: рефлексивность, симметричность и транзитивность.
	\begin{itemize}
		\item \textbf{Рефлексивность.} Пусть \( X \) — топологическое пространство. Тождественное отображение \( \mathrm{id}_X : X \to X \), заданное как \( \mathrm{id}_X(x) = x \) для всех \( x \in X \), является биективным, непрерывным, и обратное ему \( \mathrm{id}_X^{-1} = \mathrm{id}_X \) также непрерывно. Следовательно, \( X \cong X \), то есть гомеоморфность рефлексивна.
		\item \textbf{Симметричность.} Пусть \( X \cong Y \), то есть существует биективное отображение \( f : X \to Y \), непрерывное, с непрерывным обратным \( f^{-1} : Y \to X \). Тогда \( f^{-1} \) является гомеоморфизмом \( Y \to X \), что означает \( Y \cong X \). Таким образом, гомеоморфность симметрична.
		\item \textbf{Транзитивность.} Пусть \( X \cong Y \) и \( Y \cong Z \). Тогда существуют гомеоморфизмы \( f : X \to Y \) и \( g : Y \to Z \). Композиция \( g \circ f : X \to Z \) является биективным отображением, так как \( f \) и \( g \) биективны. Более того, так как композиция непрерывных отображений непрерывна, \( g \circ f \) непрерывно. Обратное отображение \( (g \circ f)^{-1} = f^{-1} \circ g^{-1} \) также является непрерывной композицией непрерывных отображений. Следовательно, \( X \cong Z \), то есть гомеоморфность транзитивна.
	\end{itemize}
	Проверенные свойства рефлексивности, симметричности и транзитивности показывают, что гомеоморфность является отношением эквивалентности.
\end{proof}


\begin{remark}[Cовпадения непрерывных отображений в хаусдорфово пространство.]
	Классическая проблема топологии -- определить, являются ли два пространства гомеоморфными. Для доказательства гомеоморфности обычно строят соответствующий гомеоморфизм. Для доказательства негомеоморфности часто используются косвенные методы: находят свойство, которым обладает одно пространство, но не другое, и которое сохраняется при гомеоморфизме. Топологические свойства и инварианты, такие как мощность множества точек и мощность топологической структуры, являются очевидными примерами.
\end{remark}

\begin{statement}[О сужение на всюду плотное множество]
	Предположим, что \( f \) и \( g \) -- две непрерывные функции такие, что \( f : X \rightarrow Y \) и \( g : X \rightarrow Y \). \( Y \) -- пространство Хаусдорфа. Предположим, что \( f(x) = g(x) \) для всех \( x \in A \subseteq X \), где \( A \) является всюду плотным в \( X \), тогда \( f(x) = g(x) \) для всех \( x \in X \).
\end{statement}
\begin{proof}
	Предположим, что \( f(x_0) \neq g(x_0) \). Так как \( Y \) является пространством Хаусдорфа, существуют открытые множества \( U, V \subset Y \) такие, что \( f(x_0) \in U \), \( g(x_0) \in V \), и \( U \cap V = \varnothing \). Теперь
\[ x_0 \in f^{-1}(U) \cap g^{-1}(V) =: W \]
и \( W \) открыто, следовательно, \(\exists a \in A \cap W\), откуда
\[ f(a) = g(a) \in U \cap V \Rightarrow U \cap V \neq \varnothing \]
Это противоречие доказывает результат.
\end{proof}

\begin{statement}[О замкнутости непрерывных отображений в Хаусдорфовом пространстве]
	Пусть \( Y \) -- топологическое пространство, а \( f, g : Y \rightarrow X \) -- непрерывные функции. Докажите, что множество
\[ E = \{ y \in Y : f(y) = g(y) \} \] -- замкнуто.
\end{statement}
\begin{proof}
	Функция \( f - g \) непрерывна, следовательно, \( E = (f - g)^{-1}(\{0\}) \) является обратным образом множества относительно непрерывного отображения. Осталось убедиться, что \(\{0\}\) замкнуто.
 Рассмотрим \( h : Y \rightarrow X \times X \), \( y \mapsto (f(y), g(y)) \). Заметим, что \( E = h^{-1}(\Delta) \). Поскольку \( X \) является Хаусдорфом, то \(\Delta\) замкнута, следовательно, также и \( E \).
\end{proof}

\subsection{Примеры гомеоморфных пространств}
\begin{statement}
	Отрезок \( [0; 1] \) гомеоморфен отрезку \( [a; b] \), где \( a < b \).
\end{statement}
\begin{proof} Проиллюстрируем доказательство.
\begin{figure}[h!]
	\begin{minipage}{0.55\textwidth}
		Рассмотрим отображение \( f: [0; 1] \to [a; b] \), определённое следующим образом:
		\[
		f(x) = a + (b - a) \cdot x \quad \text{для} \quad x \in [0; 1].
		\]
		Это отображение является непрерывным, биективным (для каждого \( y \in [a; b] \) существует единственное \( x \in [0; 1] \), такое что \( f(x) = y \)), и его обратное отображение имеет вид:
		\[
		f^{-1}(y) = \frac{y - a}{b - a} \quad \text{для} \quad y \in [a; b].
		\]
		Следовательно, \( f \) является гомеоморфизмом.
	\end{minipage}
	\hfill
	\begin{minipage}{0.45\textwidth}
		\begin{center}
			\begin{tikzpicture}[scale=1, line join=round,line cap=round,
				dot/.style={circle,fill,inner sep=1pt},>={Stealth[length=1.2ex]}]
				% Оси
				\draw[->] (-0.5, 0) -- (5, 0) node[below] {$x$}; 
				\draw[->] (0, -0.5) -- (0, 6) node[left] {$f(x)$};

				% Отрезок [0, 1] на Ox
				\draw (0, 0) -- (3, 0);
				\fill (0, 0) circle (2pt) node[below left] {$0$};
				\fill (3, 0) circle (2pt) node[below right] {$1$};

				% Отрезок [a, b] на Oy
				\draw (0, 2) -- (0, 5);
				\fill (0, 2) circle (2pt) node[left] {$a$};
				\fill (0, 5) circle (2pt) node[left] {$b$};

				% Прямая f(x) = a + (b-a)x
				\draw (0, 2) -- (3, 5) node[above] {};

				% Проекция x
				\draw[->-] (1.5, 0) -- (1.5, 3.5);
				\draw[->-] (1.5, 3.5)  -- (0, 3.5);

				% Точка на графике
				\fill (1.5, 3.5) circle (2pt);
				\fill (1.5, 0) circle (2pt) node[below] {$x$};
				\fill (0, 3.5) circle (2pt) node[left] {$x^*$};
			\end{tikzpicture}
		\end{center}
	\end{minipage}
\end{figure}
\end{proof}

\begin{definition}[Сфера Римана]
	\textbf{Сфера Римана} в комплексной плоскости \( \mathbb{C} \) — это единичная сфера в трёхмерном пространстве, где выкалывается точка северного полюса \( N = (0, 0, 1) \).
\end{definition}

\begin{statement}
	Множество комплексных чисел \( \mathbb{C} \) гомеоморфно сфере Римана, где сфера Римана задается уравнением \( x^2 + y^2 + z^2 = 1 \) в трехмерном пространстве. 
\end{statement}
\begin{proof}
	Проиллюстрируем доказательство.
	\begin{center}
			\begin{tikzpicture}[declare function={%
			stereox(\x,\y)=2*\x/(1+\x*\x+\y*\y);%
			stereoy(\x,\y)=2*\y/(1+\x*\x+\y*\y);%
			stereoz(\x,\y)=(-1+\x*\x+\y*\y)/(1+\x*\x+\y*\y);
			Px=1.75;Py=-1.5;Qx=-1.5;Qy=-1.25;amax=2.5;},scale=2.5,
			line join=round,line cap=round,
			dot/.style={circle,fill,inner sep=1pt},>={Stealth[length=1.2ex]}]

		\pgfmathsetmacro{\myaz}{15}
		\pgfmathsetmacro{\amax}{2.5}

		\path[save path=\pathSphere] (0,0) circle[radius=1];

		\begin{scope}[3d view={\myaz}{18}]

			\draw[->] (-1.5, 0, 0) -- (1.5, 0, 0);
			\draw[->] (0, -1.5, 0) -- (0, 1.5, 0);
			\draw[->] (0, 0, -1.5) -- (0, 0, 1.5);
			
			\draw (-\amax,\amax) -- (-\amax,-\amax) coordinate (bl) -- (\amax,-\amax)
				coordinate (br) -- (\amax,\amax);

			\begin{scope}
				\tikzset{protect=\pathSphere}
				\draw (-\amax,\amax) -- (\amax,\amax) node[below left,xshift=-2em]{$\mathbb{C}$};
			\end{scope}

			\begin{scope}
				\clip[reuse path=\pathSphere];
				\draw[dashed] (-\amax,\amax) -- (\amax,\amax);
			\end{scope}

			\begin{scope}[canvas is xy plane at z=0]
				\draw[thick, dashed] (\myaz:1) arc[start angle=\myaz,end angle=\myaz+180,radius=1];
				\draw[thick] (\myaz:1) arc[start angle=\myaz,end angle=\myaz-180,radius=1];

				\path[save path=\pathPlane] (\myaz:\amax) -- (\myaz+180:\amax) --(bl) -- (br) -- cycle;

				\begin{scope}
					\clip[use path=\pathPlane];
					\draw[dashed,thick,use path=\pathSphere];
				\end{scope}

				\begin{scope}
					\tikzset{protect=\pathPlane}
					\draw[thick, use path=\pathSphere];
				\end{scope}
			\end{scope}

			\draw[->-=0.3] (Px,Py,0) node[dot,label=below:{$w$}](w){}
			-- ({stereox(Px,Py)},{stereoy(Px,-1)},{stereoz(Px,Py)})
			node[dot,label=above right:{$w^*$}](w*){};

			\draw[->-] (Qx,Qy,0) node[dot,label=below:{$z$}](z){}
			--  ({stereox(Qx,Qy)},{stereoy(Qx,-1)},{stereoz(Qx,Qy)})
			node[dot,label=below right:{$z^*$}](z*){};
			\draw[dashed] (w*) -- (0,0,1) node[dot,label=above left:{$N$}](zeta){} -- (z*) -- (w*);
			\draw[dashed] (w*) -- (0,0,1) node[circle, draw, inner sep=0.7pt,fill=white, label=above left:{$N$}](zeta){} -- (z*) -- (w*);
		\end{scope}
	\end{tikzpicture}
\end{center}
Рассмотрим отображение \( f: \mathbb{C} \to S^2 \), задаваемое формулой:

\[
f(u) = \left( \frac{2\Re(u)}{1 + |u|^2}, \frac{2\Im(u)}{1 + |u|^2}, \frac{|u|^2 - 1}{1 + |u|^2} \right),
\]

где \( u \in \mathbb{C} \), а \( S^2 \) — сфера Римана. Это отображение непрерывно и биективно, поскольку для каждой точки на сфере существует уникальное \( u \in \mathbb{C} \). Обратное отображение:

\[
u = \frac{x + iy}{1 - z},
\]

где \( (x, y, z) \) — координаты точки на сфере, является непрерывным. Следовательно, \( f \) — гомеоморфизм.
\end{proof}
\starsection{Задачи и упражнения}

\begin{task}[Вновь Канторово множество]
    Рассмотрим множество \( \mathcal{C}^2 \), содержащее пары точек из множества \( \mathcal{C} \) в пространстве \( \mathbb{R}^2 \), то есть
    \[
		\mathcal{C}^2 = \{(x, y) \in \mathbb{R}^2 \mid x \in \mathcal{C}, y \in \mathcal{C}\}.
    \]
    Докажите, что отображение \( f: \mathcal{C} \to \mathcal{C}^2 \), заданное по формулам
    \[
    f: \sum_{k=1}^{\infty} \frac{a_k}{3^k} \mapsto  \left(\sum_{k=1}^{\infty} \frac{a_{2k-1}}{3^k}, \sum_{k=1}^{\infty} \frac{a_{2k}}{3^k}\right),
    \]
    является непрерывной сюръекцией.
\end{task}

\begin{task}
    Постройте непрерывную биекцию между интервалом \( [0; 1) \) и окружностью \( S^1 \), которая не будет являться гомеоморфизмом.
\end{task}


\begin{task}
    Пусть \( f : X \to Y \) — гомеоморфизм. Тогда для любого множества \( A \subset X \) выполняются следующие утверждения:
    \begin{enumerate}
        \item Множество \( A \) замкнуто в \( X \), тогда и только тогда, когда \( f(A) \) замкнуто в \( Y \);
        \item \( f(\text{Cl} A) = \text{Cl} f(A) \);
        \item \( f(\text{Int} A) = \text{Int} f(A) \);
        \item \( f(\text{Fr} A) = \text{Fr} f(A) \);
        \item \( A \) — окрестность точки \( x \in X \), тогда и только тогда, когда  \( f(A) \) — окрестность точки \( f(x) \).
    \end{enumerate}
\end{task}


\begin{task}[Комплексный гомеоморфизм]
	Докажите, что конформное отображение, заданное дробно-линейным преобразованием \( f : \hat{\mathbb{C}} \to \hat{\mathbb{C}} \), является гомеоморфизмом на расширенной комплексной плоскости \( \hat{\mathbb{C}} = \mathbb{C} \cup \{\infty\} \). 

	Преобразование задается соотношением:
	\[
	f: z \mapsto \frac{az + b}{cz + d}, \quad ad - bc \neq 0,
	\]
	где \( a, b, c, d \) — комплексные числа.	
\end{task}
\begin{task}[Вещественный гомеоморфизм]
    Докажите, что любое биективное монотонное отображение \( f : \mathbb{R} \to \mathbb{R} \) является гомеоморфизмом. 
\end{task}

\begin{task}[Классические гомеооморфизмы]
		Докажите, что выполняются следующие гомеоморфизмы и проиллюстрируйте их:
		\begin{enumerate}
			\item \( [0; 1) \cong [a; b) \cong (0; 1] \cong (a; b] \) для любых \( a < b \);
			\item \( (0; 1) \cong (a; b) \) для любых \( a < b \);
			\item \( (-1; 1) \cong \mathbb{R} \);
			\item \( [0; 1) \cong [0; +\infty) \) и \( (0; 1) \cong (0; +\infty) \);
			\item \( \mathbb{C}^n \cong \mathbb{R}^{2n} \);
			\item \( S^n \setminus N \cong \mathbb{R}^n \).
		\end{enumerate}	
\end{task}

\begin{task}[Гомеоморфизм Канторово множества]
    Покажите, что существует гомеоморфизм между множеством Кантора \( \mathcal{C} \) и счётным произведением двуточечного множества \( \{0, 1\}^{\aleph_0} \) с дискретной топологией, то есть:
    \[
    \{0, 1\}^{\aleph_0} \cong \mathcal{C}
    \]
\end{task}


\newpage
\section{Компактное пространство}
\subsection{Топологическое определение}

В математическом анализе часто возникают множества с особыми свойствами, которые обеспечивают нам удобные условия для работы. Например, в контексте анализа мы сталкиваемся с леммой Гейне-Бореля, которая утверждает, что замкнутые и ограниченные множества в евклидовой топологии обладают удивительными свойствами. На курсах дифференциальных уравнений мы сталкиваемся с теоремой Асколи — Арцела, которая для пространств функций утверждает подобное свойство.

Эти теоремы подсказывают нам важность множества, которое можно "обрабатывать" в каком-то смысле — извлекать сходящиеся подпоследовательности или гарантировать существование решения, ограничивая его поиски конечным числом шагов. Эти свойства лежат в основе ключевой идеи в топологии, связанной с компактностью.

Компактность множества помогает гарантировать, что оно будет вести себя "хорошо" в различных математических контекстах, и является важным понятием в топологии. Перейдем к формальному определению компактного множества.

\begin{definition}[Компакт]
	\textbf{Компактное топологическое пространство} \( C \) — это пространство, в котором из любого открытого покрытия можно выбрать конечное подпокрытие. То есть, для любого покрытия \( \{ U_i \}_{i \in \mathbb{I}} \) пространства \( X \) существует конечное подмножество индексов \( \mathcal{J} \subseteq \mathbb{I} \), такое что:
	\begin{equation*}
		C = \bigcup_{i \in \mathcal{J}} U_i
	\end{equation*}
	где \( C \Subset X \) — компактное подмножество.
\end{definition}


\begin{example}
	Рассмотрим следующие примеры компактных пространств:
	\begin{enumerate}
		\item \textbf{Замкнутый отрезок на прямой:} Отрезок \( [0,1]^n \) с обычной топологией компактен. Это следует из леммы Гейне-Бореля, которая утверждает, что в пространстве размерности \( n \) замкнутое и ограниченное множество компактно.
		\item \textbf{Тор:} Прямоугольник, образующий тор \( S^1 \times S^1 \), является компактным пространством.
		
		\item \textbf{Замкнутый шар в \( \mathbb{R}^n \):} Замкнутый шар \( \{x \in \mathbb{R}^n : \|x\| \leq r\} \) в евклидовом пространстве \( \mathbb{R}^n \) также компактен.
	\end{enumerate}
\end{example}


\begin{statement}[О замкнутых подмножествах компакта]
	Если множество $S$ содержится в компактном множестве $T$ и $S$ замкнуто, то $S$ также компактно.
\end{statement}
\begin{proof}
	Предположим, что $\mathcal{U}$ -- открытое покрытие множества $S$. Каждое открытое множество в $\mathcal{U}$ имеет вид $U \cap S$ для некоторого открытого множества $U$ (открытого в $T$). Пусть $\mathcal{V} = \{ U \subseteq T \mid U$ открыто, и $\exists U' \in \mathcal{U} : U \cap S = U' \}$. Тогда $\mathcal{V}$ также является открытым покрытием множества $S$, так как $S$ замкнуто, мы имеем, что $T \setminus S$ открыто, поэтому $\mathcal{V} \cup \{ T \setminus S \}$ -- открытое покрытие множества $T$.

Из компактности $T$ следует, что у нас есть конечное подпокрытие, из которого мы можем получить конечное подпокрытие для $\mathcal{U}$.
\end{proof}


\begin{statement}[О произведение компактов]
	Пусть \(X, \ Y\) -- компактные пространство, тогда \(X \times Y\) -- компактное пространство.
\end{statement}
\begin{proof}
	Пусть $M$ -- открытое множество в $X \times Y$, такое что $\{x_0\} \times Y \subseteq M$. Тогда существует открытое множество $W \subseteq X$, содержащее $x_0$, такое что $W \times Y \subseteq M$.

Используя данное утверждение, необходимо показать, что для любого $x \in X$ можно создать трубку (\textit{tubular}) (покрытую конечным количеством открытых множеств) вокруг $\{x\} \times Y$.

Для начала рассмотрим открытое покрытие для $X \times Y$, обозначим его как $\{U_{\alpha}\}_{\alpha \in J} = \{A_{\alpha} \times B_{\alpha}\}$, где каждое $A_{\alpha}$ открыто в $X$, а каждое $B_{\alpha}$ открыто в $Y$. Что дальше?

Мы знаем, что для любого открытого покрытия для $X$, обозначим его как $\{A_{\alpha}\}_{\alpha \in J}$, существует конечное подпокрытие для $X$. Аналогично для $Y$.
\end{proof}

\begin{statement}[О замкнутости компакта в Хаусдорфовом пространстве]
	Компактное множество \( K \) в Хаусдорфовом пространстве \( X \) замкнуто.
\end{statement}
\begin{proof}
	Зафиксируем \( x \in X \setminus K \). Так как \( X \) является пространством Хаусдорфа, для каждого \( y \in K \) существуют различные открытые множества \( U_y \) и \( V_y \), такие что \( x \in U_y \) и \( y \in V_y \). Множество \( \{ V_y : y \in K \} \) является открытым покрытием \( K \), поэтому оно имеет конечное подпокрытие, скажем, \( \{ V_y : y \in F \} \), где \( F \) -- некоторое конечное подмножество множества \( K \). Положим:
\[ U = \bigcap_{y \in F} U_y; \]
очевидно, что \( U \) является открытым окрестностью \( x \), не пересекающейся с \( K \). Так как \( x \) была произвольной точкой из \( X \setminus K \), то \( K \) должно быть замкнутым.
\end{proof}

\subsection{Определение в метрическом пространстве}

В дальнейшем мы будем рассматривать многообразия, которые могут быть метризуемыми. Это означает, что на таких многообразиях можно ввести метрику, позволяющую работать с ними как с метрическими пространствами. В связи с этим имеет смысл сформулировать эквивалентное определение компактного множества в контексте метрических пространств. Однако прежде чем это сделать, напомним определение самого метрического пространства.

\begin{definition}[Метрическое пространство]
    Пусть \( M \) — множество. Метрикой на \( M \) называется функция \( d: M \times M \to \mathbb{R}_{\geq 0} \), удовлетворяющая следующим аксиомам:
    \begin{enumerate}
        \item \textbf{Невырожденность:} \( d(x, y) = 0 \) тогда и только тогда, когда \( x = y \);
        \item \textbf{Симметричность:} \( d(x, y) = d(y, x) \);
        \item \textbf{Субаддитивность:} \( d(x, y) \leq d(x, z) + d(z, y) \),
        \\ для любых точек \( x, y, z \in M \).
    \end{enumerate}
\end{definition}

Перед введением понятия открытого шара отметим, что в метрических пространствах для изучения окрестностей точек необходимо выделить базовое множество, позволяющее задавать такие окрестности. Этим множеством является открытый шар.

\begin{definition}[Открытый шар в метрическом пространстве]
	Пусть \( x \in M \) — точка в метрическом пространстве. \textbf{Открытый} \( \varepsilon \)-\textbf{шар} \( B_\varepsilon(x) \) с центром в \( x \) — это множество всех точек, отстоящих от \( x \) на расстояние меньше, чем \( \varepsilon \):
	\[
	B_\varepsilon(x) = \{ y \in M \mid d(x, y) < \varepsilon \}.
	\]
\end{definition}

Понятие ограниченного множества уже встречалось в курсе математического анализа, где оно играло важную роль при работе с числовыми множествами. В контексте метрических пространств это определение также имеет смысл, так как позволяет говорить об удалённости точек множества от заданной точки. Прежде чем продолжить, напомним формальное определение ограниченного множества в метрическом пространстве.

\begin{definition}[Ограниченность]
	Метрическое пространство \( (M,d) \) называется ограниченным, если 
\[ \exists R > 0 : \forall x \in M, \exists a \in M : d(x,a) \leq R \]
\end{definition}

\begin{statement}[Об ограниченности компакта в метрическом пространстве]
	Компактное подмножество метрического пространства ограниченно.
\end{statement}
\begin{proof}
	Пусть \( a \in M \). Пусть \( n \in \mathbb{N} > 0 \). Пусть \( B_n(a) \) -- открытый \( n \)-шар с центром в точке \( a \). Тогда \( C \subseteq \bigcup_{n=1}^{\infty} B_n(a) \), потому что для любой точки \( x \in C \) существует такое \( n \in \mathbb{N} \), что \( d(x, a) < n \).

Таким образом, множество \( \{ B_n(a) : n \in \mathbb{N} \} \) образует открытое покрытие \( C \). Поскольку \( C \) компактно, у него есть конечное подпокрытие, скажем: \( \{ B_{n_1}(a), B_{n_2}(a), \ldots, B_{n_r}(a) \} \). Пусть \( n = \max\{ n_1, n_2, \ldots, n_r \} \). Тогда:
\[ C \subseteq \bigcup_{n=1}^{r} B_{n_r}(a) = B_n(a) \]
Результат следует из определения ограниченности.
\end{proof}

\begin{theorem}[Критерий компактности в метрическом пространстве]
	Пространство \( X \) в \( \mathbb{R}^n \) является компактным тогда и только тогда, когда \( X \) замкнуто и ограничено.
\end{theorem}
\begin{proof}
	Воспользоваться доказанными утверждениями выше.
\end{proof}

\subsection{Связь компактности и непрерывных отображений}

\begin{definition}[Собственное отображение]
	Пусть \( f : X \to Y \) — отображение между топологическими пространствами. Говорят, что \( f \) является \textbf{собственным отображением}, если для любого компактного множества \( C \Subset X \) его образ \( f(C) \Subset Y \).
\end{definition}


\begin{statement}[Непрерывный образ компакта]
	Пусть \( K \Subset X \) -- компактное множество, и \( f : X \rightarrow Y \) -- непрерывная функция. Тогда множество \( f(K) \) является компактным.
\end{statement}
\begin{proof}
	Пусть \( f \) непрерывно. Возьмем любое открытое покрытие \( f(K) \). Так как \( f \) непрерывно, обратные образы этих открытых множеств образуют открытое покрытие \( K \). Так как \( K \) компактно, существует конечное подпокрытие. По построению образы этого конечного подпокрытия дают конечное подпокрытие \( f(K) \), поэтому \( f(K) \) также компактно.
\end{proof}

\begin{corollary}[Теорема Вейерштрасса]
	Если множество $A$ компактно, то непрерывная функция \( f : A \rightarrow \mathbb{R} \) принимает максимальное и минимальное значение.
\end{corollary}
\begin{proof}
	Множество \( f(A) \) является непрерывным образом компактного множества, поэтому оно является компактным подмножеством \( \mathbb{R} \). Следовательно, оно ограничено и имеет как верхнюю, так и нижнюю грани. Докажем, что верхняя грань достигается в \( A \). Обозначим верхнюю грань \( s \). По определению верхней грани вещественных чисел существует последовательность \( (y_n) \) точек из \( f(A) \), сходящаяся к \( s \). Но тогда, так как множество \( f(A) \) замкнуто, мы имеем \( s \in f(A) \).
\end{proof}

\begin{statement}
	Пусть \( f : X \rightarrow Y \) -- непрерывное биективное отображение. Если \( X\) компактно, а \( Y \) -- хаусдорфово, то \( f \) является гомеоморфизмом.
\end{statement}
\begin{proof}
	Пусть \( g = f^{-1} \). Нам нужно показать, что \( g : Y \rightarrow X \) непрерывно. Для любого \( V \subseteq X \) у нас есть \( g^{-1}(V) = f(V) \). Нам нужно показать, что если \( V \) замкнуто в \( X \), то \( g^{-1}(V) \) замкнуто в \( Y \). 
	
	Предположим, что \( V \) замкнуто в \( X \). Поскольку \( X \) компактно, \( V \) компактно. Таким образом, \( f(V) \) компактно и \( Y \) хаусдорфово, \( f(V) \) замкнуто. Но \( f[V] = g^{-1}(V) \), поэтому \( g^{-1}(V) \) замкнуто. Из того следует, что \( g \) непрерывно. Таким образом, по определению, \( f \) является гомеоморфизмом.
\end{proof}

\begin{remark}
	Непрерывное отображение из компакта \( X \) в хаусдорфово топологическое пространство \( Y \) является собственным.
\end{remark}

\begin{corollary}
	Непрерывное отображение из компакта \( X \) в хаусдорфово топологическое пространство \( Y \) является замкнутым.
\end{corollary}

\begin{definition}
	Пусть \( f : X \to Y \) — непрерывное отображение. Напомним, что для любой точки \( y \in Y \) прообраз \( f^{-1}(y) \) называется слоем \( f \).
\end{definition}

\begin{theorem}
	Пусть \( f : X \to Y \) — замкнутое, непрерывное отображение, при этом все слои \( f \) компактны. Тогда \( f \) является собственным.
\end{theorem}

\begin{proof}
	Пусть \( K \Subset Y \). Необходимо показать, что \( f^{-1}[K] \) компактно. Заменив \( Y \) на \( K \), а \( X \) на \( f^{-1}[K] \), можно считать, что \( Y \) компактно.

	\begin{remark}
		Компактность \( M \) эквивалентна следующему свойству: пусть \( \{A_{\alpha}\} \) — набор замкнутых подмножеств \( M \), такой, что любое конечное подмножество \( \{A_1, A_2, \dots, A_n\} \subset \{A_{\alpha}\} \) имеет общую точку. Тогда все \( A_{\alpha} \) имеют общую точку. 
	\end{remark}

	Действительно, отсутствие общей точки у \( \{A_{\alpha}\} \) означает, что \( \{M \setminus A_{\alpha}\} \) — покрытие, а наличие общих точек у \( \{A_1, A_2, \dots, A_n\} \) означает, что в \( \{M \setminus A_{\alpha}\} \) нет конечного подпокрытия.

	Добавив к \( \{A_{\alpha}\} \) все конечные пересечения элементов \( \{A_{\alpha}\} \), получим набор замкнутых подмножеств \( X \), обладающий тем же свойством. Поэтому можно считать, что \( \{A_{\alpha}\} \) содержит все конечные пересечения своих элементов.

	Пусть \( \{A_{\alpha}\} \) — набор замкнутых подмножеств в \( X \), такой что любое конечное подмножество \( \{A_1, A_2, \dots, A_n\} \subset \{A_{\alpha}\} \) имеет общую точку. Поскольку \( Y \) компактно, а все \( f(A_{\alpha}) \) замкнуты, то \( \{f(A_{\alpha})\} \) имеет общую точку \( y \in Y \).

	Пусть \( y = \alpha f(A_{\alpha}) \). Любое конечное пересечение \( \bigcap_i A_i \) лежит в \( \{A_{\alpha}\} \), значит, пересекается с \( f^{-1}[y] \). В силу компактности \( f^{-1}[y] \), из этого следует, что \( \{A_{\alpha} \cap f^{-1}[y]\} \) имеет общую точку.

	Таким образом, \( X \) компактен.
\end{proof}

Наконец, можно закрыть гештальт относительно проекций. Мы уже убедились, что проекция является открытым отображением. Теперь, для проекций, возникает интересная возможность рассмотреть еще одно важное свойство — замкнутость отображений. Однако для того чтобы это свойство было верно, нам нужно наложить некоторое ограничение на домен функции. В частности, будем рассматривать только проекции, определенные на компактных множествах, и тогда утверждение о замкнутости проекций будет истинным.

\begin{statement}[о замкнутости проекции компактного множества]
    Пусть \(\pr{X \times Y}{Y}\), причём \(X\) компактно. Тогда \(\pr{X \times Y}{Y}\) замкнуто.
\end{statement}

\begin{proof}
    Пусть \(Z \subset X \times Y\) — замкнутое подмножество. Если \(\pr{X \times Y}{Y}(Z)\) не замкнуто, то для какой-то предельной точки \(y \in \mathrm{pr}(Z)\) имеем \(\mathrm{pr}[y] \cap Z = \emptyset\).

    У каждой точки \((x, y) \in \mathrm{pr}^{-1}[y]\) есть окрестность \(U_{x, y} = V_{x, y} \times W_{x, y}\), не пересекающаяся с \(Z\).

    Поскольку \(\mathrm{pr}^{-1}[y]\) компактно, можем выбрать конечное покрытие \(\{V_i \times W_i\}\) множества \(\mathrm{pr}^{-1}[y]\), не пересекающееся с \(Z\).

    Множество \(\displaystyle X \times (\bigcup_i W_i)\) не пересекает \(Z\), значит, \(y\) не предельная точка \(\mathrm{pr}(Z)\).
\end{proof}

\subsection{Последовательности и секвенциальная компактность}

\begin{definition}[Последовательность в топологическом пространстве]
	Пусть \( X \) — топологическое пространство. Последовательность в \( X \) — это отображение \( x: \mathbb{N} \to X \), которое ставит в соответствие каждому натуральному числу \( n \) элемент \( x_n \in X \). Т.е. последовательность в \( X \) представляет собой последовательность элементов \( \{x_n\} \), где для каждого \( n \in \mathbb{N} \) \( x_n \in X \).
\end{definition}
Как и всегда после введения последовательности, необходимо ввести определение её предельного значения. 
\begin{definition}[Предел последовательности в топологическом пространстве]
	\( \{a_n\} \) — последовательность элементов пространства \( X \). Точка \( L \in X \) называется пределом последовательности \( \{a_n\} \), если:
	\[
		\forall U(L) \ \exists N \in \mathbb{N}\st \forall n > N \ a_n \in U(L),
	\]
	В этом случае говорят, что последовательность \( \{a_n\} \) сходится к \( L \), и пишут \(\displaystyle \lim_{n \to \infty} a_n = L \).
\end{definition}

Подход к изучению через последовательности широко применяется в математическом анализе и других областях математики. Этот метод позволяет формулировать и доказывать результаты, используя понятие предела последовательности, что значительно упрощает понимание и решение задач. Часто для известных понятий, даются формулировки именно через последовательности, которые в общем случаи не эквивалентны топологическим. 
Слово "последовательность" на латинском языке будет \textit{sequance}, и по этой причине подход, в котором используется анализ последовательностей, часто называют \textit{секвенциальным}.

\begin{definition}[Секвенциальная компактность]
	Топологическое пространство \(X\) называется секвенциально компактным, если из любой последовательности точек в \(X\) можно выделить сходящуюся подпоследовательность.
\end{definition}


\begin{statement}
	Из компактности следует секвенциальная компактность в пространствах со счётной базой. То есть, если \( M \) — компактное множество в \( X \), то всякая последовательность \( \{x_i\} \) в \( M \) имеет сходящуюся подпоследовательность.
\end{statement}

\begin{proof}
	Пусть \( \{x_i\} \) — последовательность точек множества \( M \subset X \). Рассмотрим множество \( \mathcal{R}_n = \{ x_n, x_{n+1}, x_{n+2}, \dots \} \) и его замыкание \( \cl{\mathcal{R}_n} \). Тогда пересечение всех таких множеств
	\[
		\bigcap_{i} \cl{\mathcal{R}_n}
	\]
	непусто. Ясно, что это пересечение состоит из предельных точек последовательности \( \{ x_i \} \).
\end{proof}

\begin{remark}
    Любая непрерывная функция \( f : K \to \mathbb{R} \) на секвенциально компактном множестве \( K \) принимает максимум и минимум. Это утверждение известно как теорема Вейерштрасса.
\end{remark}

\begin{remark}
    Для метрических пространств секвенциальная компактность эквивалентна компактности в обычном смысле. Это утверждение является следствием теоремы Гейне-Бореля-Лебега.
\end{remark}


\begin{remark}

	В общем случае из компактности не следует секвенциальная компактность, и наоборот. Приведённые примеры не являются частью основного курса, но служат для иллюстрации того, что в общем смысле из одного вида компактности не следует другой.	\begin{enumerate}
		\item Примером секвенциально компактного пространства, которое не является компактным, служит первое несчётное ординальное число с топологией порядка.
		\item Примером компактного пространства, которое не является секвенциально компактным, является топологический произведение \([0,1]^{\mathfrak{c}}\).
	\end{enumerate}
\end{remark}

\subsection{Компактификация Александрова}

Ранее мы рассматривали гомеоморфизм между сферой без одной точки и комплексной плоскостью. Однако, при удалении точки из сферы теряется её компактность.

Добавив точку бесконечности, мы превращаем комплексную плоскость в компактное пространство, где все последовательности, стремящиеся к бесконечности, теперь сходятся к этой новой точке. Интуитивно это можно представить как процесс сворачивания всей комплексной плоскости в сферу.

Таким образом, добавление одной точки превращает комплексную плоскость в компактное пространство, позволяя работать с бесконечностью в компактных рамках.
\begin{definition}
	Пусть \( X \) — топологическое пространство. Одноточечная компактификация пространства \( X \) — это расширение \( \hat{X} = X \cup \{\infty\} \), где добавляется одна новая точка \( \infty \), называемая точкой бесконечности. Топология \( \hat{\Omega} \) на \( \hat{X} \) строится следующим образом:
	
	\begin{itemize}
		\item открытые множества \( X \open \hat{X} \),
		\item множества вида \( \hat{X} \setminus K \), где \( K \Subset X \), являются открытыми в \( \hat{X} \).
	\end{itemize}
\end{definition}
\begin{statement}
	\((\hat{X}, \hat{\Omega})\) -- компактное топологическое пространство.
\end{statement}

\begin{proof}
	Рассмотрим топологическое пространство \( (\hat{X}, \hat{\Omega}) \), где \( \hat{X} = X \cup \{\infty\} \), и \( \hat{\Omega} \) — топология на \( \hat{X} \).

\( \displaystyle  \varnothing \) и \( \hat{X} \) лежат в \( \hat{\Omega} \) по определению топологии.

Рассмотрим множества вида \( \displaystyle \hat{X} \setminus K_i \), где \( K_i \) — компактные подмножества \( X \). Эти множества открыты в \( \hat{\Omega} \).

Объединение \(\displaystyle  \bigcup_{i \in \mathbb{I}} (\hat{X} \setminus K_i) = \hat{X} \setminus \bigcup_{i \in \mathbb{I}} K_i \) также открыто в \( \hat{\Omega} \), так как топология замкнута относительно объединений.

Для конечных объединений \(\displaystyle  \bigcup_{j = 1}^N (\hat{X} \setminus K_j) = \hat{X} \setminus \left( \bigcup_{j = 1}^N K_j \right) \) также справедливо, так как конечное объединение компактных подмножеств остаётся компактным.

Таким образом, пространство \( \hat{X} \) компактно, так как все необходимые объединения открыты в \( \hat{\Omega} \).

\end{proof}
	
\begin{example}
	Компактификация вещественной прямой \( \mathbb{R} \) — это окружность \( S^1 \). 
	
	Добавим к \( \mathbb{R} \) точку бесконечности \( \infty \). Её окрестностью будем считать множество вида \( (-\infty, -a) \cup (a, \infty) \), где \( a > 0 \), то есть открытое множество, состоящее из двух бесконечных интервалов. Дополнением к такой окрестности является множество \( \mathbb{R} \setminus U(\infty) = [-a, a] \), которое представляет собой отрезок, а каждый отрезок, как известно, является компактным.
	
	Аналогично, компактификация \( \mathbb{R}^n \) — это \( n \)-мерная сфера \( S^n \), получаемая путём добавления одной точки бесконечности \( \infty \). Для любой окрестности \( \infty \) её дополнение является замкнутым и компактным множеством в \( \mathbb{R}^n \).
	
	\end{example}
	
\begin{example}[двуточненая компактификация]
Рассмотрим пространство \( X = (a, b) \), где \( a, b \in \mathbb{R} \) и \( a < b \). Компактификация \( \hat{X} \) получается добавлением концов \( a \) и \( b \), превращая \( X \) в замкнутый отрезок \( \hat{X} = [a, b] \). В новой топологии окрестностями точек \( a \) и \( b \) являются полуинтервалы вида \( [a, a+\varepsilon) \) и \( (b-\varepsilon, b] \), соответственно, что делает \( \hat{X} \) компактным пространством.
\end{example}
	
	
\starsection{Задачи и упражнения}

\begin{task}
	Докажите, что любое конечное дискретное и любое антидискретное пространства компактны.
\end{task}

\begin{task}
	Приведите пример замкнутого и ограниченного подмножества метрического пространства, которое не будет являться компактным.
\end{task}

\begin{task}
	Показать, что куб \(I^n\) компактен, где куб задаётся следующим множеством: 
	\[
		I^n = \{x \in \mathbb{R}^n: \ x_i \in \left[0, 1\right] \ \text{для} \ i=1,\dots,n\}.
	\]
\end{task}

\begin{task}
    Пусть \( X \) — секвенциально компактное пространство, а \( \{A_n\} \) — последовательность непустых замкнутых множеств, упорядоченная по включению, то есть \( A_1 \supset A_2 \supset A_3 \supset \dots \). Докажите, что пересечение всех этих множеств непусто, то есть
    \[
    \bigcap_{n=1}^{\infty} A_n \neq \emptyset.
    \]
\end{task}

\begin{task}[Вновь Канторово множество]
    Докажите, что множество Кантора является компактным.
\end{task}

\begin{task}
	Покажите, что \(f: X \to Y\) -- собственное и непрерывное отображение, тогда и только тогда, когда \(\hat{f}\) -- непрерывное отображение между \(\hat{X}\) и \(\hat{Y}\), где:
	\begin{equation*}
		\hat{f}: x \mapsto \begin{cases}
			f(x) & x \in X, \\
			\hat{Y} \setminus Y & x = \hat{X} \setminus X.
		\end{cases}
	\end{equation*}
\end{task}

\begin{task}
	Явно опишите одноточечные компактификации следующих топологических пространств:
	\begin{enumerate}
		\item Кольцо \( \{(x, y) \in \mathbb{R}^2 \ | \ 1 < x^2 + y^2 < 2\} \).
		\item Квадрат без вершин \( \{(x, y) \in \mathbb{R}^2 \ | \ x, y \in [-1, 1], \ |xy| < 1\} \).
		\item Полоса \( \{(x, y) \in \mathbb{R}^2 \ | \ x \in [0, 1]\} \).
		\item Компактное пространство.
	\end{enumerate}
\end{task}	


\newpage
\section{Факторпространства}

\subsection{Теоретико-множественное отступление}

Факторизация множеств и отображений упрощает сложные структуры, сводя их к более удобным формам. Она широко применяется в топологии, алгебре и других областях для изучения симметрий и инвариантов.

Интуитивно, факторизация группирует элементы множества так, чтобы сосредоточиться на глобальных свойствах, игнорируя различия внутри классов эквивалентности. Например, окружность можно рассматривать как фактор-пространство плоскости, где точки с одинаковым расстоянием до начала координат объединены в один класс.




\begin{definition}[Фактор-множество]
    Пусть \( X \) — множество, а \( \sim \) — отношение эквивалентности на \( X \). Тогда \textbf{фактор-множество} \( X \) по отношению эквивалентности \( \sim \) — это множество классов эквивалентности:
    \[
    X / \mathord{\sim} = \{ [x] \mid x \in X \},
    \]
    где \( [x] = \{ y \in X \mid y \sim x \} \) — класс эквивалентности элемента \( x \). 

    На множестве \( X / \sim \) вводится естественная проекция \( \text{pr}: X \to X / \sim \), сопоставляющая каждому \( x \in X \) его класс эквивалентности \( [x] \).
\end{definition}


\begin{example}
	Рассмотрим множество целых чисел \( \mathbb{Z} \). Определим отношение эквивалентности \( \sim \), задав правило: два числа \( m, n \in \mathbb{Z} \) эквивалентны (\( m \sim n \)), если разность \( m - n \) делится на 2, то есть \( m - n \in 2\mathbb{Z} \).
	
	\textbf{Классы эквивалентности.}  
	Отношение \( \sim \) разбивает множество \( \mathbb{Z} \) на два класса эквивалентности:  
	\begin{itemize}
		\item \( [0] = \{ n \in \mathbb{Z} \mid n \text{ чётное} \} = \{ ..., -4, -2, 0, 2, 4, ... \} \) -- класс чётных чисел.
		\item \( [1] = \{ n \in \mathbb{Z} \mid n \text{ нечётное} \} = \{ ..., -3, -1, 1, 3, 5, ... \} \) -- класс нечётных чисел.
	\end{itemize}
	
	Таким образом, множество классов эквивалентности можно записать как:  
	\[
	\mathbb{Z}/{\sim }= \{ [0], [1] \}.
	\]
	
	Фактор-множество \( \mathbb{Z}/{\sim }\) содержит ровно два элемента: класс чётных чисел и класс нечётных чисел. Интуитивно это соответствует разбиению множества целых чисел на чётные и нечётные.
	\end{example}
	
\begin{remark}[об обозначачение]
Символ \( / \) традиционно используется для обозначения факторизации, так как этот процесс напоминает деление. В контексте фактор-множеств \( X / \sim \) выражает разбиение \( X \) на подмножества, подобно тому, как деление числа разделяет его на равные части. 
\end{remark}


\subsection{Фактортопология}


Для удобства работы с топологическими пространствами вводим понятие фактор-пространства. Оно позволяет свести исследование сложных структур к более простым, рассматривая пространство как "свертку" с учетом некоторого разбиения на классы эквивалентности. Это полезно, например, при изучении симметрий и инвариантов.

\begin{definition}[Фактор-пространство]
\textbf{Фактор-пространство} \( X/{\sim} \) топологического пространства \( X \) по отношению эквивалентности \( \sim \) наделяется топологией, в которой множество \( U \subset X/{\sim} \) считается открытым тогда и только тогда, когда его прообраз \( \text{pr}^{-1}(U) \) открыт в \( X \).
\end{definition}

\begin{statement}
	Топология \( \Omega_{X/{\sim}} \), индуцированная на множестве \( X/{\sim} \) отображением \( \mathrm{pr} \), является сильнейшей, в которой отображение \( \mathrm{pr} \) является непрерывным.
\end{statement}
	
\begin{proof}
Предположим, что существует более сильная топология, то есть к топологии \( \Omega_{X/{\sim}} \) добавлены дополнительные множества. Однако прообразы этих добавленных множеств уже не будут открытыми в топологии \( \Omega_X \), поскольку в топологии \( \Omega_{X/{\sim}} \) выбраны все такие множества, прообразы которых открыты в \( X \). Следовательно, прообразы добавленных множеств не будут открытыми, что приведет к нарушению непрерывности отображения \( \mathrm{pr} \).
\end{proof}
	
	

\subsection{Построение фигур}

В этом параграфе мы рассмотрим, как различные геометрические объекты могут быть получены с помощью факторизации. В частности, мы покажем, как при помощи эквивалентности можно склеивать стороны прямоугольника, получая такие фигуры, как цилиндр, тор, бутылка Клейна, лента Мёбиуса и конус.

Исходным объектом для всех конструкций будет прямоугольник в \( \mathbb{R}^2 \) со стандартной топологией \( \Omega_{\mathrm{std}} \). Мы будем вводить на нём отношение эквивалентности, которое позволит "идентифицировать" определённые его точки. Это приведёт к возникновению новых топологических пространств, обладающих особыми свойствами. 

Наша цель — исследовать, какие поверхности можно получить таким способом, и как их топологическая структура зависит от способа склейки границ.

\begin{definition}[Стягивание]
    Факторизация пространства \( X \) по отношению эквивалентности, отождествляющему все точки некоторого подмножества \( A \) в одну точку, называется \textbf{стягиванием} множества \( A \). Полученное фактор-пространство обозначается \( X/A \).
\end{definition}

При работе с фактор-пространствами зачастую удобнее давать определение через коммутативные диаграммы. Такой подход позволяет наглядно отразить взаимосвязь между исходным пространством, подмножеством, которое стягивается, и полученным фактор-пространством. Коммутативная диаграмма ниже демонстрирует процесс стягивания подмножества $ A \subset X $ в одну точку (обозначаемую как $\mathrm{point}$) и построение фактор-пространства $ X/A $.

\begin{center}
    \[
    \begin{tikzcd}[row sep=60pt, column sep=60pt, scale=1]
        A \arrow[r, hook] \arrow{d}{\sim} & X \arrow{d}{\mathrm{pr}} \\
        \mathrm{point} \arrow{r} & X/A
    \end{tikzcd}
    \]
\end{center}

\begin{remark}[Об обозначениях]
	Объяснение диаграммы.
	\begin{itemize}
		\item \textbf{Верхняя стрелка}: $ A \hookrightarrow X $ обозначает вложение подмножества $ A $ в пространство $ X $;
		
		\item \textbf{Левая вертикальная стрелка}: $ A \xrightarrow{\sim} \mathrm{point} $ показывает, что все точки множества $ A $ отождествляются в одну точку ($\mathrm{point}$);
		
		\item \textbf{Правая вертикальная стрелка}: $ X \xrightarrow{\mathrm{pr}} X/A $ — это естественная проекция пространства $ X $ на фактор-пространство $ X/A $. Каждая точка из $ A $ переводится в одну точку в $ X/A $.
		
		\item \textbf{Нижняя стрелка}: $ \mathrm{point} \to X/A $ указывает, что образ $ A $ в $ X/A $ — это одна точка.
		
		\item \textbf{Коммутативность}: Диаграмма коммутативна, так как композиция отображений вдоль двух путей (через $ A $ и через $ X $) приводит к одному и тому же результату в $ X/A $.
	\end{itemize}	
\end{remark}

Перед рассмотрением примера на стягивание подмножества в топологическом пространстве нам необходимо ввести понятие цилиндра. Это базовая конструкция, которая часто используется при построении более сложных топологических объектов.

\begin{definition}[Цилиндр над топологическим пространством]
    Для топологического пространства $ X $ \textbf{цилиндром} называется пространство $ \mathrm{Z}X $, строящееся как произведение $ X \times [0, 1] $, где $ [0, 1] $ — единичный отрезок с обычной топологией. Формально:
    \[
    \mathrm{Z}X = X \times [0, 1].
    \]
    Топология на $ \mathrm{Z}X $ индуцируется произведением топологий на $ X $ и $ [0, 1] $.
\end{definition}

Теперь, используя понятие цилиндра, мы можем рассмотреть построение конуса как фактор-пространства.

\begin{example}[Построение конуса через стягивание]
    Пусть $ X $ — топологическое пространство, а $ \mathrm{Z}X = X \times [0, 1] $ — его цилиндр. \textbf{Конусом} над пространством $ X $ называется фактор-пространство $ \mathrm{C}X $, полученное из цилиндра $ \mathrm{Z}X $ путём стягивания подмножества $ X \times \{1\} $ в одну точку. Формально:
    \[
    \mathrm{C}X = \mathrm{Z}X / (X \times \{1\}),
    \]
    где $ X \times \{1\} $ отождествляется в одну точку.

    Интуитивно, конус можно представить как "сжатие" верхнего основания цилиндра $ X \times \{1\} $ в одну точку, в то время как нижнее основание $ X \times \{0\} $ остаётся неизменным.
	Чтобы наглядно проиллюстрировать этот процесс, приведём рисунок:

	
		\begin{center}
			\begin{tikzpicture}[scale=1.3]
				\draw[dashed] (1, 0) arc (0:180:1 and 0.3);
				\draw[thick] (1, 0) arc (360:180:1 and 0.3);
				\draw[very thick] (0, 3) ellipse (1 and 0.3); 
				\fill[pattern=north east lines] (0, 3) ellipse (1 and 0.3); 
				\draw[thick] (-1, 0) -- (-1, 3); 
				\draw[thick] (1, 0) -- (1, 3);
		
				\node at (0, -0.5) {$X \times \{0\}$};
				\node at (0, 3.5) {$X \times \{1\}$};
		
				\draw[->, thick] (2.3, 1.5) -- node[above] {$\sim$} (4, 1.5);
		
				\begin{scope}[shift={(6, 0)}]
				   \draw[dashed] (1, 0) arc (0:180:1 and 0.3); 
				   \draw[thick] (1, 0) arc (360:180:1 and 0.3);
					\draw[thick] (-1, 0) -- (0, 3); 
					\draw[thick] (1, 0) -- (0, 3);
					\fill[black] (0, 3) circle (1pt); 
		
					\node at (0, -0.5) {$X \times \{0\}$};
					\node at (0, 3.3) {$\mathrm{point}$};
				\end{scope}
		   \end{tikzpicture}
		\end{center}
		
	

\end{example}

\begin{statement}
	Факторпространство $ I / [0 \sim 1] $, где $ I = [0, 1] $ — единичный отрезок, а точки $ 0 $ и $ 1 $ отождествлены, гомеоморфно окружности $ S^1 $.
	\end{statement}
	
	\begin{proof}
	Рассмотрим отображение $ f: [0, 1] \to S^1 $, заданное формулой:
	\[
	f(t) = (\cos 2\pi t, \sin 2\pi t), \quad t \in [0, 1].
	\]
	Это отображение переводит отрезок $ [0, 1] $ в окружность $ S^1 $, обходя её один раз против часовой стрелки. Заметим, что $ f(0) = f(1) = (1, 0) $, то есть концы отрезка переходят в одну и ту же точку на окружности.
	
	Разбиение $ S(f) $, индуцированное отображением $ f $, совпадает с заданным отношением эквивалентности $ [0 \sim 1] $. Следовательно, факторотображение $ f/S(f): I / [0 \sim 1] \to S^1 $ является непрерывной биекцией.
	
	Так как $ I / [0 \sim 1] $ компактно (как фактор компактного пространства), а $ S^1 $ хаусдорфово, то любая непрерывная биекция из компактного пространства в хаусдорфово является гомеоморфизмом. Следовательно, $ f/S(f) $ — это гомеоморфизм, и $ I / [0 \sim 1] \cong S^1 $.
	\end{proof}
	
	\begin{center}
		\begin{tikzpicture}[scale=2]
	
			% Отрезок I = [0, 1]
			\draw[thick] (0, 0) -- (2, 0);
			\fill[black] (0, 0) circle (1pt); % Точка 0
			\fill[black] (2, 0) circle (1pt); % Точка 1
	        \draw[<->, thick] (0, 0.15) to[out=45, in=135] (2, 0.15);% Дуга с двумя стрелками

			% Стрелка перехода
			\node at (3, 0) {\scalebox{1.2}{$\cong$}};

			% Окружность S^1
			\begin{scope}[shift={(5, 0)}]
				\draw[thick] (0, 0) circle (1);
				\fill[black] (0, 1) circle (1pt); 
			\end{scope}
	
		\end{tikzpicture}
	\end{center}


\begin{definition}[Приклеивание]
    Пусть \( X \) — топологическое пространство, а \( A \) — другое пространство. \textbf{Приклеиванием} пространства \( A \) к \( X \) по отображению \( f: A \to X \) называется факторизация пространства \( X \sqcup A \) по отношению эквивалентности, отождествляющему каждую точку \( a \in A \) с её образом \( f(a) \) в \( X \). Обозначается как \( X \cup_f A \).
\end{definition}

\begin{center}
    \[
    \begin{tikzcd}[row sep=60pt, column sep=60pt, scale=1]
        A \arrow{r}{f} \arrow[d, hook] & X \arrow{d}{\mathrm{incl}} \\
        X \sqcup A \arrow{r}{\mathrm{pr}} & X \cup_f A
    \end{tikzcd}
    \]
\end{center}
\begin{remark}[Об обозначениях]
	Объяснение диаграммы.
	\begin{itemize}
		\item \textbf{Верхняя стрелка}: $ A \xrightarrow{f} X $ обозначает непрерывное отображение $ f $, которое каждую точку $ a \in A $ переводит в её образ $ f(a) \in X $;
		
		\item \textbf{Левая вертикальная стрелка}: $ A \hookrightarrow X \sqcup A $ обозначает каноническое вложение подмножества $ A $ в дизъюнктное объединение $ X \sqcup A $;
		
		\item \textbf{Правая вертикальная стрелка}: $ X \xrightarrow{\mathrm{incl}} X \cup_f A $ обозначает естественное вложение пространства $ X $ в результирующее пространство $ X \cup_f A $;
		
		\item \textbf{Нижняя стрелка}: $ X \sqcup A \xrightarrow{\mathrm{pr}} X \cup_f A $ — это проекция дизъюнктного объединения $ X \sqcup A $ на фактор-пространство $ X \cup_f A $. Проекция $\mathrm{pr}$ отождествляет каждую точку $ a \in A $ с её образом $ f(a) \in X $;
		
		\item \textbf{Коммутативность}: Диаграмма коммутативна, так как композиция отображений вдоль двух путей (через $ A $ и через $ X \sqcup A $) приводит к одному и тому же результату в $ X \cup_f A $.
	\end{itemize}
\end{remark}

\begin{example}[Приклеивание пространств для построения букета]
	Рассмотрим два топологических пространства $ X $ и $ Y $. Пусть $ x_0 \in X $ и $ y_0 \in Y $ — выделенные точки в этих пространствах. \textit{Букетом} $ X \vee Y $ называется факторпространство, полученное из дизъюнктного объединения $ X \sqcup Y $ путем отождествления точек $ x_0 $ и $ y_0 $:
	\[
	X \vee Y = (X \sqcup Y) / (x_0 \sim y_0).
	\]
	
	Это означает, что точки $ x_0 \in X $ и $ y_0 \in Y $ склеиваются в одну общую точку, которую мы будем обозначать как $ p $. Таким образом, букет $ X \vee Y $ можно представить как пространство, состоящее из $ X $ и $ Y $, соединенных в одну точку $ p $.

	\begin{center}
		\begin{tikzpicture}[scale=1.5]
	
			% Пространство X (например, окружность)
			\draw[thick] (0, 0) circle (1);
			\fill[black] (0, 1) circle (1pt); % Выделенная точка x_0
			\node at (0, -1.3) {$X$};
			\node at (0, 1.3) {$x_0$};
	
			% Пространство Y (например, окружность)
			\draw[thick] (3, 0) circle (1);
			\fill[black] (3, 1) circle (1pt); % Выделенная точка y_0
			\node at (3, -1.3) {$Y$};
			\node at (3, 1.3) {$y_0$};
	
			% Стрелка перехода
			\draw[->, thick] (1.2, 1) -- node[above] {$\sim$} (1.8, 1);
	
			% Букет пространств
			\begin{scope}[shift={(6, 0)}]
				% Пространство X
				\draw[thick] (0, 0) circle (1);
				% Пространство Y
				\draw[thick] (2, 0) circle (1);
				% Общая точка склейки
				\fill[black] (1, 0) circle (1pt);
				\node at (1, -1.3) {$X \vee Y$};
				\node[left] at (1, 0) {$p$};
			\end{scope}
	
		\end{tikzpicture}
		
	\end{center}
	\end{example}
	
	
% \begin{example}
% 	\begin{enumerate}
% 		\item Букет
% 		\item Цилиндр 
% 		\end{enumerate}
% \end{example}
\subsection{Действие группы}


Для точки $x\in X$ множество $G(x) = \{g(x)\ |\ g\in G\}$ называют орбитой точки $x$ под действием группы $G.$
Согласно задаче (тут надо на неё послать), орбиты задают разбиение множества $X.$ \


Лемма 1 ( правильная).  Пусть $X$ - хаусдорфово пространство, $K\subset X$ - компакт, и $x\in X\setminus K.$ Тогда тогда $x$ и множество $K$ могут быть отделены непересекающимися окрестностями.\\

Док-во. Для всякой точки $k\in K$ существует окрестность $U_{k}$, замыкание которой не содержит точку $x.$ В силу компактности, множество $K$ покрывается конечным числом таких окрестностей; обозначим их
через $U_{1},\ldots, U_{N}$. Положим $W = \bigcup_{i=1}^{N} U_{i}$. Тогда $K\subset W$ и $x\notin Cl(W)$, откуда $W$ и $X\setminus Cl(W)$ и дают искомые непересекающиеся окрестности.\\



Теорема (тоже правильная версия) Пусть компактная группа $G$ действует на хаусдорфовом пространстве $X.$
Тогда пространство орбит $X/G$ хаусдорфово.

Док-во.  Орбита $G(x)$ компактна, поскольку она является образом компакта $G$ при непрерывном отображении
$G \to G\times \{x\} \to G(x).$ Если $G(x)$ и $G(y)$ - различные орбиты, то по предыдущей лемме точка $x$ имеет окрестность $U$, замыкание которой не пересекается с $G(y).$
Тогда множества $p(U)$ и $(X/G) \setminus p(U)$ задают непересекающиеся окрестности точек $G(x)$ и $G(y)$ в пространстве $X/G.$

% \begin{definition}
% 	Пусть \( G \) -- группа и \( X \) -- множество. Действием группы \( G \) на множество \( X \) называется отображение \( \cdot: G \times X \rightarrow X \), удовлетворяющее следующим свойствам:
% \begin{enumerate}
%     \item Для всех \( x \in X \) выполнено \( e \cdot x = x \), где \( e \) -- единичный элемент группы \( G \).
%     \item Для всех \( g, h \in G \) и \( x \in X \) выполнено \( (gh) \cdot x = g \cdot (h \cdot x) \), где \( gh \) -- произведение элементов группы \( G \).
% \end{enumerate}
% \end{definition}

\begin{definition}[Топологическая группа]
	\textbf{Топологическая группа} $ G $ — это множество, наделенное структурой группы и хаусдорфовое топологическое пространство, такое что выполняются следующие условия:
	\begin{enumerate}
		\item Отображение умножения $ m: G \times G \to G $, заданное формулой $ m(g_1, g_2) = g_1 g_2 $, является непрерывным.
		\[
		\begin{tikzcd}[row sep=40pt, column sep=60pt]
			G \times G \arrow{r}{m} \arrow{d}{\mathrm{id} \times \mathrm{id}} & G \\
			G \times G \arrow{ur}{m}
		\end{tikzcd}
		\]
		\item Отображение взятия обратного элемента $ i: G \to G $, заданное формулой $ i(g) = g^{-1} $, является непрерывным.
		\[
		\begin{tikzcd}[row sep=40pt, column sep=60pt]
			G \arrow{r}{i} \arrow{d}{\mathrm{id}} & G \\
			G \arrow{ur}{i}
		\end{tikzcd}
		\]
	\end{enumerate}
	\end{definition}
	Когда изучается теория групп, часто говорят, например, что группа $ U(1) $ представляет собой вращения на плоскости. Однако важно понимать, что такие утверждения подразумевают не просто саму группу, а её {действие} на некотором пространстве. Действие группы формализует, как элементы группы преобразуют точки пространства, и является ключевым инструментом для изучения симметрий.

\begin{definition}[Действие группы]
Пусть $ G $ — группа, а $ X $ — множество. \textbf{Действием группы $ G $ на множестве $ X $} называется отображение
\[
\cdot : G \times X \to X, \quad (g, x) \mapsto g \cdot x,
\]
удовлетворяющее следующим условиям:
\begin{enumerate}
    \item Для любого $ x \in X $ выполнено $ e \cdot x = x $, где $ e $ — единичный элемент группы $ G $.
    \item Для любых $ g_1, g_2 \in G $ и $ x \in X $ выполнено $ g_1 \cdot (g_2 \cdot x) = (g_1 g_2) \cdot x $.
\end{enumerate}
\end{definition}
\begin{remark}
	Действие группы позволяет связать абстрактные элементы группы с конкретными преобразованиями множества $ X $. Например, группа $ U(1) $, представляющая собой комплексные числа модуля 1, действует на плоскости $ \mathbb{C} $ через умножение, что соответствует вращениям вокруг начала координат.
	\end{remark}
	Понятие орбиты точки играет ключевую роль в изучении действий групп на множествах. Оно позволяет понять, как группа преобразует пространство, и выделить подмножества точек, связанных между собой через действие группы. Это важно в различных областях математики, таких как теория групп, топология и геометрия.

	\begin{definition}[Орбита точки]
	Пусть $ G $ — группа, действующая на топологическом пространстве $ X $. Для произвольной точки $ x \in X $ множество
	\[
	G(x) = \{g(x) \mid g \in G\}
	\]
	называется \textbf{орбитой точки $ x $} под действием группы $ G $.
	\end{definition}

	Обозначим через $ X/G $ факторпространство $ X / \sim_G $, где отношение эквивалентности $ \sim_G $ задается следующим образом:
	\[
	x \sim_G y \quad \text{тогда и только тогда, когда} \quad x = g(y) \quad \text{для некоторого } g \in G.
	\]
	Таким образом, $ X/G $ — это пространство орбит, то есть множество всех классов эквивалентности точек из $ X $ относительно действия группы $ G $.
	
	Через $ p: X \to X/G $ будем обозначать соответствующее \textit{факторотображение}, которое каждой точке $ x \in X $ сопоставляет её орбиту $ G(x) $. Формально:
	\(
	p(x) = G(x) = \{g(x) \mid g \in G\}.
	\)
	\[
\begin{tikzcd}[row sep=40pt, column sep=60pt]
    G \times X \arrow{r}{\cdot} \arrow{d}{\mathrm{pr}_2} & X \arrow{d}{p} \\
    X \arrow{r}{p} & X/G
\end{tikzcd}
\]
	
\begin{example}[Примеры действий групп]
	Рассмотрим несколько важных примеров действий групп на множествах:
	
	\begin{enumerate}
		\item \textbf{Действие группы $ \mathbb{Z} $ на себе умножением:}
		Группа целых чисел $ \mathbb{Z} $ действует на множестве целых чисел $ \mathbb{Z} $ по правилу:
		\[
		\mathbb{Z} \curvearrowright \mathbb{Z}, \quad n \cdot m = n \cdot m, \quad \forall n, m \in \mathbb{Z}.
		\]
		Это действие сохраняет структуру группы и является тривиальным примером действия группы на себе.
	
		\item \textbf{Действие группы вращений $ \mathrm{SO}(2) $ на плоскости $ \mathbb{R}^2 $:}
		Группа специальных ортогональных преобразований $ \mathrm{SO}(2) $ (вращений) действует на плоскости $ \mathbb{R}^2 $. Каждое вращение $ R_\theta \in \mathrm{SO}(2) $, задаваемое углом $ \theta $, применяется к точке $ (x, y) \in \mathbb{R}^2 $ по формуле:
		\[
		\mathrm{SO}(2) \curvearrowright \mathbb{R}^2, \quad R_\theta \cdot (x, y) = (x', y'), \quad \text{где } 
		\begin{cases}
		x' = x \cos \theta - y \sin \theta, \\
		y' = x \sin \theta + y \cos \theta.
		\end{cases}
		\]
		Это действие описывает вращение точки вокруг начала координат.
	
		\item \textbf{Действие симметрической группы $ S_n $ на множестве $ \{1, 2, \ldots, n\} $:}
		Группа перестановок $ S_n $ действует на множестве $ \{1, 2, \ldots, n\} $ следующим образом:
		\[
		S_n \curvearrowright \{1, 2, \ldots, n\}, \quad \sigma \cdot i = \sigma(i), \quad \forall \sigma \in S_n, \, i \in \{1, 2, \ldots, n\}.
		\]
		Это действие соответствует перестановке элементов множества.
	
		\item \textbf{Действие классических матричных групп на $ \mathbb{R}^n $:}
		\begin{itemize}
			\item $ \mathrm{GL}_n(\mathbb{R}) $ (общая линейная группа): Действует на $ \mathbb{R}^n $ умножением матриц на векторы:
			\[
			\mathrm{GL}_n(\mathbb{R}) \curvearrowright \mathbb{R}^n.
			\]
			\item $ \mathrm{SL}_n(\mathbb{R}) $ (специальная линейная группа): Действует аналогично $ \mathrm{GL}_n(\mathbb{R}) $, но сохраняет ориентацию пространства.
			\item $ \mathrm{O}_n(\mathbb{R}) $ (ортогональная группа): Действует на $ \mathbb{R}^n $ изометриями, сохраняющими длины векторов:
			\[
			\mathrm{O}_n(\mathbb{R}) \curvearrowright \mathbb{R}^n.
			\]
			\item $ \mathrm{SO}_n(\mathbb{R}) $ (специальная ортогональная группа): Действует как $ \mathrm{O}_n(\mathbb{R}) $, но также сохраняет ориентацию.
		\end{itemize}
	
		\item \textbf{Действие унитарных групп на $ \mathbb{C}^n $:}
		\begin{itemize}
			\item $ \mathrm{U}_n(\mathbb{C}) $ (унитарная группа): Действует на $ \mathbb{C}^n $ унитарными преобразованиями, сохраняющими эрмитово скалярное произведение:
			\[
			\mathrm{U}_n(\mathbb{C}) \curvearrowright \mathbb{C}^n.
			\]
			\item $ \mathrm{SU}_n(\mathbb{C}) $ (специальная унитарная группа): Действует аналогично $ \mathrm{U}_n(\mathbb{C}) $, но с дополнительным условием сохранения определителя равным 1.
		\end{itemize}
	
		\item \textbf{Действие симметрической группы $ S_n $ на множестве подмножеств:}
		Группа $ S_n $ также может действовать на множестве всех подмножеств множества $ \{1, 2, \ldots, n\} $, переставляя элементы каждого подмножества согласно перестановке $ \sigma \in S_n $:
		\[
		S_n \curvearrowright \mathcal{P}(\{1, 2, \ldots, n\}),
		\]
		где $ \mathcal{P}(\{1, 2, \ldots, n\}) $ — множество всех подмножеств множества $ \{1, 2, \ldots, n\} $.
	
	\end{enumerate}
	\end{example}


% У нас есть: $G$ -- топологическая группа (то есть, топологическое пространство и непрерывность операций), $X$ -- топологическое пространство и пусть $G \curvearrowright X$  -- непрерывно. Тогда $X/G = X/\sim_G \ x \sim y \Leftrightarrow x = g(y)$, где $X$ -- пространство фбит.


\begin{lemma}
	Пусть $ X $ — хаусдорфово пространство, $ K \subset X $ — компактное подмножество, и $ x \in X \setminus K $. Тогда точка $ x $ и множество $ K $ могут быть отделены непересекающимися окрестностями.
	\end{lemma}
	
\begin{proof}
Для каждой точки $ k \in K $, поскольку $ X $ является хаусдорфовым пространством, существуют непересекающиеся окрестности $ U_k $ точки $ k $ и $ V_k $ точки $ x $. Заметим, что замыкание $ \overline{U_k} $ не содержит точку $ x $, так как $ V_k \cap \overline{U_k} = \emptyset $.

Таким образом, семейство открытых множеств $ \{U_k\}_{k \in K} $ образует открытое покрытие компактного множества $ K $. В силу компактности $ K $, существует конечное подпокрытие:
\[
K \subset \bigcup_{i=1}^N U_i,
\]
где $ U_1, \ldots, U_N $ — соответствующие окрестности точек из $ K $.

Определим множество
\[
W = \bigcup_{i=1}^N U_i.
\]
Тогда $ W $ является окрестностью множества $ K $, причем $ x \notin \overline{W} $ (замыкание $ W $), так как $ x \notin \overline{U_i} $ для каждого $ i = 1, \ldots, N $.

Теперь рассмотрим множества $ W $ и $ X \setminus \overline{W} $. Эти множества являются непересекающимися окрестностями множества $ K $ и точки $ x $ соответственно:
\[
K \subset W, \quad x \in X \setminus \overline{W}, \quad W \cap (X \setminus \overline{W}) = \emptyset.
\]

Следовательно, точка $ x $ и множество $ K $ могут быть отделены непересекающимися окрестностями.
\end{proof}

\begin{theorem}[Теорема о хаусдорфовости пространства орбит]
	Пусть компактная группа $ G $ действует на хаусдорфовом пространстве $ X $. Тогда пространство орбит $ X/G $ является хаусдорфовым.
	\end{theorem}
	
	\begin{proof}
	Каждая орбита $ G(x) $ компактна, поскольку она является образом компактного множества $ G $ при непрерывном отображении:
	   \[
	   G \to G \times \{x\} \to G(x).
	   \]
	Пусть $ G(x) $ и $ G(y) $ — различные орбиты в $ X $. По предыдущей лемме точка $ x $ имеет окрестность $ U $, замыкание которой не пересекается с орбитой $ G(y) $:
	   \[
	   \overline{U} \cap G(y) = \emptyset.
	   \]
	Рассмотрим образ $ p(U) $ окрестности $ U $ при факторотображении $ p: X \to X/G $. Тогда множества $ p(U) $ и $ (X/G) \setminus p(\overline{U}) $ являются непересекающимися окрестностями точек $ G(x) $ и $ G(y) $ в пространстве $ X/G $:
	   \[
	   p(U) \cap \big((X/G) \setminus p(\overline{U})\big) = \emptyset.
	   \]
	\end{proof}

	\begin{remark}[Пример нехаусдорфова факторпространства]
		Факторпространство по действию группы может оказаться нехаусдорфовым. Рассмотрим следующий пример:
		
		Группа $ \mathrm{GL}_2(\mathbb{R}) $ действует на пространстве всех $ 2 \times 2 $-матриц $ M_{2 \times 2}(\mathbb{R}) $ следующим образом:
		\[
		\mathrm{GL}_2(\mathbb{R}) \curvearrowright M_{2 \times 2}(\mathbb{R}), \quad A \cdot X = A X A^{-1}.
		\]
		
		Рассмотрим две матрицы:
		\[
		I = \begin{pmatrix}
		1 & 0 \\
		0 & 1
		\end{pmatrix}, \quad
		T_\lambda = \begin{pmatrix}
		1 & \lambda \\
		0 & 1
		\end{pmatrix}, \quad \lambda \neq 0.
		\]
		
		Заметим, что при $ \lambda \to 0 $ матрица $ T_\lambda $ стремится к единичной матрице $ I $:
		\[
		\lim_{\lambda \to 0} T_\lambda = I.
		\]
		
		Однако орбиты этих матриц под действием группы $ \mathrm{GL}_2(\mathbb{R}) $ различны:
		\[
		\text{Орбита } I: \{A I A^{-1} \mid A \in \mathrm{GL}_2(\mathbb{R})\} = \{I\},
		\]
		\[
		\text{Орбита } T_\lambda: \{A T_\lambda A^{-1} \mid A \in \mathrm{GL}_2(\mathbb{R})\}.
		\]
		
		Несмотря на то, что $ T_\lambda \to I $ при $ \lambda \to 0 $, орбиты этих матриц не могут быть разделены непересекающимися окрестностями в факторпространстве. Это показывает, что факторпространство $ M_{2 \times 2}(\mathbb{R}) / \mathrm{GL}_2(\mathbb{R}) $ не является хаусдорфовым.
		\end{remark}







	

\newpage
\section{Связность и линейная связность}
\subsection{Связность}
% \begin{definition}[Связность]
% 	Топологическое пространство \(X\) называется связным, если его нельзя разбить на два непустых открытых множества:
% 	\begin{equation*}
% 		\forall U, V \open X: \ U \cap V = \varnothing \Rightarrow X \neq U \cup V.
% 	\end{equation*}
% \end{definition}
% \begin{statement}[Эквивалетное определение]
% 	Можно задать связности следующим образом:
% 	\begin{enumerate}
% 		\item \(\forall U, V \closed X: \ U \cap V = \varnothing \Rightarrow X \neq U \cup V\)
% 		\item \(\not\exists U\neq \varnothing\  \text{или} \ U = X : \ U \open X \ \text{и} \ U \closed X\)
% 	\end{enumerate}
% \end{statement}
% \begin{proof}
% 	Упражнение.
% \end{proof}


% \begin{example}
% 	\begin{enumerate}
% 		\item \textbf{Вещественная прямая} \( \mathbb{R} \) с обычной топологией является связным пространством.
		
% 		\item \textbf{Замкнутый интервал} \( [a, b] \) в \(\mathbb{R}\) также является связным пространством.
				
% 		\item \textbf{Окружность} \( S^1 \) в \(\mathbb{R}^2\) с обычной топологией также является связным пространством.
% 	\end{enumerate}
% \end{example}


% \begin{statement}[Непрерывный образ связного пространства]
% 	Пусть \( T_1 \) и \( T_2 \) являются топологическими пространствами и \( S_1 \subseteq T_1 \) является связным. Если \( f : T_1 \rightarrow T_2 \) является непрерывным отображением, тогда образ \( f(S_1) \) является связным.
% \end{statement}
% \begin{proof}
% 	Известно, что \( f : T_1 \rightarrow T_2 \). Пусть \(f(S_1) = U \sqcup V\), тогда \(S_1 = f^{-1}(U) \sqcup f^{-1}(V)\). Противоречие.
% \end{proof}

% \begin{corollary}[о промежуточных значениях]
% 	\(f\) принимает все промежуточные значения между инфимумом и супремумом.
% \end{corollary}



Связность — одно из фундаментальных свойств топологических пространств, которое позволяет описать их "целостность". Интуитивно, связное пространство — это такое пространство, которое нельзя разбить на две непересекающиеся части, каждая из которых является открытой. Это свойство играет ключевую роль в анализе, геометрии и физике, например, при изучении непрерывных процессов или путей в пространстве.

\begin{definition}[Связность]
Топологическое пространство $ X $ называется \textit{связным}, если его нельзя разбить на два непустых открытых множества:
\[
\forall U, V \subset X: \ U \cap V = \varnothing \Rightarrow X \neq U \cup V.
\]
Другими словами, не существует двух непустых открытых множеств $ U $ и $ V $, таких что $ X = U \cup V $ и $ U \cap V = \varnothing $.
\end{definition}

Связность можно определить несколькими эквивалентными способами. Эти формулировки полезны для доказательства различных свойств связных пространств.

\begin{statement}[Эквивалентные определения]
Следующие утверждения эквивалентны:
\begin{enumerate}
    \item Пространство $ X $ нельзя разбить на два непустых замкнутых множества:
    \[
    \forall U, V \closed X: \ U \cap V = \varnothing \Rightarrow X \neq U \cup V.
    \]
    \item Не существует нетривиального подмножества $ U \subset X $, которое одновременно открыто и замкнуто ($ U \neq \varnothing $ и $ U \neq X $).
\end{enumerate}
\end{statement}


Рассмотрим несколько примеров связных пространств, чтобы лучше понять это понятие.

\begin{example}
\begin{enumerate}
    \item \textbf{Вещественная прямая} $ \mathbb{R} $ с обычной топологией является связным пространством. Любая попытка разделить $ \mathbb{R} $ на два непересекающихся открытых множества приводит к противоречию.
    
    \item \textbf{Замкнутый интервал} $ [a, b] \subset \mathbb{R} $ также является связным. Например, если бы $ [a, b] $ можно было разбить на два непересекающихся открытых множества, то существовало бы "разрыв" между ними, что противоречит непрерывности интервала.
    
    \item \textbf{Окружность} $ S^1 \subset \mathbb{R}^2 $ с обычной топологией также является связным пространством. Окружность нельзя разделить на две непересекающиеся открытые части.
\end{enumerate}
\end{example}

Одно из важнейших свойств связности заключается в том, что она сохраняется при непрерывных отображениях. Это свойство широко используется в математическом анализе и топологии.

\begin{statement}[Непрерывный образ связного пространства]
Пусть $ T_1 $ и $ T_2 $ — топологические пространства, и пусть $ S_1 \subseteq T_1 $ является связным. Если $ f : T_1 \to T_2 $ — непрерывное отображение, то образ $ f(S_1) $ также является связным.
\end{statement}

\begin{proof}
Предположим, что $ f(S_1) $ можно разбить на два непересекающихся открытых множества $ U $ и $ V $, то есть $ f(S_1) = U \sqcup V $. Тогда прообразы $ f^{-1}(U) $ и $ f^{-1}(V) $ образуют разбиение $ S_1 $ на два непересекающихся открытых множества, что противоречит связности $ S_1 $.
\end{proof}


Свойство сохранения связности при непрерывных отображениях можно наглядно представить с помощью коммутативной диаграммы:

\[
\begin{tikzcd}[row sep=40pt, column sep=60pt]
    S_1 \arrow{r}{f} \arrow{d}{\text{связное}} & f(S_1) \arrow{d}{\text{связное}} \\
    T_1 \arrow{r}{f} & T_2
\end{tikzcd}
\]


Одним из важных следствий свойства связности является теорема о промежуточных значениях.

\begin{corollary}[Теорема о промежуточных значениях]
Если $ f : [a, b] \to \mathbb{R} $ — непрерывная функция, то $ f $ принимает все значения между $ \inf f $ и $ \sup f $.
\end{corollary}

\begin{proof}
Образ $ f([a, b]) $ связен, так как $ [a, b] $ связен и $ f $ непрерывно. Связное подмножество $ \mathbb{R} $ является интервалом, а значит, $ f $ принимает все промежуточные значения.
\end{proof}


% \begin{definition}[Компоненты связности]
% 	Компоненты связности -- это максимальные по включению связные подмножества пространства \(X\), такие что никакие два из них не пересекаются.
% \end{definition}

% \begin{example}
% 	\begin{enumerate}
% 		\item \textbf{Разрывная прямая:}
		
% 		Рассмотрим топологическое пространство \( X = [0,1] \cup [2,3] \subset \mathbb{R} \) с обычной топологией. Его компонентами связности являются отрезок \( [0,1] \) и отрезок \( [2,3] \).
		
% 		\item \textbf{Кольцо:}
		
% 		Пусть \( X = S^1 \times [0,1] \), где \( S^1 \) -- единичная окружность в плоскости, а \( [0,1] \) -- отрезок с обычной топологией. Тогда компонентами связности пространства \( X \) будут открытые диски, целиком содержащиеся внутри кольца \( S^1 \times (0,1) \).
		
% 		\item \textbf{Бесконечная лесенка:}
		
% 		Рассмотрим бесконечную лесенку \( X \) на плоскости, состоящую из счетного числа ступенек. Компонентами связности этой лесенки будут бесконечные лучи, выходящие из начальной точки и проходящие через каждую ступеньку.
% 	\end{enumerate}
% \end{example}

% \begin{statement}[о разбиение пространсва на компоненты связности]
% 	Каждое пространство разбивается на непересекающиеся компоненты связности.
% \end{statement}
% \begin{proof}
% 	Пусть \(x \in X\), рассмотрим \(Y = \bigcup_{x \in Y_x} Y_x\) -- компоненты связности в \(X\). Пусть \(Y_1, Y_2\) -- две различные компоненты связности, пусть \(x \in Y_1, Y_2\), тогда \(Y_1 \cup Y_2 = Y_1 = Y_2\).
% \end{proof}


% \begin{statement}
% 	Замыкание связного пространства -- связно.
% \end{statement}
% \begin{proof}
% 	Пусть \(Y\) -- связное пространство, тогда рассмотрим \(\mathrm{Cl}(Y)\), пусть \(\mathrm{Cl}(Y) = C \sqcup Z\), тогда \(Y \subseteq C\), следовательно \(\mathrm{Cl}(Y) \subseteq C\), отсюда \(Z = \varnothing\).
% \end{proof}

% \begin{corollary}
% 	Компоненты связности замкнуты.
% \end{corollary}
% \begin{proof}
% 	Пусть \(Y\) -- компонента связности, тогда \(\mathrm{Cl}(Y)\) -- связно, отсюда \(Y \subseteq \mathrm{Cl}(Y)\).
% \end{proof}


Компоненты связности — это максимальные связные подмножества топологического пространства. Они позволяют разбить пространство на "целостные" части, каждая из которых сама по себе является связной. Это свойство особенно полезно при анализе структуры сложных пространств, таких как разрывные множества или многообразия с особенностями.

\begin{definition}[Компоненты связности]
\textbf{Компоненты связности} — это максимальные по включению связные подмножества пространства $ X $, такие что никакие два из них не пересекаются.
\end{definition}

\begin{example}
\begin{enumerate}
    \item \textbf{Разрывная прямая:}
    
    Рассмотрим топологическое пространство $ X = [0,1] \cup [2,3] \subset \mathbb{R} $ с обычной топологией. Компонентами связности этого пространства являются отрезки $ [0,1] $ и $ [2,3] $. Эти отрезки не пересекаются и не могут быть объединены в одно связное множество.
    
    \begin{center}
        \begin{tikzpicture}[scale=1.5]
            % Оси
            \draw[->] (-0.5, 0) -- (3.5, 0) node[right] {$x$};
            
            % Отрезки
            \draw[thick] (0, 0) -- (1, 0);
            \draw[thick] (2, 0) -- (3, 0);
            
            % Подписи
            \node at (0.5, -0.3) {$[0,1]$};
            \node at (2.5, -0.3) {$[2,3]$};
            
        \end{tikzpicture}
    \end{center}

    \item \textbf{Кольцо:}
    
    Пусть $ X = S^1 \times [0,1] $, где $ S^1 $ — единичная окружность в плоскости, а $ [0,1] $ — отрезок с обычной топологией. Компонентами связности пространства $ X $ будут открытые диски, целиком содержащиеся внутри кольца $ S^1 \times (0,1) $.

    \item \textbf{Бесконечная лесенка:}
    
    Рассмотрим бесконечную лесенку $ X $ на плоскости, состоящую из счетного числа ступенек. Компонентами связности этой лесенки будут бесконечные лучи, выходящие из начальной точки и проходящие через каждую ступеньку.
\end{enumerate}
\end{example}

Одним из ключевых свойств компонент связности является то, что они образуют разбиение пространства.

\begin{statement}[О разбиении пространства на компоненты связности]
Каждое топологическое пространство разбивается на непересекающиеся компоненты связности.
\end{statement}

\begin{proof}
Пусть $ x \in X $. Рассмотрим $ Y_x $ — объединение всех связных подмножеств, содержащих точку $ x $. Покажите, что $ Y_x $ является связным и максимальным по включению. Если две компоненты связности пересекаются, то их объединение также связно, что противоречит максимальности.
\end{proof}


Еще одно важное свойство связности связано с замыканием множеств.

\begin{statement}[Замыкание связного пространства]
Замыкание связного пространства является связным.
\end{statement}

\begin{proof}
Пусть $ Y $ — связное пространство. Предположим, что $ \mathrm{Cl}(Y) $ можно разбить на два непересекающихся замкнутых множества $ C $ и $ Z $. Тогда $ Y \subseteq C $, и следовательно, $ \mathrm{Cl}(Y) \subseteq C $, что означает $ Z = \varnothing $.
\end{proof}


\begin{corollary}
Компоненты связности замкнуты.
\end{corollary}

\begin{proof}
Пусть $ Y $ — компонента связности. Поскольку $ \mathrm{Cl}(Y) $ связно и содержит $ Y $, то $ Y = \mathrm{Cl}(Y) $, то есть $ Y $ замкнуто.
\end{proof}

Свойства компонент связности можно наглядно представить с помощью коммутативной диаграммы:

\[
\begin{tikzcd}[row sep=40pt, column sep=60pt]
    X \arrow{r}{\text{разбиение}} \arrow{d}{\mathrm{Cl}} & \{Y_\alpha\} \arrow[d, closed hook arrow] \\
    \mathrm{Cl}(X) \arrow{r}{\text{разбиение}} & \{\mathrm{Cl}(Y_\alpha)\}
\end{tikzcd}
\]


\subsection{Линейная связность}

Линейная связность — это свойство топологического пространства, которое означает, что любые две точки в пространстве можно соединить непрерывным путем. Это свойство играет ключевую роль в анализе, геометрии и физике, например, при изучении непрерывных процессов или путей в пространстве. Линейно связные пространства часто возникают в задачах, связанных с движением частиц, деформациями объектов или потоками жидкости.

\begin{definition}[Линейная связность]
Топологическое пространство $ X $ называется \textbf{линейно связным}, если для любых двух точек $ x, y \in X $ существует непрерывная функция $ f: [0,1] \to X $, такая что:
\[
f(0) = x \quad \text{и} \quad f(1) = y.
\]
Функция $ f $ называется \textbf{путем}, соединяющим точки $ x $ и $ y $.
\end{definition}


Рассмотрим несколько примеров линейно связных пространств, чтобы лучше понять это понятие.

\begin{example}
\begin{enumerate}
    \item \textbf{Отрезок} $ [a, b] $ на вещественной прямой с обычной топологией является линейно связным. Любой путь между двумя точками можно задать как линейную интерполяцию:
    \[
    f(t) = (1-t)a + tb, \quad t \in [0,1].
    \]

    \item \textbf{Единичный круг} $ S^1 = \{(x, y) \in \mathbb{R}^2 \mid x^2 + y^2 = 1\} $ в плоскости также является линейно связным. Путь между двумя точками можно задать как дугу окружности.

    \item \textbf{Выпуклое множество}:
    Множество $ X \subset \mathbb{R}^n $ называется выпуклым, если для любых двух точек $ x_1, x_2 \in X $ и любого числа $ t \in [0,1] $ точка $ tx_1 + (1-t)x_2 $ также принадлежит множеству $ X $. Выпуклые множества всегда линейно связны, так как путь между любыми двумя точками можно задать как отрезок прямой.

    \begin{center}
		\begin{tikzpicture}[scale=1.5]
			% Выпуклое множество (штриховка)
			\draw[thick, fill=black!10] (0, 0) ellipse (2 and 1);
			
			% Две точки с метками
			\fill[black] (-1, 0.5) circle (1pt) node[above right] {$x_1$};
			\fill[black] (1, -0.3) circle (1pt) node[below left] {$x_2$};
			
			% Путь между точками
			\draw[thick] (-1, 0.5) -- (1, -0.3) node[midway, below left] {$f(t)$};
		\end{tikzpicture}
	\end{center}
\end{enumerate}
\end{example}


\begin{definition}[Компоненты линейной связности]
Компоненты линейной связности — это максимальные по включению линейно связные подмножества топологического пространства.
\end{definition}

Компоненты линейной связности можно рассматривать как "части" пространства, которые можно обойти, не покидая их. Например, если пространство состоит из нескольких разрозненных кусков, каждый кусок будет своей компонентой линейной связности.



Рассмотрим пример так называемого \textbf{топологического косинуса}. Это пространство показывает, что связность и линейная связность — это разные понятия.

\begin{example}[Топологический косинус]
Пусть $ X = \{ (x, \cos(1/x)) \mid x > 0 \} \cup \{ (0, y) \mid -1 \leq y \leq 1 \} $ — подмножество плоскости $ \mathbb{R}^2 $.

\begin{center}
    \begin{tikzpicture}[scale=2]
        % Ось Y
        \draw[->] (0, -1.5) -- (0, 1.5);
        
        % Ось X
        \draw[->] (-0.5, 0) -- (2.5, 0);
        
        % График cos(1/x)
        \draw[thick, domain=0.025:2, samples=500, smooth] plot (\x, {cos(1/\x r)});
        
        % Вертикальный отрезок
        \draw[thick] (0, -1) -- (0, 1);
	\end{tikzpicture}
\end{center}

Покажем, что точка $ (0,0) $ не соединяется путем с другими точками множества $ X $.

\begin{proof}
Пусть $ \gamma : [0,1] \to X $ — путь с началом в $ (0,0) $, то есть $ \gamma(0) = (0,0) $. Рассмотрим множество:
\[
T = \{ t \in [0,1] \mid \gamma(t) = (0,0) \}.
\]
Очевидно, что $ T $ замкнуто в $ [0,1] $, так как прообраз замкнутого множества $ \{(0,0)\} $ под непрерывным отображением $ \gamma $ является замкнутым.

Докажем, что $ T $ также открыто. Пусть $ t_0 \in T $. По определению, $ \gamma(t_0) = (0,0) $. По непрерывности пути $ \gamma $, найдётся $ \delta > 0 $, такое что для всех $ t \in (t_0 - \delta, t_0 + \delta) $ выполнено $ |\gamma(t)| < 1 $. Покажем, что $ \gamma(t) = (0,0) $ для всех $ t \in (t_0 - \delta, t_0 + \delta) $.

Допустим противное: существует $ t_1 \in (t_0 - \delta, t_0 + \delta) $, для которого $ \gamma(t_1) \neq (0,0) $. Тогда $ \gamma(t_1) = (x_1, \cos(1/x_1)) $ для некоторого $ x_1 > 0 $. Обозначим через $ \gamma_1(t) $ первую координату пути $ \gamma(t) $. Так как $ \gamma_1(t_1) > 0 $, по непрерывности найдётся $ t_2 \in [t_0, t_1] $, такое что $ \gamma_1(t_2) = \frac{1}{2\pi n} $, где $ n \in \mathbb{N} $. Тогда:
\[
\gamma(t_2) = \left(\frac{1}{2\pi n}, \cos(2\pi n)\right) = \left(\frac{1}{2\pi n}, 1\right).
\]
Но это противоречит предположению, что $ |\gamma(t)| < 1 $ для всех $ t \in (t_0 - \delta, t_0 + \delta) $. Таким образом, $ \gamma(t) = (0,0) $ для всех $ t \in (t_0 - \delta, t_0 + \delta) $.

Итак, множество $ T $ одновременно открыто и замкнуто в отрезке $ [0,1] $. Поскольку отрезок $ [0,1] $ является связным, то $ T = [0,1] $. Следовательно, путь $ \gamma(t) $ является постоянным и равен $ (0,0) $ для всех $ t \in [0,1] $.

Точка $ (0,0) $ не соединяется путем с другими точками множества $ X $.
\end{proof}
\end{example}

\starsection{Задачи и упражнения}

\begin{task}[Связность произведения пространств]
Докажите, что если $\mathrm{X}$ и $\mathrm{Y}$ — связные топологические пространства, то их произведение $\mathrm{X} \times \mathrm{Y}$ также является связным.
\end{task}

\begin{task}[Связность факторгруппы]
Пусть $\mathrm{H}$ и $\mathrm{G}$ — топологические группы, причем $\mathrm{H} \trianglelefteq \mathrm{G}$. Докажите, что если $\mathrm{H}$ и $\mathrm{G}/\mathrm{H}$ связны, то $\mathrm{G}$ также связно.

\textbf{Пояснение:} 
Используйте свойство фактортопологии. Постройте непрерывное отображение $\mathrm{G} \to \mathrm{G}/\mathrm{H}$ и покажите, что связность $\mathrm{G}/\mathrm{H}$ и $\mathrm{H}$ влечет связность $\mathrm{G}$.
\end{task}

\begin{task}[Общая линейная группа над $\mathbb{R}$]
Рассмотрим общую линейную группу $\mathrm{GL}_n(\mathbb{R})$ как подмножество пространства матриц $\mathrm{M}_{n \times n}(\mathbb{R})$. Покажите, что:
\[
\mathrm{GL}_n(\mathbb{R}) \overset{\det}{\longrightarrow} \mathbb{R} \setminus \{0\}
\]
неограниченно и несвязно.

\textbf{Пояснение:} 
Множество $\mathrm{GL}_n(\mathbb{R})$ состоит из всех невырожденных матриц. Определитель разбивает это множество на две компоненты: $\det(\mathrm{A}) > 0$ и $\det(\mathrm{A}) < 0$. Эти компоненты не пересекаются, что доказывает несвязность.
\end{task}

\begin{task}[Связность общей линейной группы над $\mathbb{C}$]
Докажите, что $\mathrm{GL}_n(\mathbb{C})$ — линейно связное пространство.
\end{task}

\begin{task}[Ортогональная группа $\mathrm{O}_n(\mathbb{R})$]
Докажите, что множество ортогональных матриц $\mathrm{O}_n(\mathbb{R})$:
\begin{enumerate}
	\item Компактно.
	\item Несвязно.
	\item Замкнуто.
\end{enumerate}

\textbf{Пояснение:} 
Компактность следует из ограниченности и замкнутости множества $\mathrm{O}_n(\mathbb{R})$. Несвязность обусловлена тем, что $\mathrm{O}_n(\mathbb{R})$ разбивается на две компоненты: матрицы с $\det(\mathrm{A}) = 1$ и $\det(\mathrm{A}) = -1$.
\end{task}

\begin{task}[Специальная ортогональная группа $\mathrm{SO}_n(\mathbb{R})$]
Докажите, что $\mathrm{SO}_n(\mathbb{R})$ (ортогональные матрицы с $\det(\mathrm{A}) = 1$) является линейно связным подмножеством $\mathrm{O}_n(\mathbb{R})$.
\end{task}

\begin{task}[Компоненты связности $\mathrm{O}_n(\mathbb{R})$]
Докажите, что $\mathrm{O}_n(\mathbb{R}) = \mathrm{SO}_n(\mathbb{R}) \sqcup \mathrm{O}_n^{-}(\mathbb{R})$, где $\mathrm{O}_n^{-}(\mathbb{R})$ — множество ортогональных матриц с $\det(\mathrm{A}) = -1$. Покажите, что каждая компонента линейно связна.
\end{task}

\begin{task}[Специальная линейная группа $\mathrm{SL}_n(\mathbb{R})$]
Докажите, что $\mathrm{SL}_n(\mathbb{R}) = \{\mathrm{A} \in \mathrm{M}_{n \times n}(\mathbb{R}) \mid \det(\mathrm{A}) = 1\}$:
\begin{enumerate}
	\item Линейно связно.
	\item Неограниченно.
	\item Замкнуто.
\end{enumerate}

\textbf{Пояснение:} 
Линейная связность следует из того, что любую матрицу с $\det(\mathrm{A}) = 1$ можно деформировать в единичную матрицу. Неограниченность обусловлена тем, что элементы матрицы могут быть сколь угодно большими. Замкнутость следует из непрерывности определителя.
\end{task}

\newpage

\section{Проективное пространство}
\subsection{Определение через факторизацию}
% \begin{definition}[Проективное пространство] 
% 	Пусть \( V \) -- векторное пространство над полем \( K \). Проективным пространством называется множество всех одномерных подпространств в \( V \), обозначаемое как \( \mathbb{P}(V) \). 
% \end{definition}
% \begin{example}
% 	\begin{enumerate}
% 		\item \textbf{Проективное пространство двумерных подпространств}: Пусть \( V \) -- четырехмерное векторное пространство над полем \( \mathbb{R} \). Тогда множество всех плоскостей в \( V \) является проективным пространством, обозначаемым как \( \mathbb{P}^2(\mathbb{R}) \). Каждая плоскость в \( V \) соответствует одному элементу в \( \mathbb{P}^2(\mathbb{R}) \).
    
% 		\item \textbf{Проективное пространство прямых в трехмерном пространстве}: Пусть \( V \) -- трехмерное векторное пространство над полем \( \mathbb{R} \). Тогда множество всех прямых в \( V \) является проективным пространством, обозначаемым как \( \mathbb{P}^2(\mathbb{R}) \). Каждая прямая в \( V \) соответствует одному элементу в \( \mathbb{P}^2(\mathbb{R}) \).
% 	\end{enumerate}
% \end{example}

% \begin{gather*}
% 	\mathbb{R}/\mathbb{Z} \cong S^1 \\
% 	\mathbb{R}\mathbb{P} = S^n/(\mathbb{Z}/2) = S^n/_{x \sim -x} \\
% 	\mathbb{Z}/2 \curvearrowright S^n \left(a \cdot s = s\right)
% \end{gather*}




Проективное пространство является важным примером в теории многообразий и играет ключевую роль в дальнейшем повествовании. Оно возникает естественным образом при изучении геометрии и топологии, а также находит применение в алгебраической геометрии, физике и других областях.


\begin{definition}[Проективное пространство]
$n$-мерное \textbf{вещественное проективное пространство} $ \mathbb{RP}^n $ определяется как факторпространство сферы $ S^n $ по отношению эквивалентности, которое отождествляет диаметрально противоположные точки:
\[
\mathbb{RP}^n = S^n / \sim, \quad \text{где } x \sim y \iff y = -x.
\]
Эквивалентно, $ \mathbb{RP}^n $ можно определить как множество прямых в $ \mathbb{R}^{n+1} $, проходящих через начало координат:
\[
\mathbb{RP}^n = (\mathbb{R}^{n+1} \setminus \{0\}) / \sim, \quad \text{где } x \sim y \iff y = \lambda x \text{ для некоторого } \lambda \in \mathbb{R} \setminus \{0\}.
\]
\end{definition}


Построение проективного пространства можно наглядно представить с помощью следующей коммутативной диаграммы:

\[
\begin{tikzcd}[row sep=40pt, column sep=60pt]
    S^n \arrow{r}{\mathrm{pr}} \arrow{d}{q} & \mathbb{RP}^n \\
    S^n / \sim \arrow{ur}{\cong}
\end{tikzcd}
\]
\begin{example}
    \begin{enumerate}
        \item \textbf{Одномерное проективное пространство $ \mathbb{RP}^1 $} можно интерпретировать как множество всех прямых в $ \mathbb{R}^2 $, проходящих через начало координат. Каждая прямая определяется своим направлением, и противоположные направления отождествляются. На рисунке ниже различные прямые изображены разными стилями линий, чтобы подчеркнуть их эквивалентность по направлениям.
        \begin{center}
            \begin{tikzpicture}[scale=1.5]
        
                % Оси координат
                \draw[thick, ->] (-2, 0) -- (2, 0) node[right] {$x$};
                \draw[thick, ->] (0, -2) -- (0, 2) node[above] {$y$};
        
                % Примеры прямых через начало координат
                \draw[thick, densely dotted] (-2, 1) -- (2, -1); % Прямая 1 (пунктирная)
                \draw[thick, dashed] (-2, -1) -- (2, 1); % Прямая 2 (штриховая)
                \draw[thick] (-1, -2) -- (1, 2); % Прямая 3 (сплошная)
                \draw[thick, dash dot] (-1, 2) -- (1, -2); % Прямая 4 (штрих-пунктирная)
        
                % Единичная окружность
                \draw[dashed] (0, 0) circle (1);
        
                % Стрелка перехода
                \node at (3, 0) {$\cong$};
        
                % Проективная прямая (окружность)
                \begin{scope}[shift={(5, 0)}]
                    \draw[thick] (0, 0) circle (1);
                    \fill[black] (1, 0) circle (1pt); % Точка для прямой 1
                    \fill[black] (0, 1) circle (1pt); % Точка для прямой 3
                    \node at (0, -1.3) {$\mathbb{RP}^1$};
                \end{scope}
        
            \end{tikzpicture}
        \end{center}
        \item \textbf{Двумерное проективное пространство $ \mathbb{RP}^2 $:}
        Двумерное проективное пространство $ \mathbb{RP}^2 $ можно представить как фактор сферы $ S^2 $, где каждая пара диаметрально противоположных точек отождествлена. Это пространство является примером двумерного многообразия, которое не может быть вложено в $ \mathbb{R}^3 $ без самопересечений.
    \end{enumerate}
\end{example}

\begin{theorem}[Проективное пространство как фактор сферы]
$ \mathbb{RP}^n $ можно представить как фактор двумерной сферы $ S^n $ по действию группы $ \mathbb{Z}_2 $, где элемент $ 1_{\mathbb{Z}_2} $ переводит точку в диаметрально противоположную:
\[
\mathbb{RP}^n \cong S^n / \mathbb{Z}_2.
\]
\end{theorem}

\begin{proof}
Каждая прямая в $ \mathbb{R}^{n+1} $, проходящая через начало координат, соответствует классу эквивалентности точек на сфере $ S^n $. При этом диаметрально противоположные точки на сфере отождествляются, что приводит к изоморфизму $ \mathbb{RP}^n \cong S^n / \mathbb{Z}_2 $.
\end{proof}

\subsection{Однородные координаты}

\begin{definition}[Однородные координаты]
    \textbf{Однородные координаты} — это способ представления точек проективного пространства $ \mathbb{RP}^n $ с помощью наборов чисел $ [x_0 : x_1 : \dots : x_n] $, где $ (x_0, x_1, \dots, x_n) \in \mathbb{R}^{n+1} \setminus \{0\} $. Эти координаты определены с точностью до пропорциональности, то есть:
    \[
    [x_0 : x_1 : \dots : x_n] = [\lambda x_0 : \lambda x_1 : \dots : \lambda x_n], \quad \forall \lambda \in \mathbb{R} \setminus \{0\}.
    \]
    \end{definition}
    
    \begin{remark}[Связь с проективным пространством]
    Однородные координаты естественным образом возникают при построении проективного пространства. Каждая точка $ \mathbb{RP}^n $ соответствует классу эквивалентности ненулевых векторов $ (x_0, x_1, \dots, x_n) \in \mathbb{R}^{n+1} $, где два вектора считаются эквивалентными, если они пропорциональны. Таким образом, однородные координаты предоставляют удобный способ описания точек проективного пространства.
    \end{remark}
    
    \begin{example}[Однородные координаты на проективной прямой $ \mathbb{RP}^1 $]
    Точки проективной прямой $ \mathbb{RP}^1 $ могут быть заданы однородными координатами $ [x_0 : x_1] $, где $ (x_0, x_1) \neq (0, 0) $. Например:
    \[
    [1 : 2] = [2 : 4], \quad [0 : 1] \neq [1 : 0].
    \]
    Здесь $ [1 : 0] $ и $ [0 : 1] $ представляют две "бесконечно удаленные" точки на проективной прямой.
    \end{example}
    
    \begin{statement}[Геометрическая интерпретация]
    Однородные координаты позволяют рассматривать проективное пространство как множество прямых в $ \mathbb{R}^{n+1} $, проходящих через начало координат. Каждая такая прямая задается набором однородных координат $ [x_0 : x_1 : \dots : x_n] $, который определяет направление прямой.
    \end{statement}
    \starsection{Задачи и упражнени}

\begin{task}[Проективное замыкание аффинной прямой]
    Докажите, что проективное замыкание аффинной прямой содержит ровно одну бесконечно удалённую точку.
    
    \textbf{Пояснение:} 
    Аффинная прямая в $\mathbb{R}^2$ задается уравнением $ax + by + c = 0$. При переходе к проективному пространству $\mathbb{RP}^2$, уравнение прямой записывается в однородных координатах как $aX_0 + bX_1 + cX_2 = 0$. Бесконечно удалённые точки соответствуют $X_2 = 0$. Покажите, что существует ровно одна такая точка.
    \end{task}
    
    \begin{task}[Пересечение двух различных проективных прямых]
    Докажите, что две различные проективные прямые в $\mathbb{RP}^2$ пересекаются ровно в одной точке.
    \end{task}
    
    \begin{task}[Уравнение прямой через две точки]
    Найдите уравнение прямой, проходящей через точки $[1 : 0 : -1]$ и $[2 : 1 : 0]$ в $\mathbb{RP}^2$.
    \end{task}
    
    \begin{task}[Точки на проективном замыкании единичной окружности]
    Найдите все точки на проективном замыкании единичной окружности $C : x^2 + y^2 = 1$ над полями $K = \mathbb{F}_3$, $\mathbb{F}_5$ и $\mathbb{F}_7$.
    \end{task}

\end{document}
